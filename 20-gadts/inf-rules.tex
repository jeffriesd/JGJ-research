% For double-blind review submission, w/o CCS and ACM Reference (max
% submission space)
\documentclass[acmsmall,review,anonymous]{acmart}
\settopmatter{printfolios=true,printccs=false,printacmref=false}
%% For double-blind review submission, w/ CCS and ACM Reference
%\documentclass[acmsmall,review,anonymous]{acmart}\settopmatter{printfolios=true}
%% For single-blind review submission, w/o CCS and ACM Reference (max submission space)
%\documentclass[acmsmall,review]{acmart}\settopmatter{printfolios=true,printccs=false,printacmref=false}
%% For single-blind review submission, w/ CCS and ACM Reference
%\documentclass[acmsmall,review]{acmart}\settopmatter{printfolios=true}
%% For final camera-ready submission, w/ required CCS and ACM Reference
%\documentclass[acmsmall]{acmart}\settopmatter{}


%% Journal information
%% Supplied to authors by publisher for camera-ready submission;
%% use defaults for review submission.
\acmJournal{PACMPL}
\acmVolume{1}
\acmNumber{POPL} % CONF = POPL or ICFP or OOPSLA
\acmArticle{1}
\acmYear{2020}
\acmMonth{1}
\acmDOI{} % \acmDOI{10.1145/nnnnnnn.nnnnnnn}
\startPage{1}

%% Copyright information
%% Supplied to authors (based on authors' rights management selection;
%% see authors.acm.org) by publisher for camera-ready submission;
%% use 'none' for review submission.
\setcopyright{none}
%\setcopyright{acmcopyright}
%\setcopyright{acmlicensed}
%\setcopyright{rightsretained}
%\copyrightyear{2018}           %% If different from \acmYear

%% Bibliography style
\bibliographystyle{ACM-Reference-Format}
%% Citation style
%% Note: author/year citations are required for papers published as an
%% issue of PACMPL.
\citestyle{acmauthoryear}   %% For author/year citations
%\citestyle{acmnumeric}

%%%%%%%%%%%%%%%%%%%%%%%%%%%%%%%%%%%%%%%%%%%%%%%%%%%%%%%%%%%%%%%%%%%%%%
%% Note: Authors migrating a paper from PACMPL format to traditional
%% SIGPLAN proceedings format must update the '\documentclass' and
%% topmatter commands above; see 'acmart-sigplanproc-template.tex'.
%%%%%%%%%%%%%%%%%%%%%%%%%%%%%%%%%%%%%%%%%%%%%%%%%%%%%%%%%%%%%%%%%%%%%%



\usepackage[utf8]{inputenc}
\usepackage{ccicons}
\usepackage{verbatim}

\usepackage{amsmath}
\usepackage{amsthm}
\usepackage{amscd}
%\usepackage{MnSymbol}
\usepackage{xcolor}

\usepackage{bbold}
\usepackage{url}
\usepackage{upgreek}
%\usepackage{stmaryrd}

\usepackage{lipsum}
\usepackage{tikz-cd}
\usetikzlibrary{cd}
\usetikzlibrary{calc}
\usetikzlibrary{arrows}

\usepackage{bussproofs}
\EnableBpAbbreviations

\DeclareMathAlphabet{\mathpzc}{OT1}{pzc}{m}{it}

%\usepackage[amsmath]{ntheorem}

\newcommand{\lam}{\lambda}
\newcommand{\eps}{\varepsilon}
\newcommand{\ups}{\upsilon}
\newcommand{\mcB}{\mathcal{B}}
\newcommand{\mcD}{\mathcal{D}}
\newcommand{\mcE}{\mathcal{E}}
\newcommand{\mcF}{\mathcal{F}}
\newcommand{\mcP}{\mathcal{P}}
\newcommand{\mcI}{\mathcal{I}}
\newcommand{\mcJ}{\mathcal{J}}
\newcommand{\mcK}{\mathcal{K}}
\newcommand{\mcL}{\mathcal{L}}
\newcommand{\WW}{\mathcal{W}}

\newcommand{\ex}{\mcE_x}
\newcommand{\ey}{\mcE_y}
\newcommand{\bzero}{\boldsymbol{0}}
\newcommand{\bone}{{\boldsymbol{1}}}
\newcommand{\tB}{{\bone_\mcB}}
\newcommand{\tE}{{\bone_\mcE}}
\newcommand{\bt}{\mathbf{t}}
\newcommand{\bp}{\mathbf{p}}
\newcommand{\bsig}{\mathbf{\Sigma}}
\newcommand{\bpi}{\boldsymbol{\pi}}
\newcommand{\Empty}{\mathtt{Empty}}
\newcommand{\truthf}{\mathtt{t}}
\newcommand{\id}{id}
\newcommand{\coo}{\mathtt{coo\ }}
\newcommand{\mcC}{\mathcal{C}}
\newcommand{\Rec}{\mathpzc{Rec}}
\newcommand{\types}{\mathcal{T}}



%\newcommand{\semof}[1]{\llbracket{#1}\rrbracket^\rel}
\newcommand{\sem}[1]{\llbracket{#1}\rrbracket}
\newcommand{\setsem}[1]{\llbracket{#1}\rrbracket^\set}
\newcommand{\relsem}[1]{\llbracket{#1}\rrbracket^\rel}
\newcommand{\dsem}[1]{\llbracket{#1}\rrbracket^{\mathsf D}}
\newcommand{\setenv}{\mathsf{SetEnv}}
\newcommand{\relenv}{\mathsf{RelEnv}}
\newcommand{\denv}{\mathsf{DEnv}}

\newcommand{\rel}{\mathsf{Rel}}
\newcommand{\setof}[1]{\{#1\}}
\newcommand{\letin}[1]{\texttt{let }#1\texttt{ in }}
\newcommand{\comp}[1]{{\{#1\}}}
\newcommand{\bcomp}[1]{\{\![#1]\!\}}
\newcommand{\beps}{\boldsymbol{\eps}}
%\newcommand{\B}{\mcB}
%\newcommand{\Bo}{{|\mcB|}}

\newcommand{\lmt}{\longmapsto}
\newcommand{\RA}{\Rightarrow}
\newcommand{\LA}{\Leftarrow}
\newcommand{\rras}{\rightrightarrows}
\newcommand{\colim}[2]{{{\underrightarrow{\lim}}_{#1}{#2}}}
\newcommand{\lift}[1]{{#1}\,{\hat{} \; \hat{}}}
\newcommand{\graph}[1]{\langle {#1} \rangle}

\newcommand{\carAT}{\mathsf{car}({\mathcal A}^T)}
\newcommand{\isoAto}{\mathsf{Iso}({\mcA^\to})}
\newcommand{\falg}{\mathsf{Alg}_F}
\newcommand{\CC}{\mathsf{Pres}(\mathcal{A})}
\newcommand{\PP}{\mathcal{P}}
\newcommand{\DD}{D_{(A,B,f)}}
\newcommand{\from}{\leftarrow}
\newcommand{\upset}[1]{{#1}{\uparrow}}
\newcommand{\smupset}[1]{{#1}\!\uparrow}

\newcommand{\Coo}{\mathpzc{Coo}}
\newcommand{\code}{\#}
\newcommand{\nat}{\mathpzc{Nat}}

\newcommand{\eq}{\; = \;}
\newcommand{\of}{\; : \;}
\newcommand{\df}{\; := \;}
\newcommand{\bnf}{\; ::= \;}

\newcommand{\zmap}[1]{{\!{\between\!\!}_{#1}\!}}
\newcommand{\bSet}{\mathbf{Set}}

\newcommand{\dom}{\mathsf{dom}}
\newcommand{\cod}{\mathsf{cod}}
\newcommand{\adjoint}[2]{\mathrel{\mathop{\leftrightarrows}^{#1}_{#2}}}
\newcommand{\isofunc}{\mathpzc{Iso}}
\newcommand{\ebang}{{\eta_!}}
\newcommand{\lras}{\leftrightarrows}
\newcommand{\rlas}{\rightleftarrows}
\newcommand{\then}{\quad\Longrightarrow\quad}
\newcommand{\hookup}{\hookrightarrow}

\newcommand{\spanme}[5]{\begin{CD} #1 @<#2<< #3 @>#4>> #5 \end{CD}}
\newcommand{\spanm}[3]{\begin{CD} #1 @>#2>> #3\end{CD}}
\newcommand{\pushout}{\textsf{Pushout}}
\newcommand{\mospace}{\qquad\qquad\!\!\!\!}

\newcommand{\natur}[2]{#1 \propto #2}

\newcommand{\Tree}{\mathsf{Tree}\,}
\newcommand{\GRose}{\mathsf{GRose}\,}
\newcommand{\List}{\mathsf{List}\,}
\newcommand{\PTree}{\mathsf{PTree}\,}
\newcommand{\Bush}{\mathsf{Bush}\,}
\newcommand{\Forest}{\mathsf{Forest}\,}
\newcommand{\Lam}{\mathsf{Lam}\,}
\newcommand{\LamES}{\mathsf{Lam}^+}
\newcommand{\Expr}{\mathsf{Expr}\,}

\newcommand{\ListNil}{\mathsf{Nil}}
\newcommand{\ListCons}{\mathsf{Cons}}
\newcommand{\LamVar}{\mathsf{Var}}
\newcommand{\LamApp}{\mathsf{App}}
\newcommand{\LamAbs}{\mathsf{Abs}}
\newcommand{\LamSub}{\mathsf{Sub}}
\newcommand{\ExprConst}{\mathsf{Const}}
\newcommand{\ExprPair}{\mathsf{Pair}}
\newcommand{\ExprProj}{\mathsf{Proj}}
\newcommand{\ExprAbs}{\mathsf{Abs}}
\newcommand{\ExprApp}{\mathsf{App}}
\newcommand{\Ptree}{\mathsf{Ptree}}

\newcommand{\kinds}{\mathpzc{K}}
\newcommand{\tvars}{\mathbb{T}}
\newcommand{\fvars}{\mathbb{F}}
\newcommand{\consts}{\mathpzc{C}}
\newcommand{\Lan}{\mathsf{Lan}}
\newcommand{\zerot}{\mathbb{0}}
\newcommand{\onet}{\mathbb{1}}
\newcommand{\bool}{\mathbb{2}}
\renewcommand{\nat}{\mathbb{N}}
%\newcommand{\semof}[1]{[\![#1]\!]}
%\newcommand{\setsem}[1]{\llbracket{#1}\rrbracket^\set}
\newcommand{\predsem}[1]{\llbracket{#1}\rrbracket^\pred}
%\newcommand{\todot}{\stackrel{.}{\to}}
\newcommand{\todot}{\Rightarrow}
\newcommand{\bphi}{{\bm \phi}}

\newcommand{\bm}[1]{\boldsymbol{#1}}

\newcommand{\cL}{\mathcal{L}}
\newcommand{\T}{\mathcal{T}}
\newcommand{\Pos}{P\!}
%\newcommand{\Pos}{\mathcal{P}\!}
\newcommand{\Neg}{\mathcal{N}}
\newcommand{\Hf}{\mathcal{H}}
\newcommand{\V}{\mathbb{V}}
\newcommand{\I}{\mathcal{I}}
\newcommand{\Set}{\mathsf{Set}}
\newcommand{\Nat}{\mathsf{Nat}}
\newcommand{\Homrel}{\mathsf{Hom_{Rel}}}
\newcommand{\CV}{\mathcal{CV}}
\newcommand{\lan}{\mathsf{Lan}}
\newcommand{\Id}{\mathit{Id}}
\newcommand{\mcA}{\mathcal{A}}
\newcommand{\inl}{\mathsf{inl}}
\newcommand{\inr}{\mathsf{inr}}
\newcommand{\case}[3]{\mathsf{case}\,{#1}\,\mathsf{of}\,\{{#2};\,{#3}\}}
\newcommand{\tin}{\mathsf{in}}
\newcommand{\fold}{\mathsf{fold}}
\newcommand{\Eq}{\mathsf{Eq}}
\newcommand{\Hom}{\mathsf{Hom}}
\newcommand{\curry}{\mathsf{curry}}
\newcommand{\uncurry}{\mathsf{uncurry}}
\newcommand{\eval}{\mathsf{eval}}
\newcommand{\apply}{\mathsf{apply}}

\newcommand{\ar}[1]{\##1}
\newcommand{\mcG}{\mathcal{G}}
\newcommand{\mcH}{\mathcal{H}}
\newcommand{\TV}{\mathpzc{V}}

\newcommand{\essim}[1]{\mathsf{EssIm}(#1)}
\newcommand{\hra}{\hookrightarrow}

\newcommand{\ol}[1]{\overline{#1}}
\newcommand{\ul}[1]{\underline{#1}}
\newcommand{\op}{\mathsf{op}}
\newcommand{\trige}{\trianglerighteq}
\newcommand{\trile}{\trianglelefteq}
\newcommand{\LFP}{\mathsf{LFP}}
\newcommand{\LAN}{\mathsf{Lan}}
%\newcommand{\Mu}{{\mu\hskip-4pt\mu}}
\newcommand{\Mu}{{\mu\hskip-5.5pt\mu}}
%\newcommand{\Mu}{\boldsymbol{\upmu}}
\newcommand{\Terms}{\mathpzc{Terms}}
\newcommand{\Ord}{\mathpzc{Ord}}
\newcommand{\Anote}[1]{{\color{blue} {#1}}}
\newcommand{\Pnote}[1]{{\color{red} {#1}}}

\newcommand{\greyout}[1]{{\color{gray} {#1}}}
\newcommand{\ora}[1]{\overrightarrow{#1}}

%\newcommand{\?}{{.\ }}
%\theoremheaderfont{\scshape}
%\theorembodyfont{\normalfont}
%\theoremseparator{.\ \ }
\newtheorem{thm}{Theorem}
\newtheorem{dfn}[thm]{Definition}
\newtheorem{prop}[thm]{Proposition}
\newtheorem{cor}[thm]{Corollary}
\newtheorem{lemma}[thm]{Lemma}
\newtheorem{rmk}[thm]{Remark}
\newtheorem{expl}[thm]{Example}
\newtheorem{notn}[thm]{Notation}
%\theoremstyle{nonumberplain}
%\theoremsymbol{\Box}


\theoremstyle{definition}
\newtheorem{exmpl}{Example}

\renewcommand{\greyout}[1]{} %{{\color{gray} {#1}}} -- toggle to remove greyed text
\newcommand{\new}[1]{\color{blue}#1}

\newcommand{\emptyfun}{{[]}}
\newcommand{\cal}{\mathcal}
%\newcommand{\fold}{\mathit{fold}}
\newcommand{\F}{\mathcal{F}}
\renewcommand{\G}{\mathcal{G}}
\newcommand{\N}{\mathcal{N}}
\newcommand{\E}{\mathcal{E}}
\newcommand{\B}{\mathcal{B}}
\renewcommand{\P}{\mathcal{A}}
\newcommand{\pred}{\mathsf{Fam}}
\newcommand{\env}{\mathsf{Env}}
\newcommand{\set}{\mathsf{Set}}
\renewcommand{\S}{\mathcal S}
\renewcommand{\C}{\mathcal{C}}
\newcommand{\D}{\mathcal{D}}
\newcommand{\A}{\mathcal{A}}
\renewcommand{\id}{\mathit{id}}
\newcommand{\map}{\mathsf{map}}
\newcommand{\pid}{\underline{\mathit{id}}}
\newcommand{\pcirc}{\,\underline{\circ}\,}
\newcommand{\pzero}{\underline{0}}
\newcommand{\pone}{\underline{1}}
\newcommand{\psum}{\,\underline{+}\,}
\newcommand{\pinl}{\underline{\mathit{inL}}\,}
\newcommand{\pinr}{\underline{\mathit{inR}}\,}
\newcommand{\ptimes}{\,\underline{\times}\,}
\newcommand{\ppi}{\underline{\pi_1}}
\newcommand{\pppi}{\underline{\pi_2}}
\newcommand{\pmu}{\underline{\mu}}
\newcommand{\semmap}{\mathit{map}}
\newcommand{\subst}{\mathit{subst}}

\newcommand{\tb}[1]{~~ \mbox{#1} ~~}
\newcommand{\listt}[1]{(\mu \phi. \lambda \beta . \onet + \beta \times
  \phi \beta) #1} 
\newcommand{\filtype}{\Nat^\emptyset 
 (\Nat^\emptyset \, \alpha \, \mathit{Bool})\, (\Nat^\emptyset 
  (List \, \alpha) \, (List \, \alpha))} 
\newcommand{\filtypeGRose}{\Nat^\emptyset 
 (\Nat^\emptyset \, \alpha \, \mathit{Bool})\, (\Nat^\emptyset 
  (\mathit{GRose}\,\psi \, \alpha) \, (\mathit{GRose}\,\psi \, (\alpha
  + \onet)))} 
\newcommand{\maplist}{\mathit{map}_{\lambda A. \setsem{\emptyset; \alpha
      \vdash \mathit{List} \, \alpha} \rho[\alpha := A]}} 
\newcommand{\PLeaves}{\mathsf{PLeaves}}
\newcommand{\swap}{\mathsf{swap}}
\newcommand{\reverse}{\mathsf{reverse}}
\newcommand{\Bcons}{\mathit{Bcons}}
\newcommand{\Bnil}{\mathit{Bnil}}

\title[Practical Parametricity for GADTs]{Practical Parametricity for GADTs}
  %Primitive
  %% [Short Title] is optional; when present,
                         %% will be used in header instead of Full
                         %% Title.
%\titlenote{with title note}             %% \titlenote is optional;
                                        %% can be repeated if necessary;
                                        %% contents suppressed with 'anonymous'
%\subtitle{Subtitle}                     %% \subtitle is optional
%\subtitlenote{with subtitle note}       %% \subtitlenote is optional;
                                        %% can be repeated if necessary;
                                        %% contents suppressed with 'anonymous'


%% Author information
%% Contents and number of authors suppressed with 'anonymous'.
%% Each author should be introduced by \author, followed by
%% \authornote (optional), \orcid (optional), \affiliation, and
%% \email.
%% An author may have multiple affiliations and/or emails; repeat the
%% appropriate command.
%% Many elements are not rendered, but should be provided for metadata
%% extraction tools.

%% Author with single affiliation.
\author{Patricia Johann, Enrico Ghiorzi, and Daniel Jeffries}
%\authornote{with author1 note}          %% \authornote is optional;
%                                        %% can be repeated if necessary
%\orcid{nnnn-nnnn-nnnn-nnnn}             %% \orcid is optional
\affiliation{
%  \position{Position1}
%  \department{Department1}              %% \department is recommended
  \institution{Appalachian State University}            %% \institution is required
%  \streetaddress{Street1 Address1}
%  \city{City1}
%  \state{State1}
%  \postcode{Post-Code1}
%  \country{Country1}                    %% \country is recommended
}
\email{johannp@appstate.edu, ghiorzie@appstate.edu, jeffriesd@appstate.edu}          %% \email is recommended


\begin{document}

\begin{abstract}
Abstract goes here
\end{abstract}

%\begin{CCSXML}
%<ccs2012>
%<concept>
%<concept_id>10011007.10011006.10011008</concept_id>
%<concept_desc>Software and its engineering~General programming languages</concept_desc>
%<concept_significance>500</concept_significance>
%</concept>
%<concept>
%<concept_id>10003456.10003457.10003521.10003525</concept_id>
%<concept_desc>Social and professional topics~History of programming languages</concept_desc>
%<concept_significance>300</concept_significance>
%</concept>
%</ccs2012>
%\end{CCSXML}
%
%\ccsdesc[500]{Software and its engineering~General programming languages}
%\ccsdesc[300]{Social and professional topics~History of programming languages}
%% End of generated code


%% Keywords
%% comma separated list
%\keywords{keyword1, keyword2, keyword3}  %% \keywords is optional


\maketitle

\section{Typing Rules}\label{sec:types}

\begin{dfn}
The formation rules for the set $\F \subseteq \bigcup_{V \subseteq
  \tvars, \Pos\, \subseteq \fvars}\F^\Pos(V)$ of\, {\em well-formed
  types} are

\[\begin{array}{cc}
\AXC{\phantom{$\Gamma,\Phi$}}
\UIC{$\Gamma;\Phi \vdash \zerot$}
\DisplayProof
&
\AXC{\phantom{$\Gamma,\Phi$}}
\UIC{$\Gamma;\Phi \vdash \onet$}
\DisplayProof
\end{array}\]

\vspace*{0.1in}

\[\begin{array}{cc}
\AXC{$\Gamma;\Phi \vdash F$}
\AXC{$\Gamma;\Phi \vdash G$}
\BIC{$\Gamma; \Phi \vdash F + G$}
\DisplayProof
&
\AXC{$\Gamma;\Phi \vdash F$}
\AXC{$\Gamma;\Phi \vdash G$}
\BIC{$\Gamma; \Phi \vdash F \times G$}
\DisplayProof
\end{array}\]

\vspace*{0.05in}

\[\begin{array}{c}
\new{\AXC{$\Gamma;\Phi \vdash F$}
\AXC{$\Gamma;\Phi  \vdash G$}
\BIC{$\Gamma;\emptyset \vdash \Nat^\Phi F \,G$}
\DisplayProof}
\\[4ex]
\AXC{$\phi^k \in \Gamma \cup \Phi$}
\AXC{$\quad\quad\ol{\Gamma;\Phi \vdash F}$}
\BIC{$\Gamma;\Phi \vdash \phi^k \ol{F}$}
\DisplayProof
\\[3ex]
\AXC{$\Gamma;\ol{\gamma^0},\ol{\alpha^0},\phi^k \vdash F$}
\AXC{$\quad\quad\ol{\Gamma;\Phi,\ol{\gamma^0} \vdash G}$}
\BIC{$\Gamma;\Phi,\ol{\gamma^0} \vdash (\mu \phi^k.\lambda
  \ol{\alpha^0}. \,F)\,\ol{G}$} 
\DisplayProof
\\[5ex]
\new{
\AXC{$\Gamma; \Phi,\ol\alpha^0 \vdash F$}
\AXC{$\ol{\Gamma;\ol\alpha^0 \vdash K}$}
\AXC{$\ol{\Gamma;\Phi \vdash A}$}
\TIC{$\Gamma;\Phi \vdash (\Lan^{\ol{\alpha^0}}_{\ol K} F)\, \ol A$}
\DisplayProof}
\end{array}\]
\end{dfn}




\begin{dfn}\label{def:well-formed-terms}
Let $\Delta$ be a term context for $\Gamma$ and $\Phi$.  The formation
rules for the set of\, {\em well-formed terms over $\Delta$} are
\[\begin{array}{ccc}
\AXC{$\Gamma;\Phi \vdash F$}
\UIC{$\Gamma;\Phi \,|\, \Delta,x :F \vdash x : F$}
\DisplayProof
&
\AXC{$\Gamma;\Phi \,|\, \Delta \vdash t : \zerot$}
\AXC{$\Gamma;\Phi \vdash F$}
\BIC{$\Gamma;\Phi \,|\, \Delta \vdash \bot_F t  : F$}
\DisplayProof
&
\AXC{$\phantom{\Gamma;\Phi}$}
\UIC{$\Gamma;\Phi \,|\, \Delta \vdash \top : \onet$}
\DisplayProof\\\\
\end{array}\]
\[\begin{array}{cc}
\AXC{$\Gamma;\Phi \,|\, \Delta \vdash s: F$}
\UIC{$\Gamma;\Phi \,|\, \Delta \vdash \inl \,s: F + G$}
\DisplayProof
&
\AXC{$\Gamma;\Phi \,|\, \Delta \vdash t : G$}
\UIC{$\Gamma;\Phi \,|\, \Delta \vdash \inr \,t: F + G$}
\DisplayProof\\\\
\end{array}\]
\[\begin{array}{c}
\AXC{$\Gamma; \Phi \vdash F,G$}
\AXC{$\Gamma;\Phi \,|\, \Delta \vdash t : F+G$}
\AXC{$\Gamma;\Phi \,|\, \Delta, x : F \vdash l : K \hspace{0.3in} \Gamma;\Phi \,|\, \Delta, y : G \vdash r : K$}
\TIC{$\Gamma;\Phi~|~\Delta \vdash \case{t}{x \mapsto l}{y \mapsto r} : K$}
\DisplayProof
\end{array}\]

\vspace*{0.05in}

\[\begin{array}{lll}
\AXC{$\Gamma;\Phi \,|\, \Delta \vdash s: F$}
\AXC{$\Gamma;\Phi \,|\, \Delta \vdash t : G$}
\BIC{$\Gamma;\Phi \,|\, \Delta \vdash (s,t) : F \times G$}
\DisplayProof
&
\AXC{$\Gamma;\Phi \,|\, \Delta \vdash t : F \times G$}
\UIC{$\Gamma;\Phi \,|\, \Delta \vdash \pi_1 t : F$}
\DisplayProof
&
\AXC{$\Gamma;\Phi \,|\, \Delta \vdash t : F \times G$}
\UIC{$\Gamma;\Phi \,|\, \Delta \vdash \pi_2 t : G$}
\DisplayProof
\end{array}\]

\vspace*{0.05in}

\[\begin{array}{c}
\new{
\AXC{$\Gamma; \Phi \vdash F$}
\AXC{$\Gamma; \Phi \vdash G$}
\AXC{$\Gamma; \Phi \,|\, \Delta, x : F \vdash t: G$} 
\TIC{$\Gamma; \emptyset
  \,|\, \Delta \vdash L_\Phi x.t : \Nat^\Phi \,F \,G$}
\DisplayProof}
\\\\
\new{\AXC{$\ol{\Gamma;\Phi,\ol\beta \vdash K}$}
\AXC{$\Gamma; \emptyset
  \,|\, \Delta \vdash t : \Nat^{\ol\psi} \,F \,G$}
\AXC{$\Gamma;\Phi \,|\, \Delta \vdash s: F[\overline{\psi :=_{\ol\beta} K}]$}
\TIC{$\Gamma;\Phi\,|\, \Delta \vdash t_{\ol K} s:
  G[\overline{\psi :=_{\ol\beta} K}]$}
\DisplayProof}
\\\\
\new{
\AXC{$\Gamma; \Phi, \ol{\phi} \vdash H$}
\AXC{$\ol{\Gamma; \Phi,\ol{\beta} \vdash F}$}
\AXC{$\ol{\Gamma; \Phi,\ol{\beta} \vdash
    G}$}
\TIC{$\Gamma; \emptyset
  ~|~\emptyset
  \vdash \map^{\ol{F},\ol{G}}_H :
  \Nat^\emptyset\;(\ol{\Nat^{\Phi,\ol{\beta}}\,F\,G})\;
  (\Nat^\Phi\,H[\ol{\phi :=_{\ol{\beta}} F}]\;H[\ol{\phi
      :=_{\ol{\beta}} G}])$} 
\DisplayProof}
\\\\
\new{\AXC{$\Gamma; \Phi, \phi, \ol{\alpha} \vdash H$}
\UIC{$\Gamma; \emptyset  \,|\, \emptyset \vdash \tin_H :
  \Nat^{\Phi,\ol{\beta}} H[\phi :=_{\ol{\beta}} (\mu
    \phi.\lambda \ol{\alpha}.H)\ol{\beta}][\ol{\alpha := \beta}]\,(\mu
  \phi.\lambda \ol{\alpha}.H)\ol{\beta}$}
\DisplayProof}
\\\\
\new{\AXC{$\Gamma; \phi,\Phi,\ol{\alpha} \vdash H$}
\AXC{$\Gamma; \Phi, \ol{\beta} \vdash F$}
\BIC{$\Gamma; \emptyset  \,|\, \emptyset \vdash \fold^F_H :
  \Nat^\emptyset\; (\Nat^{\Phi,\ol{\beta}}\,H[\phi
    :=_{\ol{\beta}} F][\ol{\alpha := \beta}]\,F)\; (\Nat^{\Phi,\ol{\beta}}
    \,(\mu \phi.\lambda \ol{\alpha}.H)\ol{\beta}\,F)$}
\DisplayProof}
\\\\
\new{
\AXC{$\Gamma; \Phi,\ol{\alpha}\vdash F$}
\AXC{$\ol{\Gamma; \ol{\alpha} \vdash K}$}
\AXC{$\ol{\Gamma;\Phi \vdash A} \hspace*{0.3in} \Gamma;\Phi~|~\Delta \vdash t : F[\ol{\alpha :=A}]$}
\TIC{$\Gamma;\Phi~|~\Delta \vdash \int_{K,F} t :
  (\Lan^{\ol\alpha}_{\ol K}\,  F)\,\ol{K[\ol{\alpha := A}]}$}  
\DisplayProof}
\\\\
\new{
\AXC{$\Gamma;\emptyset~|~\Delta \vdash \eta : \Nat^{\Phi,\ol\alpha}\,
  F\;G[\ol{\beta := K}]$}
\AXC{$\ol{\Gamma;\Phi \vdash A}$}
\AXC{$\Gamma;\Phi~|~\Delta \vdash t : (\Lan^{\ol\alpha}_{\ol K} F)\,\ol A $}
\TIC{$\Gamma;\Phi~|~\Delta \vdash \partial^{G, \ol K}_F \eta\,t :
  G[\ol{\beta := A}]$}
\DisplayProof}
\end{array}\]
\end{dfn}

\vspace*{0.2in}

Things to check:
\begin{itemize}
\item \new{Are these $\Nat$-types $\omega$CPOs? An argument similar to the
  one in the POPL submission should hold.}
\item \new{Does IEL hold if we allow functoriality in variables of arity $> 0$?}
\item \new{Does AT hold if we allow functoriality in variables of arity $> 0$?}
\end{itemize}



Sum and product intro and elim rules should be annotated with
  constituent types for consistency.

\vspace*{0.2in}

We should have a computation rule along the lines of: If $\eta :
\Nat^{\ol\alpha} F \,G[\ol{\beta := K}]$ then
\[\begin{array}{ll}
& (\partial_F^{G,\ol K} \eta)_{\ol{K[\ol{\alpha := A}]}} \circ (\int_{K,F})_{\ol
  A}\; \rightarrow \; \eta_{\ol A}\\
: & F[\ol{\alpha := A}] \to G[\ol{\beta := K[\ol{\alpha :=A}]}]\\
= & F[\ol{\alpha := A}] \to G[\ol{\beta := K}][\ol{\alpha := A}]
\end{array}\] 



Some questions/issues:
\begin{itemize}
\item Can we write zipBush and appendBush with $\partial$ and $\int$?
  We could already represent the uncurried type of appendBush
  (although not its curried type), but couldn't recurse over both
  input bushes because folds take natural transformations as inputs.
\item More generally, how do we compute with $\partial$ and $\int?$
  Can we use the colimit formulation of Lans (see Lemma 6.3.7
  of~\cite{rie16}) to get a handle on this?
\item What is the connection between exponentials and natural
  transformations? (Should we assume only small objects are
  exponentiable?) Do we want the former or the latter for
  computational purposes? (I suspect the latter.)

  [From nlab:
  In a functor category $D^C$ , a natural transformation $\alpha : F \to
  G$ is exponentiable if (though probably not ``only if'') it is
  cartesian and each component $\alpha_c : F c \to G c$ is
  exponentiable in $D$.  Given $H \to F$ we define $(\Pi_\alpha H) c =
  \Pi_{\alpha_c}(Hc)$; then for $u : c \to c'$ to obtain a map
  $\Pi_{\alpha_c} (Hc) \to \Pi_{\alpha_{c'}} (Hc')$ we need a map
  $\alpha^*_{c'}(\Pi_{\alpha_c}(Hc)) \to Hc'$. But since $\alpha$ is
  cartesian, $\alpha^*_{c'}(\Pi_{\alpha_c}(Hc)) \cong
  \alpha^*_{c}(\Pi_{\alpha_c}(Hc))$, so we have the counit
  $\alpha^*_{c}(\Pi_{\alpha_c}(Hc)) \to Hc$ that we can compose with
  $Hu$.]
\item After we understand what we can do with Lans and folds on GADTs
  we might want to try to extend calculus with term-level
  fixpoints. This would give a categorical analogue for GADTs
  of~\cite{pit98,pit00} for ADTs. Would it also more accurately
  reflect how GADTs are used in practice, or are functions over GADTs
  usually folds? Investigate applications in the literature and/or in
  implementations.
\item What kind of category can handle $\Lan$ types? Can $\omega$CPO?
  This caregory is locally $\omega_1$-presentable (but not locally
  finitely presentable); see 1.18(2) of~\cite{ar94}). The
  $\omega_1$-presentable objects are the $\omega$CPOs cardinality
  less than $\omega_1$, i.e., the countable $\omega$CPOs.
\item $\omega$CPO is a natural choice for modeling general
  recursion. We know $(Lan^\gamma_C \onet)D$ is $C \to D$ for any
  closed type $C$. (Also for select classes of open types?) So can
  model $\Nat \to \gamma$. But the functor $NX = \Nat \to X$ isn't
  $\omega$-cocontinuous. It also doesn't preserve
  $\omega_1$-presentable objects, i.e., countable $\omega$CPOs since
  $\Nat \to \Nat$ is not countable. So we cannot have a functor like
  $N$ as the subscript to $\Lan$ and expect the resulting $\Lan$ to be
  $\omega_1$-cocontinuous.
\item What functors can be subscripts to $\Lan$ and produce
  $\omega_1$-cocontinuous functors? We can use functors that preserve
  presentable objects by theorem in~\cite{jp19}, and possibly others
  as well. These include polynomial functors, ADTs and nested types
  seen as functors, certain (which?) GADTs seen as functors? How big
  can GADTs get? 
\end{itemize}

\bibliography{references}

\end{document}
