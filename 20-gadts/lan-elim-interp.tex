% double-blind review submission, w/o CCS and ACM Reference (max
%
% submission space)
\documentclass[acmsmall,review,anonymous]{acmart}
\settopmatter{printfolios=true,printccs=false,printacmref=false}
%% For double-blind review submission, w/ CCS and ACM Reference
%\documentclass[acmsmall,review,anonymous]{acmart}\settopmatter{printfolios=true}
%% For single-blind review submission, w/o CCS and ACM Reference (max submission space)
%\documentclass[acmsmall,review]{acmart}\settopmatter{printfolios=true,printccs=false,printacmref=false}
%% For single-blind review submission, w/ CCS and ACM Reference
%\documentclass[acmsmall,review]{acmart}\settopmatter{printfolios=true}
%% For final camera-ready submission, w/ required CCS and ACM Reference
%\documentclass[acmsmall]{acmart}\settopmatter{}


%% Journal information
%% Supplied to authors by publisher for camera-ready submission;
%% use defaults for review submission.
\acmJournal{PACMPL}
\acmVolume{1}
\acmNumber{POPL} % CONF = POPL or ICFP or OOPSLA
\acmArticle{1}
\acmYear{2020}
\acmMonth{1}
\acmDOI{} % \acmDOI{10.1145/nnnnnnn.nnnnnnn}
\startPage{1}

%% Copyright information
%% Supplied to authors (based on authors' rights management selection;
%% see authors.acm.org) by publisher for camera-ready submission;
%% use 'none' for review submission.
\setcopyright{none}
%\setcopyright{acmcopyright}
%\setcopyright{acmlicensed}
%\setcopyright{rightsretained}
%\copyrightyear{2018}           %% If different from \acmYear

%% Bibliography style
\bibliographystyle{ACM-Reference-Format}
%% Citation style
%% Note: author/year citations are required for papers published as an
%% issue of PACMPL.
\citestyle{acmauthoryear}   %% For author/year citations
%\citestyle{acmnumeric}

%%%%%%%%%%%%%%%%%%%%%%%%%%%%%%%%%%%%%%%%%%%%%%%%%%%%%%%%%%%%%%%%%%%%%%
%% Note: Authors migrating a paper from PACMPL format to traditional
%% SIGPLAN proceedings format must update the '\documentclass' and
%% topmatter commands above; see 'acmart-sigplanproc-template.tex'.
%%%%%%%%%%%%%%%%%%%%%%%%%%%%%%%%%%%%%%%%%%%%%%%%%%%%%%%%%%%%%%%%%%%%%%



\usepackage[utf8]{inputenc}
\usepackage{ccicons}
\usepackage{verbatim}

\usepackage{amsmath}
\usepackage{amsthm}
\usepackage{amscd}
%\usepackage{MnSymbol}
\usepackage{xcolor}

\usepackage{bbold}
\usepackage{url}
\usepackage{upgreek}
%\usepackage{stmaryrd}

\usepackage{lipsum}
\usepackage{tikz-cd}
\usetikzlibrary{cd}
\usetikzlibrary{calc}
\usetikzlibrary{arrows}

\usepackage{bussproofs}
\EnableBpAbbreviations

\DeclareMathAlphabet{\mathpzc}{OT1}{pzc}{m}{it}

%\usepackage[amsmath]{ntheorem}

\newcommand{\lam}{\lambda}
\newcommand{\eps}{\varepsilon}
\newcommand{\ups}{\upsilon}
\newcommand{\mcB}{\mathcal{B}}
\newcommand{\mcD}{\mathcal{D}}
\newcommand{\mcE}{\mathcal{E}}
\newcommand{\mcF}{\mathcal{F}}
\newcommand{\mcP}{\mathcal{P}}
\newcommand{\mcI}{\mathcal{I}}
\newcommand{\mcJ}{\mathcal{J}}
\newcommand{\mcK}{\mathcal{K}}
\newcommand{\mcL}{\mathcal{L}}
\newcommand{\WW}{\mathcal{W}}

\newcommand{\ex}{\mcE_x}
\newcommand{\ey}{\mcE_y}
\newcommand{\bzero}{\boldsymbol{0}}
\newcommand{\bone}{{\boldsymbol{1}}}
\newcommand{\tB}{{\bone_\mcB}}
\newcommand{\tE}{{\bone_\mcE}}
\newcommand{\bt}{\mathbf{t}}
\newcommand{\bp}{\mathbf{p}}
\newcommand{\bsig}{\mathbf{\Sigma}}
\newcommand{\bpi}{\boldsymbol{\pi}}
\newcommand{\Empty}{\mathtt{Empty}}
\newcommand{\truthf}{\mathtt{t}}
\newcommand{\id}{id}
\newcommand{\coo}{\mathtt{coo\ }}
\newcommand{\mcC}{\mathcal{C}}
\newcommand{\Rec}{\mathpzc{Rec}}
\newcommand{\types}{\mathcal{T}}



%\newcommand{\semof}[1]{\llbracket{#1}\rrbracket^\rel}
\newcommand{\sem}[1]{\llbracket{#1}\rrbracket}
\newcommand{\setsem}[1]{\llbracket{#1}\rrbracket^\set}
\newcommand{\relsem}[1]{\llbracket{#1}\rrbracket^\rel}
\newcommand{\dsem}[1]{\llbracket{#1}\rrbracket^{\mathsf D}}
\newcommand{\setenv}{\mathsf{SetEnv}}
\newcommand{\relenv}{\mathsf{RelEnv}}
\newcommand{\denv}{\mathsf{DEnv}}

\newcommand{\rel}{\mathsf{Rel}}
\newcommand{\setof}[1]{\{#1\}}
\newcommand{\letin}[1]{\texttt{let }#1\texttt{ in }}
\newcommand{\comp}[1]{{\{#1\}}}
\newcommand{\bcomp}[1]{\{\![#1]\!\}}
\newcommand{\beps}{\boldsymbol{\eps}}
%\newcommand{\B}{\mcB}
%\newcommand{\Bo}{{|\mcB|}}

\newcommand{\lmt}{\longmapsto}
\newcommand{\RA}{\Rightarrow}
\newcommand{\LA}{\Leftarrow}
\newcommand{\rras}{\rightrightarrows}
\newcommand{\colim}[2]{{{\underrightarrow{\lim}}_{#1}{#2}}}
\newcommand{\lift}[1]{{#1}\,{\hat{} \; \hat{}}}
\newcommand{\graph}[1]{\langle {#1} \rangle}

\newcommand{\carAT}{\mathsf{car}({\mathcal A}^T)}
\newcommand{\isoAto}{\mathsf{Iso}({\mcA^\to})}
\newcommand{\falg}{\mathsf{Alg}_F}
\newcommand{\CC}{\mathsf{Pres}(\mathcal{A})}
\newcommand{\PP}{\mathcal{P}}
\newcommand{\DD}{D_{(A,B,f)}}
\newcommand{\from}{\leftarrow}
\newcommand{\upset}[1]{{#1}{\uparrow}}
\newcommand{\smupset}[1]{{#1}\!\uparrow}

\newcommand{\Coo}{\mathpzc{Coo}}
\newcommand{\code}{\#}
\newcommand{\nat}{\mathpzc{Nat}}

\newcommand{\eq}{\; = \;}
\newcommand{\of}{\; : \;}
\newcommand{\df}{\; := \;}
\newcommand{\bnf}{\; ::= \;}

\newcommand{\zmap}[1]{{\!{\between\!\!}_{#1}\!}}
\newcommand{\bSet}{\mathbf{Set}}

\newcommand{\dom}{\mathsf{dom}}
\newcommand{\cod}{\mathsf{cod}}
\newcommand{\adjoint}[2]{\mathrel{\mathop{\leftrightarrows}^{#1}_{#2}}}
\newcommand{\isofunc}{\mathpzc{Iso}}
\newcommand{\ebang}{{\eta_!}}
\newcommand{\lras}{\leftrightarrows}
\newcommand{\rlas}{\rightleftarrows}
\newcommand{\then}{\quad\Longrightarrow\quad}
\newcommand{\hookup}{\hookrightarrow}

\newcommand{\spanme}[5]{\begin{CD} #1 @<#2<< #3 @>#4>> #5 \end{CD}}
\newcommand{\spanm}[3]{\begin{CD} #1 @>#2>> #3\end{CD}}
\newcommand{\pushout}{\textsf{Pushout}}
\newcommand{\mospace}{\qquad\qquad\!\!\!\!}

\newcommand{\natur}[2]{#1 \propto #2}

\newcommand{\Tree}{\mathsf{Tree}\,}
\newcommand{\GRose}{\mathsf{GRose}\,}
\newcommand{\List}{\mathsf{List}\,}
\newcommand{\PTree}{\mathsf{PTree}\,}
\newcommand{\Bush}{\mathsf{Bush}\,}
\newcommand{\Forest}{\mathsf{Forest}\,}
\newcommand{\Lam}{\mathsf{Lam}\,}
\newcommand{\LamES}{\mathsf{Lam}^+}
\newcommand{\Expr}{\mathsf{Expr}\,}

\newcommand{\ListNil}{\mathsf{Nil}}
\newcommand{\ListCons}{\mathsf{Cons}}
\newcommand{\LamVar}{\mathsf{Var}}
\newcommand{\LamApp}{\mathsf{App}}
\newcommand{\LamAbs}{\mathsf{Abs}}
\newcommand{\LamSub}{\mathsf{Sub}}
\newcommand{\ExprConst}{\mathsf{Const}}
\newcommand{\ExprPair}{\mathsf{Pair}}
\newcommand{\ExprProj}{\mathsf{Proj}}
\newcommand{\ExprAbs}{\mathsf{Abs}}
\newcommand{\ExprApp}{\mathsf{App}}
\newcommand{\Ptree}{\mathsf{Ptree}}

\newcommand{\kinds}{\mathpzc{K}}
\newcommand{\tvars}{\mathbb{T}}
\newcommand{\fvars}{\mathbb{F}}
\newcommand{\consts}{\mathpzc{C}}
\newcommand{\Lan}{\mathsf{Lan}}
\newcommand{\zerot}{\mathbb{0}}
\newcommand{\onet}{\mathbb{1}}
\newcommand{\bool}{\mathbb{2}}
\renewcommand{\nat}{\mathbb{N}}
%\newcommand{\semof}[1]{[\![#1]\!]}
%\newcommand{\setsem}[1]{\llbracket{#1}\rrbracket^\set}
\newcommand{\predsem}[1]{\llbracket{#1}\rrbracket^\pred}
%\newcommand{\todot}{\stackrel{.}{\to}}
\newcommand{\todot}{\Rightarrow}
\newcommand{\bphi}{{\bm \phi}}

\newcommand{\bm}[1]{\boldsymbol{#1}}

\newcommand{\cL}{\mathcal{L}}
\newcommand{\T}{\mathcal{T}}
\newcommand{\Pos}{P\!}
%\newcommand{\Pos}{\mathcal{P}\!}
\newcommand{\Neg}{\mathcal{N}}
\newcommand{\Hf}{\mathcal{H}}
\newcommand{\V}{\mathbb{V}}
\newcommand{\I}{\mathcal{I}}
\newcommand{\Set}{\mathsf{Set}}
\newcommand{\Nat}{\mathsf{Nat}}
\newcommand{\Homrel}{\mathsf{Hom_{Rel}}}
\newcommand{\CV}{\mathcal{CV}}
\newcommand{\lan}{\mathsf{Lan}}
\newcommand{\Id}{\mathit{Id}}
\newcommand{\mcA}{\mathcal{A}}
\newcommand{\inl}{\mathsf{inl}}
\newcommand{\inr}{\mathsf{inr}}
\newcommand{\case}[3]{\mathsf{case}\,{#1}\,\mathsf{of}\,\{{#2};\,{#3}\}}
\newcommand{\tin}{\mathsf{in}}
\newcommand{\fold}{\mathsf{fold}}
\newcommand{\Eq}{\mathsf{Eq}}
\newcommand{\Hom}{\mathsf{Hom}}
\newcommand{\curry}{\mathsf{curry}}
\newcommand{\uncurry}{\mathsf{uncurry}}
\newcommand{\eval}{\mathsf{eval}}
\newcommand{\apply}{\mathsf{apply}}

\newcommand{\ar}[1]{\##1}
\newcommand{\mcG}{\mathcal{G}}
\newcommand{\mcH}{\mathcal{H}}
\newcommand{\TV}{\mathpzc{V}}

\newcommand{\essim}[1]{\mathsf{EssIm}(#1)}
\newcommand{\hra}{\hookrightarrow}

\newcommand{\ol}[1]{\overline{#1}}
\newcommand{\ul}[1]{\underline{#1}}
\newcommand{\op}{\mathsf{op}}
\newcommand{\trige}{\trianglerighteq}
\newcommand{\trile}{\trianglelefteq}
\newcommand{\LFP}{\mathsf{LFP}}
\newcommand{\LAN}{\mathsf{Lan}}
%\newcommand{\Mu}{{\mu\hskip-4pt\mu}}
\newcommand{\Mu}{{\mu\hskip-5.5pt\mu}}
%\newcommand{\Mu}{\boldsymbol{\upmu}}
\newcommand{\Terms}{\mathpzc{Terms}}
\newcommand{\Ord}{\mathpzc{Ord}}
\newcommand{\Anote}[1]{{\color{blue} {#1}}}
\newcommand{\Pnote}[1]{{\color{red} {#1}}}

\newcommand{\greyout}[1]{{\color{gray} {#1}}}
\newcommand{\ora}[1]{\overrightarrow{#1}}

%\newcommand{\?}{{.\ }}
%\theoremheaderfont{\scshape}
%\theorembodyfont{\normalfont}
%\theoremseparator{.\ \ }
\newtheorem{thm}{Theorem}
\newtheorem{dfn}[thm]{Definition}
\newtheorem{prop}[thm]{Proposition}
\newtheorem{cor}[thm]{Corollary}
\newtheorem{lemma}[thm]{Lemma}
\newtheorem{rmk}[thm]{Remark}
\newtheorem{expl}[thm]{Example}
\newtheorem{notn}[thm]{Notation}
%\theoremstyle{nonumberplain}
%\theoremsymbol{\Box}


\theoremstyle{definition}
\newtheorem{exmpl}{Example}

\renewcommand{\greyout}[1]{} %{{\color{gray} {#1}}} -- toggle to remove greyed text

\newcommand{\ininv}[2]{(\tin^{-1}_{\onet +
  \beta \times \phi (\phi\beta)})_{#1}\, #2}
\newcommand{\emptyfun}{{[]}}
\newcommand{\cal}{\mathcal}
%\newcommand{\fold}{\mathit{fold}}
\newcommand{\F}{\mathcal{F}}
\renewcommand{\G}{\mathcal{G}}
\newcommand{\N}{\mathcal{N}}
\newcommand{\E}{\mathcal{E}}
\newcommand{\B}{\mathcal{B}}
\renewcommand{\P}{\mathcal{A}}
\newcommand{\pred}{\mathsf{Fam}}
\newcommand{\env}{\mathsf{Env}}
\newcommand{\set}{\mathsf{Set}}
\renewcommand{\S}{\mathcal S}
\renewcommand{\C}{\mathcal{C}}
\newcommand{\D}{\mathcal{D}}
\newcommand{\A}{\mathcal{A}}
\renewcommand{\id}{\mathit{id}}
\newcommand{\map}{\mathsf{map}}
\newcommand{\pid}{\underline{\mathit{id}}}
\newcommand{\pcirc}{\,\underline{\circ}\,}
\newcommand{\pzero}{\underline{0}}
\newcommand{\pone}{\underline{1}}
\newcommand{\psum}{\,\underline{+}\,}
\newcommand{\pinl}{\underline{\mathit{inL}}\,}
\newcommand{\pinr}{\underline{\mathit{inR}}\,}
\newcommand{\ptimes}{\,\underline{\times}\,}
\newcommand{\ppi}{\underline{\pi_1}}
\newcommand{\pppi}{\underline{\pi_2}}
\newcommand{\pmu}{\underline{\mu}}
\newcommand{\semmap}{\mathit{map}}
\newcommand{\subst}{\mathit{subst}}

\newcommand{\tb}[1]{~~ \mbox{#1} ~~}
\newcommand{\listt}[1]{(\mu \phi. \lambda \beta . \onet + \beta \times
  \phi \beta) #1} 
\newcommand{\filtype}{\Nat^\emptyset 
 (\Nat^\emptyset \, \alpha \, \mathit{Bool})\, (\Nat^\emptyset 
  (List \, \alpha) \, (List \, \alpha))} 
\newcommand{\filtypeGRose}{\Nat^\emptyset 
 (\Nat^\emptyset \, \alpha \, \mathit{Bool})\, (\Nat^\emptyset 
  (\mathit{GRose}\,\psi \, \alpha) \, (\mathit{GRose}\,\psi \, (\alpha
  + \onet)))} 
\newcommand{\maplist}{\mathit{map}_{\lambda A. \setsem{\emptyset; \alpha
      \vdash \mathit{List} \, \alpha} \rho[\alpha := A]}} 
\newcommand{\PLeaves}{\mathsf{PLeaves}}
\newcommand{\swap}{\mathsf{swap}}
\newcommand{\reverse}{\mathsf{reverse}}
\newcommand{\Bcons}{\mathit{Bcons}}
\newcommand{\Bnil}{\mathit{Bnil}}

\title[Free Theorems for Nested Types]{Free Theorems for
  %Primitive
  Nested Types} %% [Short Title] is optional; when present,
                         %% will be used in header instead of Full
                         %% Title.
%\titlenote{with title note}             %% \titlenote is optional;
                                        %% can be repeated if necessary;
                                        %% contents suppressed with 'anonymous'
%\subtitle{Subtitle}                     %% \subtitle is optional
%\subtitlenote{with subtitle note}       %% \subtitlenote is optional;
                                        %% can be repeated if necessary;
                                        %% contents suppressed with 'anonymous'


%% Author information
%% Contents and number of authors suppressed with 'anonymous'.
%% Each author should be introduced by \author, followed by
%% \authornote (optional), \orcid (optional), \affiliation, and
%% \email.
%% An author may have multiple affiliations and/or emails; repeat the
%% appropriate command.
%% Many elements are not rendered, but should be provided for metadata
%% extraction tools.

%% Author with single affiliation.
\author{Patricia Johann, Enrico Ghiorzi, and Daniel Jeffries}
%\authornote{with author1 note}          %% \authornote is optional;
%                                        %% can be repeated if necessary
%\orcid{nnnn-nnnn-nnnn-nnnn}             %% \orcid is optional
\affiliation{
%  \position{Position1}
%  \department{Department1}              %% \department is recommended
  \institution{Appalachian State University}            %% \institution is required
%  \streetaddress{Street1 Address1}
%  \city{City1}
%  \state{State1}
%  \postcode{Post-Code1}
%  \country{Country1}                    %% \country is recommended
}
\email{johannp@appstate.edu, ghiorzie@appstate.edu, jeffriesd@appstate.edu}          %% \email is recommended


\begin{document}

\section{Lan elimination interpretation is well-defined} 

\[\begin{array}{c}
\AXC{$\Gamma;\emptyset~|~\Delta \vdash \eta : \Nat^{\Phi,\ol\alpha}\,
  F\;G[\ol{\beta := K}]$}
\UIC{$\Gamma;\emptyset~|~\Delta \vdash \partial^{G, \ol K}_F \eta :
\Nat^{\Phi, \ol\beta}\, (\Lan^{\ol\alpha}_{\ol K} F)\ol \beta\; G$}
\DisplayProof
\end{array}\]

\begin{dfn}
  Given $\rho : \setenv$, $d : \setsem{\Gamma; \emptyset \vdash \Delta} \rho$, and
  $\setsem{\Gamma; \emptyset \,|\, \Delta \vdash \eta : \Nat^{\Phi, \ol\alpha} \, F \,\, G[\ol{\beta := K}]} \rho \, d 
    \in \setsem{\Gamma; \emptyset \vdash \Nat^{\Phi, \ol\alpha} \, 
                      F \,\, G[\ol{\beta := K}]} \rho$,
  define \,
  $\setsem{\Gamma; \emptyset \,|\, \Delta \vdash \partial^{G, \ol K}_F \eta :
    \Nat^{\Phi, \ol\beta}\, (\Lan^{\ol\alpha}_{\ol K} F)\ol \beta\; G} \rho :
    \setsem{\Gamma; \emptyset \vdash \Delta} \rho \rightarrow 
        \setsem{\Gamma; \emptyset \vdash 
          \Nat^{\Phi, \ol\beta}\, (\Lan^{\ol\alpha}_{\ol K} F)\ol \beta\; G} \rho$
  as
  $$\lambda \, d
    \, \ol{N}  \, \ol{B}. ((\mu_{\ol{N}})_{\ol{B}})|_{\setsem{\Gamma; \Phi, \ol\beta 
                \vdash (\Lan^{\ol\alpha}_{\ol K} F)\ol \beta } \rho[\ol{\Phi, \beta := N, B}]} 
  $$
  where $\mu_{\ol{N}}$ is the unique natural transformation such that
  $$
  (\setsem{\Gamma; \emptyset \,|\, \Delta \vdash \eta : \Nat^{\Phi, \ol\alpha} \, F \,\, G[\ol{\beta := K}]}\rho \, d)_{\ol{N}}
  = \mu_{\ol{N}} \, \ol{\setsem{\Gamma; \ol\alpha \vdash K} \rho[\ol{\alpha := \_}]} \circ 
  \epsilon_{\ol{N}}
  $$

  where $\epsilon_{\ol{N}}$ is the natural transformation associated to the
  following left Kan extension:
  \[\begin{tikzcd} [column sep = large, row sep = large]
    \set
    \ar[rr, "\setsem{\Gamma; \Phi, \ol\alpha \vdash F}\rho {[\ol{\Phi, \alpha := N, \_} ]}"{name=F}] 
    \ar[dr, "\setsem{\Gamma; \ol\alpha \vdash K}\rho {[\ol{\alpha := \_} ]}"{left}] && \set \\ 
    &\set \ar[ur, "\Lan_{\setsem{\Gamma; \ol\alpha \vdash K}\rho {[\ol{\alpha := \_} ]}}
      \setsem{\Gamma; \Phi, \ol\alpha \vdash F}\rho {[\ol{\Phi, \alpha := N, \_} ]}"{right}]
          \ar[d, Rightarrow, shorten <= 2mm, shorten >= 1mm, 
              from=F, to=2-2,  "\epsilon_{\ol{N}}"{}]
    \end{tikzcd}\]

\end{dfn}
  For the set interpretation of $\Lan$-elimination to be well-defined, 
  we must show that, for any $\rho : \setenv$ and $d : \setsem{\Gamma; \emptyset \vdash \Delta} \rho$, 
  $\setsem{\Gamma; \emptyset \,|\, \Delta \vdash \partial^{G, \ol K}_F \eta :
    \Nat^{\Phi, \ol\beta}\, (\Lan^{\ol\alpha}_{\ol K} F)\ol \beta\; G} \rho \, d$
  has the correct type and satisfies the condition to be in 
  $\setsem{\Gamma; \emptyset \vdash 
    \Nat^{\Phi, \ol\beta}\, (\Lan^{\ol\alpha}_{\ol K} F)\ol \beta\; G} \rho$.
  First we show that all of the components of $\mu_{\ol{N}}$ have the 
  correct type to be a component of a natural transformation in 
  $\setsem{\Gamma; \emptyset \vdash 
    \Nat^{\Phi, \ol\beta}\, (\Lan^{\ol\alpha}_{\ol K} F)\ol \beta\; G} \rho$.
  % type of mu_N
  For any $\ol{N : \set}, \ol{B : \set}$, the type of $(\mu_{\ol{N}})_{\ol{B}}$ is 
  \begin{align*}
    (\mu_{\ol{N}})_{\ol{B}} :  
    (\Lan_{\setsem{\Gamma; \ol\alpha \vdash K}\rho {[\ol{\alpha := \_} ]}}
        \setsem{\Gamma; \Phi, \ol\alpha \vdash F}\rho {[\ol{\Phi, \alpha := N, \_} ]}
        )\ol{B}
  \\
  \Rightarrow 
            \setsem{\Gamma; \Phi, \ol\beta \vdash G} 
              \rho[\ol{\Phi, \beta := N, B}]
  \end{align*}
  % type of restriction
  so the restriction  
  $(\mu_{\ol{N}})_{\ol{B}}  |_{\setsem{\Gamma; \Phi, \ol\beta \,|\, 
  \Delta \vdash (\Lan^{\ol\alpha}_{\ol K} F)\ol \beta } \rho[\ol{\Phi, \beta := N, B}]}$
  has type 
  \begin{align*}
    (\mu_{\ol{N}})_{\ol{B}} |_{\setsem{\Gamma; \Phi, \ol\beta 
              \vdash (\Lan^{\ol\alpha}_{\ol K} F)\ol \beta } \rho[\ol{\Phi, \beta := N, B}]}
    : 
{\setsem{\Gamma; \Phi, \ol\beta 
              \vdash (\Lan^{\ol\alpha}_{\ol K} F)\ol \beta } \rho[\ol{\Phi, \beta := N, B}]}
  \\
  \Rightarrow 
            \setsem{\Gamma; \Phi, \ol\beta \vdash G} 
              \rho[\ol{\Phi, \beta := N, B}]
  \end{align*}

    \noindent
    % thus restricted mu_N has correct type 
    Therefore the (restricted) component
    $(\mu_{\ol{N}})_{\ol{B}} |_{\setsem{\Gamma; \Phi, \ol\beta 
    \vdash (\Lan^{\ol\alpha}_{\ol K} F)\ol \beta } \rho[\ol{\Phi, \beta := N, B}]}$
    has the correct type to be a component of a natural transformation in
    $\setsem{\Gamma; \emptyset \vdash 
      \Nat^{\Phi, \ol\beta}\, (\Lan^{\ol\alpha}_{\ol K} F)\ol \beta\; G} \rho$.
    %%% 


    It remains to be shown that 
    $ \lambda \ol{N} \, \ol{B}. (\mu_{\ol{N}})_{\ol{B}} |_{\setsem{\Gamma; \Phi, \ol\beta 
    \vdash (\Lan^{\ol\alpha}_{\ol K} F)\ol \beta } \rho[\ol{\Phi, \beta := N, B}]}$
    is natural in both $\ol{N}$ and $\ol{B}$
    and that it satisfies the condition to be in 
  $\setsem{\Gamma; \emptyset \vdash 
    \Nat^{\Phi, \ol\beta}\, (\Lan^{\ol\alpha}_{\ol K} F)\ol \beta\; G} \rho$, 
    i.e., 
    % P2 - property for mu_N
    \begin{equation}\label{eq:P2}
    \begin{split}
      &\forall \ol{N = (N^1, N^2, N^*) : RT_k}, 
          \forall \ol{B = (B^1, B^2, B^*) : RT_k}.\\
      &((\mu_{\ol{N^1}})_{\ol{B^1}}|_{\setsem{\Gamma; \Phi, \ol\beta 
    \vdash (\Lan^{\ol\beta}_{\ol K} F)\ol \beta } \rho[\ol{\Phi, \beta := N^1, B^1}]}
      ,
      (\mu_{\ol{N^2}})_{\ol{B^2}}|_{\setsem{\Gamma; \Phi, \ol\beta 
    \vdash (\Lan^{\ol\beta}_{\ol K} F)\ol \beta } \rho[\ol{\Phi, \beta := N^2, B^2}]}
      )\\
      &: \relsem{\Gamma; \Phi, \ol\beta \vdash 
      (\Lan^{\ol\beta}_{\ol K} F)\ol \beta 
      }\Eq_{\rho}[\ol{\Phi, \beta := N, B}]
      \rightarrow \relsem{\Gamma; \Phi, \ol\beta \vdash
        G}\Eq_{\rho}[\ol{\Phi, \beta := N, B}]
    \end{split}
    \end{equation}

    We prove these two remaining facts with the following 
    Lemmas. 

\begin{lemma}
  For every $\rho : \setenv$, $d : \setsem{\Gamma; \emptyset \vdash \Delta} \rho$, 
  $\ol{N = (N^1, N^2, N^*)}$, 
          $\ol{B = (B^1, B^2, B^*)}$
  in $RT_k$ and 
  $(t_1, t_2) \in \relsem{\Gamma; \Phi, \ol\beta \vdash 
        (\Lan^{\ol\beta}_{\ol K} F)\ol \beta 
        }\Eq_{\rho}[\ol{\Phi, \beta := N, B}]$, it holds that
$$((\mu_{\ol{N^1}})_{\ol{B^1}} \,\, t_1, 
                (\mu_{\ol{N^2}})_{\ol{B^2}} \,\, t_2)
                \in \relsem{\Gamma; \Phi, \ol\beta \vdash
            G}\Eq_{\rho}[\ol{\Phi, \beta := N, B}]$$

  where $\mu_{\ol{N}}$ is the unique natural transformation such that
  $$
  (\setsem{\Gamma; \emptyset \,|\, \Delta \vdash \eta : \Nat^{\Phi, \ol\alpha} \, F \,\, G[\ol{\beta := K}]}\rho \, d)_{\ol{N}}
  = \mu_{\ol{N}} \, \ol{\setsem{\Gamma; \ol\alpha \vdash K} \rho[\ol{\alpha := \_}]} \circ 
  \epsilon_{\ol{N}}
  $$
\end{lemma}


%%% goal : 
%%% ((\mu_{\ol{N^1}})_{\ol{B^1}} \,\, t_1, 
%%%                 (\mu_{\ol{N^2}})_{\ol{B^2}} \,\, t_2)
%%%                 \in \relsem{\Gamma; \Phi, \ol\beta \vdash
%%%             G}\Eq_{\rho}[\ol{\Phi, \beta := N, B}]


\begin{proof}
Unfolding the definition of relational intepretation for $\Lan$-types in the hypotheses gives
    $(t_1, t_2) : \setsem{\Gamma;\Phi, \ol\beta \vdash
    (\Lan^{\ol{\alpha}}_{\ol{K}} F)\ol{\beta}}\rho[\ol{\Phi, \beta := N^1, B^1}] \,\times\,
  \setsem{\Gamma;\Phi, \ol\beta \vdash (\Lan^{\ol{\alpha}}_{\ol{K}}
    F)\ol{\beta}}\rho[\ol{\Phi, \beta := N^2, B^2}]$ 
    and there exists $\ol{Z : \set_0}, t_1' : \setsem{\Gamma; \Phi, \ol\alpha \vdash F}
    \rho[\ol{\Phi := N^1}][\ol{\alpha := Z}],\, 
    t_2' :
    \setsem{\Gamma; \Phi, \ol\alpha \vdash F}\rho[\ol{\Phi := N^2 }][\ol{\alpha := Z}]$,\\
  \ol{f :
  \relsem{\Gamma; \ol{\alpha} \vdash K}\Eq_\rho[\ol{\alpha := \Eq_Z}] \to
    B}$ \,
  such that $\iota_{\ol{Z},\ol{\pi_1 f}}t_1' = t_1,\,
  \kappa_{\ol{Z}, \ol{\pi_2 f}}t_2' = t_2,$
  and $(t_1',t_2') \in \\
  \relsem{\Gamma; \Phi, \ol\alpha \vdash F}\Eq_\rho[\ol{\Phi := N}][\ol{\alpha := \Eq_Z}].$
Inlining $\iota_{\ol{Z},\ol{\pi_1 f}}t_1' = t_1$ and $ \kappa_{\ol{Z}, \ol{\pi_2 f}}t_2' = t_2 $
in the statement of the lemma gives 
$$((\mu_{\ol{N^1}})_{\ol{B^1}} \,\, \iota_{\ol{Z},\ol{\pi_1 f}}\, t_1' ,
                (\mu_{\ol{N^2}})_{\ol{B^2}} \,\,  \kappa_{\ol{Z}, \ol{\pi_2 f}}\, t_2'  )
                \in \relsem{\Gamma; \Phi, \ol\beta \vdash
            G}\Eq_{\rho}[\ol{\Phi, \beta := N, B}]$$

% rewrite iota/kappa
Now consider the following commuting diagram: 


  \begin{tikzcd}[column sep = large, row sep = large]
    & \setsem{\Gamma; \Phi, \ol\alpha \vdash F}\rho[\ol{\Phi, \alpha := N^1, Z}]
    \ar[dr, "\iota_{\ol{Z}, \ol{\pi_1 f}}"{right}] 
    \ar[dl, "\iota'_{\ol{\setsem{\Gamma; \ol\alpha \vdash K}\rho[\ol{\alpha := Z}]}, 
                \ol{\id_{\setsem{\Gamma; \ol\alpha \vdash K}\rho[\ol{\alpha := Z}]}}}"{left}] 
        & \\
    LKZ % (\Lan_{\setsem{\Gamma; \ol\alpha \vdash K}\rho{[\ol{\alpha := \_}]}}
    %               \setsem{\Gamma; \Phi, \ol\alpha \vdash F}
    %                 \rho{[\ol{\Phi, \alpha := N^1, \_}]}) \ol{\setsem{\Gamma; \ol\alpha \vdash K}\rho[\ol{\alpha := Z}]}
    \ar[rr, dashed, "(\Lan_{\setsem{\Gamma; \ol\alpha \vdash K}\rho{[\ol{\alpha := \_}]}}
                  \setsem{\Gamma; \Phi, \ol\alpha \vdash F}
                    \rho{[\ol{\Phi, \alpha := N^1, \_}]}) \ol{\pi_1 f}"{below}]
       & & LB
  \end{tikzcd}

  where $LKZ = (\Lan_{\setsem{\Gamma; \ol\alpha \vdash K}\rho{[\ol{\alpha := \_}]}}
    \setsem{\Gamma; \Phi, \ol\alpha \vdash F}
    \rho{[\ol{\Phi, \alpha := N^1, \_}]}) \ol{\setsem{\Gamma; \ol\alpha \vdash K}\rho[\ol{\alpha := Z}]}$
  and \\ 
  $LB = 
    (\Lan_{\setsem{\Gamma; \ol\alpha \vdash K}\rho{[\ol{\alpha := \_}]}}
      \setsem{\Gamma; \Phi, \ol\alpha \vdash F}
      \rho{[\ol{\Phi, \alpha := N^1, \_}]}) \ol{B^1}$. 
  %%
  This diagram commutes because of the universal (colimit) property of $LKZ$. 
  We also have that 
  $\iota'_{\ol{\setsem{\Gamma; \ol\alpha \vdash K}\rho[\ol{\alpha := Z}]}, 
                \ol{\id_{\setsem{\Gamma; \ol\alpha \vdash K}\rho[\ol{\alpha := Z}]}}}
   = (\epsilon_{\ol{N^1}})_{\ol{Z}}$.
  For the details of this colimit construction see Theorem 6.2.2 of \ref{}. 
  Together these two facts give 

  $$
  \iota_{\ol{Z}, \ol{\pi_1 f}}
  = (\Lan_{\setsem{\Gamma; \ol\alpha \vdash K}\rho{[\ol{\alpha := \_}]}}
                  \setsem{\Gamma; \Phi, \ol\alpha \vdash F}
                    \rho{[\ol{\Phi, \alpha := N^1, \_}]}) \ol{\pi_1 f}
  \circ (\epsilon_{\ol{N^1}})_{\ol{Z}}
  $$ \\
  and an analogous proof gives 

  $$
  \kappa_{\ol{Z}, \ol{\pi_2 f}}
  = (\Lan_{\setsem{\Gamma; \ol\alpha \vdash K}\rho{[\ol{\alpha := \_}]}}
                  \setsem{\Gamma; \Phi, \ol\alpha \vdash F}
                    \rho{[\ol{\Phi, \alpha := N^2, \_}]}) \ol{\pi_2 f}
  \circ (\epsilon_{\ol{N^2}})_{\ol{Z}}
  $$ \\
So we can restate the goal of the lemma as:
\begin{align*} 
  ((\mu_{\ol{N^1}})_{\ol{B^1}} \,\, 
  ((\Lan_{\setsem{\Gamma; \ol\alpha \vdash K}\rho{[\ol{\alpha := \_}]}}
                  \setsem{\Gamma; \Phi, \ol\alpha \vdash F}
                    \rho{[\ol{\Phi, \alpha := N^1, \_}]}) \ol{\pi_1 f}
  ((\epsilon_{\ol{N^1}})_{\ol{Z}}
        \, t_1' )) , \\
%%%%%%%%%%%%%%%%%%%%%%%%%%%%%%%%%%%%%%%%%%%%%%%%%%%%%%%
(\mu_{\ol{N^2}})_{\ol{B^2}} \,\,  
  ((\Lan_{\setsem{\Gamma; \ol\alpha \vdash K}\rho{[\ol{\alpha := \_}]}}
                  \setsem{\Gamma; \Phi, \ol\alpha \vdash F}
                    \rho{[\ol{\Phi, \alpha := N^2, \_}]}) \ol{\pi_2 f}
  ((\epsilon_{\ol{N^2}})_{\ol{Z}}
        \, t_2'  ))) \\
        \in \relsem{\Gamma; \Phi, \ol\beta \vdash
    G}\Eq_{\rho}[\ol{\Phi, \beta := N, B}]
\end{align*}
Now consider the naturality of $\mu_{\ol{N^i}}$ for 
the morphism $\ol{\pi_i f}$. The naturality condition is 
\begin{align*}
(\mu_{\ol{N^i}})_{\ol{B^i}} \circ 
  ((\Lan_{\setsem{\Gamma; \ol\alpha \vdash K}\rho{[\ol{\alpha := \_}]}}
                  \setsem{\Gamma; \Phi, \ol\alpha \vdash F}
                    \rho{[\ol{\Phi, \alpha := N^i, \_}]}) \ol{\pi_i f} \\
= \setsem{\Gamma; \Phi, \ol\beta \vdash G}\id_\rho 
    [\ol{\Phi, \beta := \id_{N^i}, \pi_i f}]
  \circ 
  (\mu_{\ol{N^i}})_{\ol{\setsem{\Gamma; \ol\alpha \vdash K}
      \rho[\ol{\alpha := Z}]}}
\end{align*}
So the goal statement can be reformulated once again as
\begin{align*} 
  (\setsem{\Gamma; \Phi, \ol\beta \vdash G}\id_\rho 
    [\ol{\Phi, \beta := \id_{N^1}, \pi_1 f}]
  ((\mu_{\ol{N^1}})_{\ol{\setsem{\Gamma; \ol\alpha \vdash K}
      \rho[\ol{\alpha := Z}]}}
  ((\epsilon_{\ol{N^1}})_{\ol{Z}}
        \, t_1' )) , \\
%%%%%%%%%%%%%%%%%%%%%%%%%%%%%%%%%%%%%%%%%%%%%%%%%%%%%%%
  \setsem{\Gamma; \Phi, \ol\beta \vdash G}\id_\rho 
    [\ol{\Phi, \beta := \id_{N^2}, \pi_2 f}]
  ((\mu_{\ol{N^2}})_{\ol{\setsem{\Gamma; \ol\alpha \vdash K}
      \rho[\ol{\alpha := Z}]}}
  ((\epsilon_{\ol{N^2}})_{\ol{Z}}
        \, t_2'  ))) \\
        \in \relsem{\Gamma; \Phi, \ol\beta \vdash
    G}\Eq_{\rho}[\ol{\Phi, \beta := N, B}]
\end{align*}

To finish the proof, we would like to use that fact that 
$\setsem{\Gamma; \Phi,\ol\beta \vdash G}\id_{\Eq_\rho} [\ol{\Phi, \beta := \id_N, f}]$
is a morphism in $\rel$ and thus preserves related elements. 
In order to apply this morphism, we must first show that 
\begin{align*}
((\mu_{\ol{N^1}})_{\ol{\setsem{\Gamma; \ol\alpha \vdash K} \rho[\ol{\alpha := Z}]}} 
  ((\epsilon_{\ol{N^1}})_{\ol{Z}} \,\, t_1') &, 
  %%%
  (\mu_{\ol{N^2}})_{\ol{\setsem{\Gamma; \ol\alpha \vdash K} \rho[\ol{\alpha := Z}]}} 
  ((\epsilon_{\ol{N^2}})_{\ol{Z}} \,\, t_2')) \\
  & \in 
  \relsem{\Gamma; \Phi, \ol\alpha \vdash
        G[\ol{\beta := K}]}\Eq_{\rho}[\ol{\Phi, \alpha := N, \Eq_Z}]
\end{align*}
To show this, we use the condition imposed on 
$\setsem{\Gamma; \emptyset \,|\, \Delta \vdash \eta : \Nat^{\Phi, \ol\alpha} \, F \,\, G[\ol{\beta := K}]} \rho \, d$ to be 
in $\setsem{\Gamma; \emptyset \vdash \Nat^{\Phi, \ol\alpha} \, F \,\, G[\ol{\beta := K}]} \rho$, i.e.,

    \begin{equation}\label{eq:P1}
    \begin{split}
      &\forall \ol{N = (N^1, N^2, N^*) : RT_k}, 
          \forall \ol{M = (M^1, M^2, M^*) : RT_k}.\\
      &((\setsem{\Gamma; \emptyset \,|\, \Delta \vdash \eta : \Nat^{\Phi, \ol\alpha} \, F \,\, G[\ol{\beta := K}]} \rho \, d)_{\ol{N^1}, \ol{M^1}}, 
      (\setsem{\Gamma; \emptyset \,|\, \Delta \vdash \eta} : \Nat^{\Phi, \ol\alpha} \, F \,\, G[\ol{\beta := K}] \rho \, d)_{\ol{N^2}, \ol{M^2}}) \\
      &: \relsem{\Gamma; \Phi, \ol\alpha \vdash F}\Eq_{\rho}[\ol{\Phi, \alpha := N, M}]
      \rightarrow \relsem{\Gamma; \Phi, \ol\alpha \vdash
        G[\ol{\beta := K}]}\Eq_{\rho}[\ol{\Phi, \alpha := N, M}]
    \end{split}
    \end{equation}
along with the universal property of 
$\Lan_{\setsem{\Gamma; \ol\alpha \vdash K}\rho {[\ol{\alpha := \_} ]}}
      \setsem{\Gamma; \Phi, \ol\alpha \vdash F}\rho {[\ol{\Phi, \alpha := N^i, \_} ]}$
      for $i = 1, 2$
and the fact that $t_1'$ and $t_2'$ are related in 
$\relsem{\Gamma; \Phi, \ol\alpha \vdash F}\Eq_{\rho}[\ol{\Phi, \alpha := N, \Eq_Z}]$.
Applying Equation \eqref{eq:P1} with $\ol{M} = \ol{Z}$ 
gives (eliding the type of $\eta$)
    \begin{equation*}
    \begin{split}
      &((\setsem{\Gamma; \emptyset \,|\, \Delta \vdash \eta} \rho \, d)_{\ol{N^1}, \ol{Z}}, 
      (\setsem{\Gamma; \emptyset \,|\, \Delta \vdash \eta} \rho \, d)_{\ol{N^2}, \ol{Z}}) \\
      &: \relsem{\Gamma; \Phi, \ol\alpha \vdash F}\Eq_{\rho}[\ol{\Phi, \alpha := N, \Eq_Z}]
      \rightarrow \relsem{\Gamma; \Phi, \ol\alpha \vdash
        G[\ol{\beta := K}]}\Eq_{\rho}[\ol{\Phi, \alpha := N, \Eq_Z}]
    \end{split}
    \end{equation*}
and since we know 
$(t_1',t_2') \in
  \relsem{\Gamma; \Phi, \ol\alpha \vdash F}\Eq_\rho[\ol{\Phi := N}][\ol{\alpha := \Eq_Z}]$, 
we can apply this morphism
%%
to $(t_1', t_2')$:
\begin{align*}
  ((\setsem{\Gamma; \emptyset \,|\, \Delta \vdash \eta} \rho \, d)_{\ol{N^1}, \ol{Z}} t_1', 
  (\setsem{\Gamma; \emptyset \,|\, \Delta \vdash \eta} \rho \, d)_{\ol{N^2}, \ol{Z}} t_2') 
  \in 
\relsem{\Gamma; \Phi, \ol\alpha \vdash
        G[\ol{\beta := K}]}\Eq_{\rho}[\ol{\Phi, \alpha := N, \Eq_Z}]
\end{align*}
%%
Applying the universal property at component $\ol{Z}$
$$
  ((\setsem{\Gamma; \emptyset \,|\, \Delta \vdash \eta}\rho \, d)_{\ol{N^i}})_{\ol{Z}}
  = (\mu_{\ol{N^i}})_{\ol{\setsem{\Gamma; \ol\alpha \vdash K} \rho[\ol{\alpha := Z}]}} \circ 
  (\epsilon_{\ol{N^i}})_{\ol{Z}}
$$
yields 
\begin{align*}
((\mu_{\ol{N^1}})_{\ol{\setsem{\Gamma; \ol\alpha \vdash K} \rho[\ol{\alpha := Z}]}} 
  ((\epsilon_{\ol{N^1}})_{\ol{Z}} \,\, t_1') &, 
  %%%
  (\mu_{\ol{N^2}})_{\ol{\setsem{\Gamma; \ol\alpha \vdash K} \rho[\ol{\alpha := Z}]}} 
  ((\epsilon_{\ol{N^2}})_{\ol{Z}} \,\, t_2')) \\
  & \in 
  \relsem{\Gamma; \Phi, \ol\alpha \vdash
        G[\ol{\beta := K}]}\Eq_{\rho}[\ol{\Phi, \alpha := N, \Eq_Z}] \\
  & = \relsem{\Gamma; \Phi,\ol\beta \vdash G}\Eq_\rho
  [\ol{\Phi, \beta := N, \relsem{\Gamma; \ol\alpha \vdash K}\Eq_\rho[\ol{\alpha := \Eq_Z}]}]
\end{align*}


Finally, we can apply the morphism 
\begin{align*}
  \relsem{\Gamma; \Phi,\ol\beta \vdash G}\id_{\Eq_\rho} [\ol{\Phi, \beta := \id_N, f}] 
  : \,\,
  &\relsem{\Gamma; \Phi,\ol\beta \vdash G}\Eq_\rho
  [\ol{\Phi, \beta := N, \relsem{\Gamma; \ol\alpha \vdash K}\Eq_\rho[\ol{\alpha := \Eq_Z}]}] \\
  \rightarrow \,\,
  &\relsem{\Gamma; \Phi,\ol\beta \vdash G}\Eq_\rho [\ol{\Phi, \beta := N, B}]
\end{align*}
to the pair
$((\mu_{\ol{N^1}})_{\ol{\setsem{\Gamma; \ol\alpha \vdash K} \rho[\ol{\alpha := Z}]}} 
  ((\epsilon_{\ol{N^1}})_{\ol{Z}} \,\, t_1') , 
  %%%
  (\mu_{\ol{N^2}})_{\ol{\setsem{\Gamma; \ol\alpha \vdash K} \rho[\ol{\alpha := Z}]}} 
  ((\epsilon_{\ol{N^2}})_{\ol{Z}} \,\, t_2'))$
, which gives 

\begin{align*}
( 
  &\setsem{\Gamma; \Phi,\ol\beta \vdash G}\id_{\rho} [\ol{\Phi, \beta := \id_{N^1}, \pi_1 f}] 
  ((\mu_{\ol{N^1}})_{\ol{\setsem{\Gamma; \ol\alpha \vdash K} \rho[\ol{\alpha := Z}]}} 
  ((\epsilon_{\ol{N^1}})_{\ol{Z}} \,\, t_1')) ,  \\
  %%%
  &\setsem{\Gamma; \Phi,\ol\beta \vdash G}\id_{\rho} [\ol{\Phi, \beta := \id_{N^2}, \pi_2 f}] 
  ((\mu_{\ol{N^2}})_{\ol{\setsem{\Gamma; \ol\alpha \vdash K} \rho[\ol{\alpha := Z}]}} 
  ((\epsilon_{\ol{N^2}})_{\ol{Z}} \,\, t_2'))) \\
  & \in 
  \relsem{\Gamma; \Phi, \ol\alpha \vdash
        G}\Eq_{\rho}[\ol{\Phi, \alpha := N,B}] 
\end{align*}
and thus 
$$((\mu_{\ol{N^1}})_{\ol{B^1}} \,\, t_1, 
                (\mu_{\ol{N^2}})_{\ol{B^2}} \,\, t_2)
                \in \relsem{\Gamma; \Phi, \ol\beta \vdash
            G}\Eq_{\rho}[\ol{\Phi, \beta := N, B}]$$

\end{proof}

\begin{lemma}
  For all $\rho : \setenv$ and $d : \setsem{\Gamma; \emptyset \vdash \Delta} \rho$, 
  the family of morphisms 
  $$
  \lambda \ol{N} \, \ol{B}. 
  ((\mu_{\ol{N}})_{\ol{B}})|_{\setsem{\Gamma; \Phi, \ol\beta 
                  \vdash (\Lan^{\ol\alpha}_{\ol K} F)\ol \beta } \rho[\ol{\Phi, \beta := N, B}]} 
  $$ 
  is natural in $\ol{N}$ and $\ol{B}$, where $\mu_{\ol{N}}$ is the unique natural transformation such that
  $$
  (\setsem{\Gamma; \emptyset \,|\, \Delta \vdash \eta : \Nat^{\Phi, \ol\alpha} \, F \,\, G[\ol{\beta := K}]}\rho \, d)_{\ol{N}}
  = \mu_{\ol{N}} \, \ol{\setsem{\Gamma; \ol\alpha \vdash K} \rho[\ol{\alpha := \_}]} \circ 
  \epsilon_{\ol{N}}
  $$

\end{lemma}
\begin{proof}
To prove that this family of morphisms is natural, it suffices to show that the family of unrestricted morphisms 
  $\lambda \ol{N} \, \ol{B}. (\mu_{\ol{N}})_{\ol{B}}$ is natural in $\ol{N}$ and $\ol{B}$.
Let us give convenient shorthand names to the following functors and natural transformations.
\begin{align*}
\ol{K} &= \ol{\lambda \ol M. \oCPOsem{\Gamma; \ol\alpha \vdash K}\rho[\ol{\alpha := M}]} \\
F_{\ol N} &= \lambda \ol M. \oCPOsem{\Gamma; \Phi, \ol\alpha \vdash F}\rho[\Phi := \ol N][\ol{\alpha := M}] \\
F_{\ol{N'}} &= \lambda \ol M. \oCPOsem{\Gamma; \Phi, \ol\alpha \vdash F}\rho[\Phi := \ol{N'}][\ol{\alpha := M}] \\
F_{\ol h} &= \lambda \ol M. \oCPOsem{\Gamma; \Phi, \ol\alpha \vdash F}\id_{\rho}[\Phi := \ol{h}][\ol{\alpha := \id_{M}}] \colon F_{\ol{N}} \to F_{\ol{N'}} \\
G_{\ol N} &= \lambda \ol B. \oCPOsem{\Gamma; \Phi, \ol\beta \vdash G}\rho[\Phi := \ol N][\ol{\beta := B}] \\
G_{\ol{N'}} &= \lambda \ol B. \oCPOsem{\Gamma; \Phi, \ol\beta \vdash G}\rho[\Phi := \ol{N'}][\ol{\beta := B}] \\
G_{\ol h} &= \lambda \ol B. \oCPOsem{\Gamma; \Phi, \ol\beta \vdash G}\id_{\rho}[\Phi := \ol{h}][\ol{\beta := \id_{B}}] \colon G_{\ol{N}} \to G_{\ol{N'}} \\
\eta &= \oCPOsem{\Gamma; \emptyset \,|\, \Delta \vdash \eta : \Nat^{\Phi, \ol\alpha} F\; G[\ol{\beta := K}]} \rho\, d
\end{align*}
Moreover, let $|\ol \alpha| = j$ and $|\ol \beta| = |\ol K| = k$.

Notice that \(\eta\) is natural, so that
\begin{equation*}\label{eq:lan-elim-eta-nat}
	\eta_{\ol{N'}} \circ F_{\ol h} = G_{\ol h} \ol{K} \circ \eta_{\ol N} : F_{\ol N} \to G_{\ol{N'}} \ol{K}
\end{equation*}
The situation is illustrated in the following diagram.
\[
\begin{tikzcd}[column sep = huge, row sep = huge]
\Set^j
\ar[r, bend left, "{F_{\ol N}}", ""{name=F, below}]
\ar[r, bend right, "{F_{\ol{N'}}}"{name=F'b, below}, ""{name=F'a, above}]
\ar[Rightarrow, from=F, to=F'a, "{F_{\ol{h}}}"']
\ar[dr, bend right, "{\ol K}"']
&\Set \\
&\Set^k
\ar[Rightarrow, from=F'b, "{\eta_{\ol{N'}}}"']
\ar[Rightarrow, from=F, "{\eta_{\ol N}}" description]
\ar[u, "{G_{\ol N}}"{name=L, description}]
\ar[u, bend right = 90, "{G_{\ol{N'}}}"{name=L', right}]
\ar[Rightarrow, from=L, to=L', "{G_{\ol h}}"]
\end{tikzcd}
\]

Let $\epsilon_{\ol N} \colon F_{\ol N} \to (\Lan_{\ol K} F_{\ol N}) K$ and $\epsilon_{\ol{N'}} \colon F_{\ol{N'}} \to (\Lan_{\ol K} F_{\ol{N'}}) \ol{K}$ be the natural transformations associated to the left Kan extensions, and notice that $\epsilon$ is natural, so that
\begin{equation}\label{eq:lan-elim-epsilon-nat}
	(\Lan_{\ol K} F_{\ol h}) \ol{K} \circ \epsilon_{\ol N} = \epsilon_{\ol{N'}} \circ F_{\ol h}
\end{equation}
The situation is illustrated in the following diagram.
\[
\begin{tikzcd}[column sep = huge, row sep = huge]
\Set^j
\ar[r, bend left, "{F_{\ol N}}", ""{name=F, below}]
\ar[r, bend right, "{F_{\ol{N'}}}"{name=F'b, below}, ""{name=F'a, above}]
\ar[Rightarrow, from=F, to=F'a, "{F_{\ol{h}}}"']
\ar[dr, bend right, "{\ol K}"']
&\Set \\
&\Set^k
\ar[Rightarrow, from=F'b, "{\epsilon_{\ol{N'}}}"']
\ar[Rightarrow, from=F, "{\epsilon_{\ol N}}" description]
\ar[u, "{\Lan_{\ol K} F_{\ol N}}"{name=L, description}]
\ar[u, bend right = 90, "{\Lan_{\ol K} F_{\ol{N'}}}"{name=L', right}]
\ar[Rightarrow, from=L, to=L', "{\Lan_K F_{\ol h}}"]
\end{tikzcd}
\]

Let $\mu_{\ol N} : \Lan_{\ol K} F_{\ol N} \to G_{\ol{N}}$ be the unique natural transformation such that
\begin{equation}\label{eq:lan-elim-mu-univ}
	\mu_{\ol N} \ol{K} \circ \epsilon_{\ol N} = \eta_{\ol N}
\end{equation}
and $\mu_{\ol{N'}} : \Lan_{\ol K} F_{\ol{N'}} \to G_{\ol{N'}}$ be the unique natural transformation such that
\begin{equation}\label{eq:lan-elim-mu'-univ}
	\mu_{\ol{N'}} \ol{K} \circ \epsilon_{\ol{N'}} = \eta_{\ol{N'}}
\end{equation}
The situation is illustrated in the following diagrams.
\[
\begin{tikzcd}[column sep = huge, row sep = huge]
\Set^j
\ar[r, "{F_{\ol N}}"{name=Fa}, ""{name=F, below}]
\ar[dr, "{\ol K}"']
&\Set \\
&\Set^k
\ar[Rightarrow, from=F, "{\epsilon_{\ol N}}" description]
\ar[u, "{\Lan_{\ol K} F_{\ol N}}"{name=L, description}]
\ar[u, bend right = 90, "{G_{\ol N}}"{name=G, right}, ""'{name=Gr, right}]
\ar[Rightarrow, bend left = 90, from=Fa, to=Gr, "{\eta_{\ol{N}}}"]
\ar[Rightarrow, from=L, to=G, "{\mu_{\ol N}}"]
\end{tikzcd}
%%%
\begin{tikzcd}[column sep = huge, row sep = huge]
\Set^j
\ar[r, "{F_{\ol{N'}}}"{name=Fa}, ""{name=F, below}]
\ar[dr, "{\ol K}"']
&\Set \\
&\Set^k
\ar[Rightarrow, from=F, "{\epsilon_{\ol{N'}}}" description]
\ar[u, "{\Lan_{\ol K} F_{\ol{N'}}}"{name=L, description}]
\ar[u, bend right = 90, "{G_{\ol{N'}}}"{name=G, right}, ""'{name=Gr, right}]
\ar[Rightarrow, bend left = 90, from=Fa, to=Gr, "{\eta_{\ol{N'}}}"]
\ar[Rightarrow, from=L, to=G, "{\mu_{\ol{N'}}}"]
\end{tikzcd}
\]

Notice that, by the universal property of the left Kan extension, there exists a unique $\theta : \Lan_K F_{\ol{N}} \to G_{\ol{N'}}$ such that
\[
	\theta \ol{K} \circ \epsilon_{\ol N} = \eta_{\ol{N'}} \circ F_{\ol h} = G_{\ol h} \ol{K} \circ \eta_{\ol N}
\]
We shall now show that both $G_{\ol h} \circ \mu_{\ol N}$ and $\mu_{\ol{N'}} \circ \Lan_{\ol K} F_{\ol{h}}$ satisfy this universal property, and thus they are equal to each other.
\[
	(G_{\ol h} \circ \mu_{\ol N}) \ol{K} \circ \epsilon_{\ol N}
	= G_{\ol h}\ol{K} \circ \mu_{\ol N} \ol{K} \circ \epsilon_{\ol N}
	= G_{\ol h}\ol{K} \circ \eta_{\ol N}
\]
where the second equality is by Equation~\ref{eq:lan-elim-mu-univ}
and
\[
	(\mu_{\ol{N'}} \circ \Lan_{\ol K} F_{\ol{h}}) \ol{K} \circ \epsilon_{\ol N}
	= \mu_{\ol{N'}}\ol{K} \circ \Lan_{\ol K} F_{\ol{h}}\ol{K} \circ \epsilon_{\ol N}
	= \mu_{\ol{N'}}\ol{K} \circ \epsilon_{\ol{N'}} \circ F_{\ol h}
	= \eta_{\ol{N'}} \circ F_{\ol h}
\]
where the second equality is by Equation~\ref{eq:lan-elim-epsilon-nat} and the third equality is by Equation~\ref{eq:lan-elim-mu'-univ}.


\end{proof}


\end{document}
