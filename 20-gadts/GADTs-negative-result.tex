\documentclass{lmcs}

\keywords{Nested types, GADTs, categorical semantics, parametricity}

\usepackage[utf8]{inputenc}
%\usepackage{ccicons}
\usepackage{verbatim}
\usepackage{adjustbox}
\usepackage{marvosym}
\usepackage{graphicx}
\usepackage{amsmath}
\usepackage{amsthm}
\usepackage{amscd}
%\usepackage{MnSymbol}
\usepackage{xcolor}

\usepackage{bbold}
\usepackage{url}
\usepackage{upgreek}
\usepackage{stmaryrd}

\usepackage{lipsum}
\usepackage{tikz-cd}
\usetikzlibrary{cd}
\usetikzlibrary{calc}
\usetikzlibrary{arrows}

\usepackage{bussproofs}
\EnableBpAbbreviations

\usepackage{hyperref}

\DeclareMathAlphabet{\mathpzc}{OT1}{pzc}{m}{it}

%\usepackage[amsmath]{ntheorem}

\newcommand{\lam}{\lambda}
\newcommand{\eps}{\varepsilon}
\newcommand{\ups}{\upsilon}
\newcommand{\mcB}{\mathcal{B}}
\newcommand{\mcD}{\mathcal{D}}
\newcommand{\mcE}{\mathcal{E}}
\newcommand{\mcF}{\mathcal{F}}
\newcommand{\mcP}{\mathcal{P}}
\newcommand{\mcI}{\mathcal{I}}
\newcommand{\mcJ}{\mathcal{J}}
\newcommand{\mcK}{\mathcal{K}}
\newcommand{\mcL}{\mathcal{L}}
\newcommand{\WW}{\mathcal{W}}

\newcommand{\ex}{\mcE_x}
\newcommand{\ey}{\mcE_y}
\newcommand{\bzero}{\boldsymbol{0}}
\newcommand{\bone}{{\boldsymbol{1}}}
\newcommand{\tB}{{\bone_\mcB}}
\newcommand{\tE}{{\bone_\mcE}}
\newcommand{\bt}{\mathbf{t}}
\newcommand{\bp}{\mathbf{p}}
\newcommand{\bsig}{\mathbf{\Sigma}}
\newcommand{\bpi}{\boldsymbol{\pi}}
\newcommand{\Empty}{\mathtt{Empty}}
\newcommand{\truthf}{\mathtt{t}}
\newcommand{\id}{id}
\newcommand{\coo}{\mathtt{coo\ }}
\newcommand{\mcC}{\mathcal{C}}
\newcommand{\Rec}{\mathpzc{Rec}}
\newcommand{\types}{\mathcal{T}}



%\newcommand{\semof}[1]{\llbracket{#1}\rrbracket^\rel}
\newcommand{\sem}[1]{\llbracket{#1}\rrbracket}
\newcommand{\setsem}[1]{\llbracket{#1}\rrbracket^\set}
\newcommand{\relsem}[1]{\llbracket{#1}\rrbracket^\rel}
\newcommand{\dsem}[1]{\llbracket{#1}\rrbracket^{\mathsf D}}
\newcommand{\setenv}{\mathsf{SetEnv}}
\newcommand{\relenv}{\mathsf{RelEnv}}
\newcommand{\denv}{\mathsf{DEnv}}

\newcommand{\rel}{\mathsf{Rel}}
\newcommand{\setof}[1]{\{#1\}}
\newcommand{\letin}[1]{\texttt{let }#1\texttt{ in }}
\newcommand{\comp}[1]{{\{#1\}}}
\newcommand{\bcomp}[1]{\{\![#1]\!\}}
\newcommand{\beps}{\boldsymbol{\eps}}
%\newcommand{\B}{\mcB}
%\newcommand{\Bo}{{|\mcB|}}

\newcommand{\lmt}{\longmapsto}
\newcommand{\RA}{\Rightarrow}
\newcommand{\LA}{\Leftarrow}
\newcommand{\rras}{\rightrightarrows}
\newcommand{\colim}[2]{{{\underrightarrow{\lim}}_{#1}{#2}}}
\newcommand{\lift}[1]{{#1}\,{\hat{} \; \hat{}}}
\newcommand{\graph}[1]{\langle {#1} \rangle}

\newcommand{\carAT}{\mathsf{car}({\mathcal A}^T)}
\newcommand{\isoAto}{\mathsf{Iso}({\mcA^\to})}
\newcommand{\falg}{\mathsf{Alg}_F}
\newcommand{\CC}{\mathsf{Pres}(\mathcal{A})}
\newcommand{\PP}{\mathcal{P}}
\newcommand{\DD}{D_{(A,B,f)}}
\newcommand{\from}{\leftarrow}
\newcommand{\upset}[1]{{#1}{\uparrow}}
\newcommand{\smupset}[1]{{#1}\!\uparrow}

\newcommand{\Coo}{\mathpzc{Coo}}
\newcommand{\code}{\#}
\newcommand{\nat}{\mathpzc{Nat}}

\newcommand{\eq}{\; = \;}
\newcommand{\of}{\; : \;}
\newcommand{\df}{\; := \;}
\newcommand{\bnf}{\; ::= \;}

\newcommand{\zmap}[1]{{\!{\between\!\!}_{#1}\!}}
\newcommand{\bSet}{\mathbf{Set}}

\newcommand{\dom}{\mathsf{dom}}
\newcommand{\cod}{\mathsf{cod}}
\newcommand{\adjoint}[2]{\mathrel{\mathop{\leftrightarrows}^{#1}_{#2}}}
\newcommand{\isofunc}{\mathpzc{Iso}}
\newcommand{\ebang}{{\eta_!}}
\newcommand{\lras}{\leftrightarrows}
\newcommand{\rlas}{\rightleftarrows}
\newcommand{\then}{\quad\Longrightarrow\quad}
\newcommand{\hookup}{\hookrightarrow}

\newcommand{\spanme}[5]{\begin{CD} #1 @<#2<< #3 @>#4>> #5 \end{CD}}
\newcommand{\spanm}[3]{\begin{CD} #1 @>#2>> #3\end{CD}}
\newcommand{\pushout}{\textsf{Pushout}}
\newcommand{\mospace}{\qquad\qquad\!\!\!\!}

\newcommand{\natur}[2]{#1 \propto #2}

\newcommand{\Tree}{\mathsf{Tree}\,}
\newcommand{\GRose}{\mathsf{GRose}\,}
\newcommand{\List}{\mathsf{List}\,}
\newcommand{\PTree}{\mathsf{PTree}\,}
\newcommand{\Bush}{\mathsf{Bush}\,}
\newcommand{\Forest}{\mathsf{Forest}\,}
\newcommand{\Lam}{\mathsf{Lam}\,}
\newcommand{\LamES}{\mathsf{Lam}^+}
\newcommand{\Expr}{\mathsf{Expr}\,}

\newcommand{\ListNil}{\mathsf{Nil}}
\newcommand{\ListCons}{\mathsf{Cons}}
\newcommand{\LamVar}{\mathsf{Var}}
\newcommand{\LamApp}{\mathsf{App}}
\newcommand{\LamAbs}{\mathsf{Abs}}
\newcommand{\LamSub}{\mathsf{Sub}}
\newcommand{\ExprConst}{\mathsf{Const}}
\newcommand{\ExprPair}{\mathsf{Pair}}
\newcommand{\ExprProj}{\mathsf{Proj}}
\newcommand{\ExprAbs}{\mathsf{Abs}}
\newcommand{\ExprApp}{\mathsf{App}}
\newcommand{\Ptree}{\mathsf{Ptree}}

\newcommand{\kinds}{\mathpzc{K}}
\newcommand{\tvars}{\mathbb{T}}
\newcommand{\fvars}{\mathbb{F}}
\newcommand{\consts}{\mathpzc{C}}
\newcommand{\Lan}{\mathsf{Lan}}
\newcommand{\zerot}{\mathbb{0}}
\newcommand{\onet}{\mathbb{1}}
\newcommand{\bool}{\mathbb{2}}
\renewcommand{\nat}{\mathbb{N}}
%\newcommand{\semof}[1]{[\![#1]\!]}
%\newcommand{\setsem}[1]{\llbracket{#1}\rrbracket^\set}
\newcommand{\predsem}[1]{\llbracket{#1}\rrbracket^\pred}
%\newcommand{\todot}{\stackrel{.}{\to}}
\newcommand{\todot}{\Rightarrow}
\newcommand{\bphi}{{\bm \phi}}

\newcommand{\bm}[1]{\boldsymbol{#1}}

\newcommand{\cL}{\mathcal{L}}
\newcommand{\T}{\mathcal{T}}
\newcommand{\Pos}{P\!}
%\newcommand{\Pos}{\mathcal{P}\!}
\newcommand{\Neg}{\mathcal{N}}
\newcommand{\Hf}{\mathcal{H}}
\newcommand{\V}{\mathbb{V}}
\newcommand{\I}{\mathcal{I}}
\newcommand{\Set}{\mathsf{Set}}
\newcommand{\Nat}{\mathsf{Nat}}
\newcommand{\Homrel}{\mathsf{Hom_{Rel}}}
\newcommand{\CV}{\mathcal{CV}}
\newcommand{\lan}{\mathsf{Lan}}
\newcommand{\Id}{\mathit{Id}}
\newcommand{\mcA}{\mathcal{A}}
\newcommand{\inl}{\mathsf{inl}}
\newcommand{\inr}{\mathsf{inr}}
\newcommand{\case}[3]{\mathsf{case}\,{#1}\,\mathsf{of}\,\{{#2};\,{#3}\}}
\newcommand{\tin}{\mathsf{in}}
\newcommand{\fold}{\mathsf{fold}}
\newcommand{\Eq}{\mathsf{Eq}}
\newcommand{\Hom}{\mathsf{Hom}}
\newcommand{\curry}{\mathsf{curry}}
\newcommand{\uncurry}{\mathsf{uncurry}}
\newcommand{\eval}{\mathsf{eval}}
\newcommand{\apply}{\mathsf{apply}}

\newcommand{\ar}[1]{\##1}
\newcommand{\mcG}{\mathcal{G}}
\newcommand{\mcH}{\mathcal{H}}
\newcommand{\TV}{\mathpzc{V}}

\newcommand{\essim}[1]{\mathsf{EssIm}(#1)}
\newcommand{\hra}{\hookrightarrow}

\newcommand{\ol}[1]{\overline{#1}}
\newcommand{\ul}[1]{\underline{#1}}
\newcommand{\op}{\mathsf{op}}
\newcommand{\trige}{\trianglerighteq}
\newcommand{\trile}{\trianglelefteq}
\newcommand{\LFP}{\mathsf{LFP}}
\newcommand{\LAN}{\mathsf{Lan}}
%\newcommand{\Mu}{{\mu\hskip-4pt\mu}}
\newcommand{\Mu}{{\mu\hskip-5.5pt\mu}}
%\newcommand{\Mu}{\boldsymbol{\upmu}}
\newcommand{\Terms}{\mathpzc{Terms}}
\newcommand{\Ord}{\mathpzc{Ord}}
\newcommand{\Anote}[1]{{\color{blue} {#1}}}
\newcommand{\Pnote}[1]{{\color{red} {#1}}}

\newcommand{\greyout}[1]{{\color{gray} {#1}}}
\newcommand{\ora}[1]{\overrightarrow{#1}}

%%\newcommand{\?}{{.\ }}
%\theoremheaderfont{\scshape}
%\theorembodyfont{\normalfont}
%\theoremseparator{.\ \ }
\newtheorem{thm}{Theorem}
\newtheorem{dfn}[thm]{Definition}
\newtheorem{prop}[thm]{Proposition}
\newtheorem{cor}[thm]{Corollary}
\newtheorem{lemma}[thm]{Lemma}
\newtheorem{rmk}[thm]{Remark}
\newtheorem{expl}[thm]{Example}
\newtheorem{notn}[thm]{Notation}
%\theoremstyle{nonumberplain}
%\theoremsymbol{\Box}


\theoremstyle{plain}\newtheorem{satz}[thm]{Satz}

\def\eg{{\em e.g.}}
\def\cf{{\em cf.}}

%\theoremstyle{definition}

\newcommand{\inl}{\mathsf{inl}}
\newcommand{\inr}{\mathsf{inr}}
\newcommand{\fold}{\mathsf{fold}}
\newcommand{\ininv}[2]{(\tin^{-1}_{\onet +
  \beta \times \phi (\phi\beta)})_{#1}\, #2}
\newcommand{\emptyfun}{{[]}}
\newcommand{\cal}{\mathcal}
%\newcommand{\fold}{\mathit{fold}}
\newcommand{\F}{\mathcal{F}}
\newcommand{\G}{\mathcal{G}}
\newcommand{\N}{\mathcal{N}}
\newcommand{\E}{\mathcal{E}}
\newcommand{\B}{\mathcal{B}}
\renewcommand{\P}{\mathcal{A}}
\newcommand{\pred}{\mathsf{Fam}}
\newcommand{\env}{\mathsf{Env}}
\newcommand{\set}{\mathsf{Set}}
\renewcommand{\S}{\mathcal S}
\newcommand{\C}{\mathcal{C}}
\newcommand{\D}{\mathcal{D}}
\newcommand{\A}{\mathcal{A}}
\renewcommand{\id}{\mathit{id}}
\newcommand{\map}{\mathsf{map}}
\newcommand{\pid}{\underline{\mathit{id}}}
\newcommand{\pcirc}{\,\underline{\circ}\,}
\newcommand{\pzero}{\underline{0}}
\newcommand{\pone}{\underline{1}}
\newcommand{\psum}{\,\underline{+}\,}
\newcommand{\pinl}{\underline{\mathit{inL}}\,}
\newcommand{\pinr}{\underline{\mathit{inR}}\,}
\newcommand{\ptimes}{\,\underline{\times}\,}
\newcommand{\ppi}{\underline{\pi_1}}
\newcommand{\pppi}{\underline{\pi_2}}
\newcommand{\pmu}{\underline{\mu}}
\newcommand{\semmap}{\mathit{map}}
\newcommand{\subst}{\mathit{subst}}

\newcommand{\tb}[1]{~~ \mbox{#1} ~~}
\newcommand{\listt}[1]{(\mu \phi. \lambda \beta . \onet + \beta \times
  \phi \beta) #1} 
\newcommand{\filtype}{\Nat^\emptyset 
 (\Nat^\emptyset \, \alpha \, \mathit{Bool})\, (\Nat^\emptyset 
  (List \, \alpha) \, (List \, \alpha))} 
\newcommand{\filtypeGRose}{\Nat^\emptyset 
 (\Nat^\emptyset \, \alpha \, \mathit{Bool})\, (\Nat^\emptyset 
  (\mathit{GRose}\,\psi \, \alpha) \, (\mathit{GRose}\,\psi \, (\alpha
  + \onet)))} 
\newcommand{\maplist}{\mathit{map}_{\lambda A. \setsem{\emptyset; \alpha
      \vdash \mathit{List} \, \alpha} \rho[\alpha := A]}} 
\newcommand{\PLeaves}{\mathsf{PLeaves}}
\newcommand{\swap}{\mathsf{swap}}
\newcommand{\reverse}{\mathsf{reverse}}
\newcommand{\bcons}{\mathit{bcons}}
\newcommand{\bnil}{\mathit{bnil}}

\DeclareMathOperator{\SumIn}{in}

\begin{document}

\title[Negative result on parametricity for GADTs]{Negative result on parametricity for GADTs}

\author[P.~Johann]{Patricia Johann}	
\address{Appalachian State University}	
\email{johannp@appstate.edu}  

\author[E.~Ghiorzi]{Enrico Ghiorzi}	
\address{Appalachian State University}	
\email{ghiorzie@appstate.edu}  

\author[D.~Jeffries]{Daniel Jeffries}	
\address{Appalachian State University}	
\email{jeffriesd@appstate.edu}

\maketitle



We augmented our calculus with $\Lan$ types to allow for primitive GADTs.
We then found out that $\Lan$ types do not support relational semantics for the usual category of relations, and thus the calculus does not have a parametric model.
Given that a counterexample was provided by relations of the form $(S, \emptyset, \emptyset)$, we consider restricting the category of relations to those relations that are onto their domain and codomain, i.e., those relations $R: \rel(A, B)$ such that for all $a \in A$ there exists $b \in B$ such that $(a, b) \in R$, and for all $b \in B$ there exists $a \in A$ such that $(a, b) \in R$.
Each onto relation $R: \rel(A, B)$ also comes equipped with two choice functions: $c_R : A \to B$ and $d_R : B \to A$, such that for every $a : A$ we have $(a, c_R a) \in R$, and for every $b : B$ we have $(d_R b, b) \in R$.
By abuse of notation, we still call $\rel$ this subcategory.

In the following, we will discuss how restricting $\rel$ to onto relations is still insufficient to provide a relational semantics for $\Lan$ types.



\section{Fibered interpretation of $\Lan$}

For every $\Gamma; \Phi, \ol\alpha \vdash F$, $\ol{\emptyset; \alpha \vdash K}$, $\ol{\Gamma; \Phi \vdash B}$ and relation environment $\rho$, we define the set interpretation of $\Gamma;\Phi \vdash (\Lan^{\ol{\alpha}}_{\ol{K}}\,F)\ol{B}$ as
\[
  \setsem{\Gamma;\Phi \vdash
    (\Lan^{\ol{\alpha}}_{\ol{K}}\,F)\ol{B}}\rho = (
  \textit{Lan}_{\lambda \ol{S}.\,\ol{\setsem{\Gamma; \ol\alpha \vdash K}
      \rho[\ol{\alpha := S}]}}\, \lambda \ol{S}.\,\setsem{\Gamma; \Phi, \ol\alpha
    \vdash F} \rho[\ol{\alpha := S}])\, \ol{\setsem{\Gamma; \Phi
      \vdash B}\rho}
\]
and its relational interpretation as
\[
  \relsem{\Gamma;\Phi \vdash
    (\Lan^{\ol{\alpha}}_{\ol{K}}\,F)\ol{B}}\rho = (
  \textit{Lan}_{\lambda \ol R.\,\ol{\relsem{\Gamma; \ol\alpha \vdash
        K}\rho[\ol{\alpha := R}]}}\, \lambda \ol R.\,\relsem{\Gamma;
    \Phi, \ol\alpha \vdash F}\rho[\ol{\alpha := R}])\,
  \ol{\relsem{\Gamma; \Phi \vdash B}\rho}
\]

We want to show that the relational interpretation for $\Lan$ types is fibered over the set interpretation, i.e., that there is an isomorphism between
\begin{equation}\label{eq:LHS}
\colim{\ol{S : \set_0}, \ol{g : \setsem{\emptyset; \ol{\alpha} \vdash K}[\ol{\alpha := S}] \to \setsem{\Gamma; \Phi \vdash B}(\pi_1 \rho)}}
{\setsem{\Gamma; \Phi, \ol{\alpha} \vdash F}(\pi_1 \rho)[\ol{\alpha := S}]}
\end{equation}
where $\iota_{\ol{S}, \ol{g}} : \setsem{\Gamma; \Phi, \ol{\alpha} \vdash F}(\pi_1 \rho)[\ol{\alpha := S}] \to \eqref{eq:LHS}$ is the cocone into the colimit~\eqref{eq:LHS}, and
\begin{equation}\label{eq:RHS}
\colim{\ol{R : \rel_0}, \ol{f : \relsem{\emptyset; \ol{\alpha} \vdash K}[\ol{\alpha := R}] \to \relsem{\Gamma; \Phi \vdash B}\rho}}
{\setsem{\Gamma; \Phi, \ol{\alpha} \vdash F}(\pi_1 \rho)[\ol{\alpha := \pi_1 R}]}
\end{equation}
where $j_{\ol{R}, \ol{f}} : \setsem{\Gamma; \Phi, \ol{\alpha} \vdash F}(\pi_1 \rho)[\ol{\alpha := \pi_1 R}] \to \eqref{eq:RHS}$ is the cocone into the colimit~\eqref{eq:RHS}.

We define $k : \eqref{eq:LHS} \to \eqref{eq:RHS}$ as the unique function such that $k \circ \iota_{\ol{S}, \ol{g}} = j_{\ol{\Eq_S}, \ol{(\id, \setsem{\Gamma; \Phi \vdash B} c_\rho) \circ \Eq_g}}$ for every $\ol{S : \set_0}$ and $\ol{g : \setsem{\emptyset; \ol{\alpha} \vdash K}[\ol{\alpha := S}] \to \setsem{\Gamma; \Phi \vdash B}(\pi_1 \rho)}$.

We define $h : \eqref{eq:RHS} \to \eqref{eq:LHS}$ as the unique function such that $h \circ j_{\ol{R}, \ol{f}} = \iota_{\ol{\pi_1 R}, \ol{\pi_1 f}}$ for every $\ol{R : \rel_0}$ and $\ol{f : \relsem{\emptyset; \ol{\alpha} \vdash K}[\ol{\alpha := R}] \to \relsem{\Gamma; \Phi \vdash B}\rho}$.

We want to show that $h$ and $k$ are inverses.

Consider the composition $h \circ k$.
We have that
\[
h \circ k \circ \iota_{\ol{S}, \ol{g}}
= h \circ j_{\ol{\Eq_S}, \ol{(\id, \setsem{\Gamma; \Phi \vdash B} c_\rho) \circ \Eq_g}}
= \iota_{\ol{\pi_1 \Eq_S}, \ol{\pi_1 ((\id, \setsem{\Gamma; \Phi \vdash B} c_\rho) \circ \Eq_g)}}
= \iota_{\ol{S}, \ol{g}}
\]
and so, $h \circ k = \id_{\eqref{eq:LHS}}$.

Consider the composition $k \circ h$.
We have that
\[
k \circ h \circ j_{\ol{R}, \ol{f}}
= k \circ \iota_{\ol{\pi_1 R}, \ol{\pi_1 f}}
= j_{\ol{\Eq_{\pi_1 R}}, \ol{(\id, \setsem{\Gamma; \Phi \vdash B} c_\rho) \circ \Eq_{\pi_1 f}}}
\]
To prove that $j_{\ol{R}, \ol{f}} = j_{\ol{\Eq_{\pi_1 R}}, \ol{(\id, \setsem{\Gamma; \Phi \vdash B} c_\rho) \circ \Eq_{\pi_1 f}}}$,
it suffices to show that, in the indexing category of the colimit~\eqref{eq:RHS}, either the triangle
\[
\begin{tikzcd}[row sep = huge]
\relsem{\emptyset; \ol{\alpha} \vdash K}[\ol{\alpha := R}]
	\ar[rr, "{\relsem{\emptyset; \ol{\alpha} \vdash K}[\ol{\alpha := (\id, d_{R})}]}"]
	\ar[dr, "{f}"']
&& \relsem{\emptyset; \ol{\alpha} \vdash K}[\ol{\alpha := \Eq_{\pi_1 R}}]
	\ar[dl, "{(\id, \setsem{\Gamma; \Phi \vdash B} c_\rho) \circ \Eq_{\pi_1 f}}"] \\
&\relsem{\Gamma; \Phi \vdash B}\rho
\end{tikzcd}
\]
or the triangle
\[
\begin{tikzcd}[row sep = huge]
\relsem{\emptyset; \ol{\alpha} \vdash K}[\ol{\alpha := R}]
	\ar[dr, "{f}"']
&& \relsem{\emptyset; \ol{\alpha} \vdash K}[\ol{\alpha := \Eq_{\pi_1 R}}]
	\ar[dl, "{(\id, \setsem{\Gamma; \Phi \vdash B} c_\rho) \circ \Eq_{\pi_1 f}}"]
	\ar[ll, "{\relsem{\emptyset; \ol{\alpha} \vdash K}[\ol{\alpha := (\id, c_{R})}]}"'] \\
&\relsem{\Gamma; \Phi \vdash B}\rho
\end{tikzcd}
\]
commutes, but that does not seem to be the case.
Indeed, there is no reason why it should be true that $\setsem{\Gamma; \Phi \vdash B} c_\rho \circ (\pi_1 f) \circ \setsem{\emptyset; \ol{\alpha} \vdash K}[\ol{\alpha := d_{R}}] = (\pi_1 f)$
or $(\pi_1 f) \circ \setsem{\emptyset; \ol{\alpha} \vdash K}[\ol{\alpha := c_{R}}] = \setsem{\Gamma; \Phi \vdash B} c_\rho \circ (\pi_1 f)$.

Since our first approach did not work, we shall instead try to show that \eqref{eq:RHS} is isomorphic to
\begin{equation}\label{eq:RHS_EQ}
\colim{\ol{S : \set_0}, \ol{f : \relsem{\emptyset; \ol{\alpha} \vdash K}[\ol{\alpha := \Eq_S}] \to \relsem{\Gamma; \Phi \vdash B}\rho}}
{\setsem{\Gamma; \Phi, \ol{\alpha} \vdash F}(\pi_1 \rho)[\ol{\alpha := S}]}
\end{equation}
which should hold true if equality relations are cofinal in $\rel$.
Moreover, the fact that the colimit \eqref{eq:RHS_EQ} is indexed over sets suggests that it is also isomorphic to \eqref{eq:LHS}.
Before proceeding, though, we need to introduce a way to associate a set to an arbitrary relation, which we shall do in the following section.






\section{Sets from relations}

Given a set $S$, there are a few canonical ways to produce a relation that represents it.
\begin{itemize}
\item The most obvious one is the equality relation, $\Eq_S = (S, S, \Delta_S)$.
\item The relations $(S, \emptyset, \emptyset)$ and $(\emptyset, S, \emptyset)$, although they are not onto relations in the sense defined in the previous section.
\item The relations $(S, 1, S \times 1)$ and $(1, S, 1 \times S)$, although they are more of a set seen as a relation than a relation representing $S$.
\end{itemize}
Overall, the most natural option is the equality relation over a set.
Moreover, for the purposes of the Identity Extension Lemma, we really need to consider equality relations.

There is an equality functor $\Eq : \set \to \rel$ sending a set to its equality relation.
We want to find a left adjoint $\Sigma : \rel \to \set$, i.e., a way to construct a set $\Sigma{R}$ out of a relation $R$ together with a canonical arrow $R \to \Eq_{\Sigma{R}}$.

\begin{defi}
Given a relation $R : \rel(A, B)$, we define an equivalence relation $\sim$ on $A + B$ as the smallest equivalence relation generated by the pairs $(\SumIn_1 a, \SumIn_2 b)$ such that $(a, b) \in R$.
Explicitly, $\SumIn_1 a \sim \SumIn_2 b$ if and only if there is a finite sequence $a_1, b_1, a_2, b_2, \dots, a_n, b_n$ such that $a_1 = a$, $b_n = b$, $(a_i, b_i) \in R$ for all $i = 1 \dots n$ and $(a_{i + 1}, b_i) \in R$ for all $i = 1 \dots n - 1$.
We define $\sim$ analogously for the remaining three cases: $\SumIn_1 a \sim \SumIn_1 a'$, $\SumIn_2 b \sim \SumIn_2 b'$, and $\SumIn_2 b \sim \SumIn_1 a$.
Finally, we define $\Sigma{R}$ as the quotient $A + B / \sim$.

Given a morphism of relations $m = (m_1, m_2) : R \to Q$ we define the function $\Sigma{m} : \Sigma{R} \to \Sigma{Q}$ by $\Sigma{m}[\SumIn_1 a] = [ \SumIn_1 (m_1 a) ]$ and $\Sigma{m}[\SumIn_2 b] = [\SumIn_2 (m_2 b)]$.
To verify that $\Sigma{m}$ is well-defined, consider two representatives of the same class of equivalence, $\SumIn_1 a \sim \SumIn_2 b$.
Then there is a finite sequence $a_1, b_1, a_2, b_2, \dots, a_n, b_n$ such that $a_1 = a$, $b_n = b$, $(a_i, b_i) \in R$ for all $i = 1 \dots n$ and $(a_{i + 1}, b_{i}) \in R$ for all $i = 1 \dots n - 1$.
The sequence $m_1 a_1, m_2 b_1, m_1 a_2, m_2 b_2, \dots, m_1 a_n, m_2 b_n$ is such that $(m_1 a_i, m_2 b_i) \in Q$ for all $i = 1 \dots n$ and $(m_1 a_{i + 1}, m_2 b_{i}) \in Q$ for all $i = 1 \dots n - 1$, thus proving that $\Sigma{m}[\SumIn_1 a] = [\SumIn_1 (m_1 a)] = [\SumIn_2 (m_2 b)] = \Sigma{m}[\SumIn_2 b]$.

These data yield a functor $\Sigma : \rel \to \set$.
\end{defi}

We want to show that the above construction is the best way to represent a relation by a set, i.e., that there is an adjunction between the functors $\Sigma$ and $\Eq$.
The intuition that this is the case is reinforced by noticing that, for every relation $R : \rel(A, B)$, there is a morphism $(q \circ \SumIn_1, q \circ \SumIn_2) : R \to \Eq_{\Sigma{R}}$, where $q : A + B \to A + B/\sim$ is the projection onto the quotient.
The next proposition gives a complete proof of this fact.

\begin{prop}
There is an adjunction $\Sigma \dashv \Eq$.
\end{prop}
\begin{proof}
Let $R : \rel(A, B)$ and $S: \set$.
We want to show that there is a bijective correspondence between functions $\Sigma{R} \to S$ and morphisms of relations $R \to \Eq_S$.

Let $f : \Sigma{R} \to S$ be a function.
Define a morphism of relations $\tilde{f} = (\tilde{f}_1, \tilde{f}_2) : R \to \Eq_S$ as $\tilde{f}_1 a = f [ \SumIn_1 a ]$ and $\tilde{f}_2 b = f [ \SumIn_2 b ]$.
This is a well-defined morphism of relations: indeed, if $(a, b) \in R$, then $\SumIn_1 a \sim \SumIn_2 b$ and thus $\tilde{f}_1 a = f [ \SumIn_1 a ] = f [ \SumIn_2 b ] = \tilde{f}_2 b$.

Let $m = (m_1, m_2) : R \to \Eq_S$ be a morphism of relations.
Define a function $\hat{m} : \Sigma{R} \to S$ as $\hat{m} [ \SumIn_1 a ] = m_1 a$ and $\hat{m} [ \SumIn_2 b ] = m_2 b$.
This is well-defined: indeed, if $\SumIn_1 a \sim \SumIn_2 b$ (one out of four possible cases, the other three being $\SumIn_1 a \sim \SumIn_1 a'$, $\SumIn_2 b \sim \SumIn_1 a$, and $\SumIn_2 b \sim \SumIn_2 b'$), there is a finite sequence $a_1, b_1, a_2, b_2, \dots, a_n, b_n$ such that $a_1 = a$, $b_n = b$, $(a_i, b_i) \in R$ for all $i = 1 \dots n$ and $(a_{i + 1}, b_{i}) \in R$ for all $i = 1 \dots n - 1$.
Then $m_1 a_1 = m_2 b_1 = m_1 a_2 = m_2 b_2 = \dots = m_1 a_n = m_2 b_n$, thus proving that $\hat{m}[\SumIn_1 a] = m_1 a = m_2 b = \hat{m}[\SumIn_2 b]$.
We have analogous proofs for the other three cases.

We now want to show that the two constructions are inverse to each other.
In one direction, $\hat{\tilde{f}} [\SumIn_1 a] = \tilde{f}_1 a = f [\SumIn_1 a]$ and $\hat{\tilde{f}} [\SumIn_2 b] = \tilde{f}_2 b = f [\SumIn_2 b]$, and thus $\hat{\tilde{f}} = f$.
In the other direction, $\tilde{\hat{m}}_1 a = \hat{m} [\SumIn_1 a] = m_1 a$ and $\tilde{\hat{m}}_2 b = \hat{m} [\SumIn_2 b] = m_2 b$, and thus $\tilde{\hat{m}} = m$.
\end{proof}

Observe that, for every relation $R : \rel(A, B)$, we can use the axiom of choice to define choice functions $\Xi_1 : \Sigma R \to A$ and $\Xi_2 : \Sigma R \to B$ such that $\SumIn_1 ( \Xi_1 x ) \in x$, $\SumIn_2 (\Xi_2 x ) \in x$, and $(\Xi_1 x, \Xi_2 x) \in R$ for every $x : \Sigma R$.
In other words, $(\Xi_1, \Xi_2)$ chooses a pair of related elements out of each equivalence class of $\Sigma R$.
Moreover, since $(\Xi_1 x, \Xi_2 x) \in R$ for every $x : \Sigma R$, it's immediate that $(\Xi_1, \Xi_2) : \Eq_{\Sigma R} \to R$ is a morphism of relations.




\section{Counterexample to fibered interpretation of $\Lan$}

We want to show that there is an isomorphism between \eqref{eq:RHS} and \eqref{eq:RHS_EQ}.
The situation is represented in the following diagram
\[
\begin{tikzcd}[column sep = huge]
&\setsem{\Gamma; \Phi, \ol\alpha \vdash F} (\pi_1 \rho) [\ol{\alpha := S}]
\ar[d, "{l_{\ol{S}, \ol{g}}}"]
\ar[dl, bend right = 15, "{j_{\ol{\Eq_{S}}, \ol{g}}}"'] \\
\eqref{eq:RHS}
\ar[r, bend right = 5, "{\Phi}"']
&\eqref{eq:RHS_EQ}
\ar[l, bend right = 5, "{\Psi}"'] \\
\setsem{\Gamma; \Phi, \ol\alpha \vdash F} (\pi_1 \rho) [\ol{\alpha := \pi_1 R}]
\ar[r, bend right = 5, "{\setsem{\Gamma; \Phi, \ol\alpha \vdash F} \id_{\pi_1 \rho} [\ol{\alpha := q \circ \SumIn_1}]}"']
\ar[u, "{j_{\ol{R}, \ol{f}}}"]
&\setsem{\Gamma; \Phi, \ol\alpha \vdash F} (\pi_1 \rho) [\ol{\alpha := \Sigma{R}}]
\ar[u, "{l_{\ol{\Sigma{R}}, \ol{f \circ \relsem{\emptyset; \ol\alpha \vdash K}[\ol{\alpha := (\Xi_1, \Xi_2)}]}}}"']
\end{tikzcd}
\]
where $l_{\ol{S}, \ol{g}} : \setsem{\Gamma; \Phi, \ol\alpha \vdash F} (\pi_1 \rho) [\ol{\alpha := S}] \to \eqref{eq:RHS_EQ}$ is the cocone into the colimit~\eqref{eq:RHS_EQ}.

We define $\Phi : \eqref{eq:RHS} \to \eqref{eq:RHS_EQ}$ as the unique function such that $\Phi \circ j_{\ol{R}, \ol{f}} = l_{\ol{\Sigma{R}}, \ol{f \circ \relsem{\emptyset; \ol\alpha \vdash K}[\ol{\alpha := (\Xi_1, \Xi_2)}]}} \circ \setsem{\Gamma; \Phi, \ol\alpha \vdash F} \id_{\pi_1 \rho}[\ol{\alpha := q \circ \SumIn_1}]$ for every $\ol{R : \rel_0}$ and $\ol{f : \relsem{\emptyset; \ol{\alpha} \vdash K}[\ol{\alpha := R}] \to \relsem{\Gamma; \Phi \vdash B}\rho}$.

We define $\Psi : \eqref{eq:RHS_EQ} \to \eqref{eq:RHS}$ as the unique function such that $\Psi \circ l_{\ol{S}, \ol{g}} = j_{\ol{\Eq_S}, \ol{g}}$ for every $\ol{S : \set_0}$ and $\ol{g : \relsem{\emptyset; \ol{\alpha} \vdash K}[\ol{\alpha := \Eq_S}] \to \relsem{\Gamma; \Phi \vdash B}\rho}$.

We want to show that $\Psi$ and $\Phi$ are inverses.
For the composition $\Phi \circ \Psi$, we have that
\[
\Phi \circ \Psi \circ l_{\ol{S}, \ol{g}}
= \Phi \circ j_{\ol{\Eq_S}, \ol{g}}
= l_{\ol{\Sigma{\Eq_S}}, \ol{g \circ \relsem{\emptyset; \ol\alpha \vdash K}[\ol{\alpha := (\Xi_1, \Xi_2)}]}} \circ \setsem{\Gamma; \Phi, \ol\alpha \vdash F} \id_{\pi_1 \rho}[\ol{\alpha := q \circ \SumIn_1}]
\]
To show that $l_{\ol{S}, \ol{g}} = l_{\ol{\Sigma{\Eq_S}}, \ol{g \circ \relsem{\emptyset; \ol\alpha \vdash K}[\ol{\alpha := (\Xi_1, \Xi_2)}]}} \circ \setsem{\Gamma; \Phi, \ol\alpha \vdash F} \id_{\pi_1 \rho}[\ol{\alpha := q \circ \SumIn_1}]$, consider the following diagram in the indexing category:
\[
\begin{tikzcd}[column sep = huge]
\relsem{\emptyset; \ol{\alpha} \vdash K}[\ol{\alpha := \Eq_S}]
\ar[r, bend left = 5, "{\relsem{\emptyset; \ol{\alpha} \vdash K}[\ol{\alpha := (q \circ \SumIn_1, q \circ \SumIn_2)}]}"]
\ar[d, "{g}"']
& \relsem{\emptyset; \ol{\alpha} \vdash K}[\ol{\alpha := \Eq_{\Sigma{\Eq_S}}}]
\ar[d, "{\relsem{\emptyset; \ol{\alpha} \vdash K}[\ol{\alpha := (\Xi_1, \Xi_2)}]}"] \\
\relsem{\Gamma; \Phi \vdash B}\rho
& \relsem{\emptyset; \ol{\alpha} \vdash K}[\ol{\alpha := \Eq_{S}}]
\ar[l, "{g}"]
\end{tikzcd}
\]
The diagram commutes because $(\Xi_1 \circ q \circ \SumIn_1, \Xi_2 \circ q \circ \SumIn_2) = \id_{\Eq_S}$, i.e., $\Xi_1 \circ q \circ \SumIn_1 = \id_S$ and $\Xi_2 \circ q \circ \SumIn_2 = \id_S$.
Indeed, $q$ is the projection into the quotient for the equivalence relation $\sim_{\Eq_S}$, for which $\SumIn_i x \sim_{\Eq_S} \SumIn_{i'} x'$ if and only if $x = x'$ for $i, i' = 1, 2$, so that $\Xi_1 ( q ( \SumIn_1 x ) ) = x$ and $\Xi_2 ( q ( \SumIn_2 x ) ) = x$.
Thus, $l_{\ol{S}, \ol{g}} = l_{\ol{\Sigma{\Eq_S}}, \ol{g \circ \relsem{\emptyset; \ol\alpha \vdash K}[\ol{\alpha := (\Xi_1, \Xi_1)}]}} \circ \setsem{\Gamma; \Phi, \ol\alpha \vdash F} \id_{\pi_1 \rho}[\ol{\alpha := q \circ \SumIn_1}]$
and we can conclude that $\Phi \circ \Psi = \id$.

In the other direction, for the composition $\Psi \circ \Phi$, we have that
\begin{align*}
\Psi \circ \Phi \circ j_{\ol{R}, \ol{f}}
&= \Psi \circ l_{\ol{\Sigma{R}}, \ol{f \circ \relsem{\emptyset; \ol\alpha \vdash K}[\ol{\alpha := (\Xi_1, \Xi_2)}]}} \circ \setsem{\Gamma; \Phi, \ol\alpha \vdash F} \id_{\pi_1 \rho}[\ol{\alpha := q \circ \SumIn_1}] \\
&= j_{\ol{\Eq_{\Sigma{R}}}, \ol{f \circ \relsem{\emptyset; \ol\alpha \vdash K}[\ol{\alpha := (\Xi_1, \Xi_2)}]}} \circ \setsem{\Gamma; \Phi, \ol\alpha \vdash F} \id_{\pi_1 \rho}[\ol{\alpha := q \circ \SumIn_1}]
\end{align*}
To show that
\begin{equation}\label{eq:PsiPhi}
j_{\ol{R}, \ol{f}} = j_{\ol{\Eq_{\Sigma{R}}}, \ol{f \circ \relsem{\emptyset; \ol\alpha \vdash K}[\ol{\alpha := (\Xi_1, \Xi_2)}]}} \circ \setsem{\Gamma; \Phi, \ol\alpha \vdash F} \id_{\pi_1 \rho}[\ol{\alpha := q \circ \SumIn_1}]
\end{equation}
it would suffice the following diagram in the indexing category to commute:
\begin{equation}\label{eq:index-comm}
\begin{tikzcd}[column sep = huge]
\relsem{\emptyset; \ol{\alpha} \vdash K}[\ol{\alpha := R}]
\ar[r, bend left = 5, "{\relsem{\emptyset; \ol{\alpha} \vdash K}[\ol{\alpha := (q \circ \SumIn_1, q \circ \SumIn_2)}]}"]
\ar[d, "{f}"']
& \relsem{\emptyset; \ol{\alpha} \vdash K}[\ol{\alpha := \Eq_{\Sigma{R}}}]
\ar[d, "{\relsem{\emptyset; \ol{\alpha} \vdash K}[\ol{\alpha := (\Xi_1, \Xi_2)}]}"] \\
\relsem{\Gamma; \Phi \vdash B}\rho
& \relsem{\emptyset; \ol{\alpha} \vdash K}[\ol{\alpha := R}]
\ar[l, "{f}"]
\end{tikzcd}
\end{equation}
As before, this requires $(\Xi_1 \circ q \circ \SumIn_1, \Xi_2 \circ q \circ \SumIn_2) = \id_{R}$, i.e., that $\Xi_1 \circ q \circ \SumIn_1 = \id_{\pi_1 R}$ and $\Xi_2 \circ q \circ \SumIn_2 = \id_{\pi_2 R}$, where $q$ is the projection into the quotient for the equivalence relation $\sim_{R}$.
But that was true in the previous direction because we were considering an equality relation, for which the equivalence classes are singletons.
For a generic relation $R$, as in this case, there seem to be no reason why this should be true, and indeed we shall provide a counterexample.

It is worth noting that we have not definitively disproved Equation~\ref{eq:PsiPhi}, but we have just shown how this way of proving the equation does not work.

\begin{exa}
Let $R$ be the relation $(\{ 1, 2, 3\}, \{a, b\}, \{(1, a), (2, b), (3, b)\}, d_R, c_R)$ where $d_R$ maps $a$ into $1$ and $b$ into $2$, and $c_R$ maps $1$ into $a$, and $2$ and $3$ into $b$.
Thus, $\Sigma{R}$ is the set $\{ \{1, a\}, \{2, 3, b\} \}$.
We also have the morphism of relations $(q \circ \SumIn_1, q \circ \SumIn_2) : R \to \Eq_{\Sigma{R}}$ where $q \circ \SumIn_1$ sends $1$ into $\{1, a\}$, and $2$ and $3$ into $\{2, 3, b\}$; $q \circ \SumIn_2$ sends $a$ into $\{1, a\}$, and $b$ into $\{2, 3, b\}$.
Moreover, we have the choice functions $\Xi_1 : \Sigma R \to \{ 1, 2, 3\}$ sending $\{1, a\}$ to $1$ and $\{2, 3, b\}$ to $2$, and $\Xi_2 : \Sigma_R \to \{a, b\}$ sending $\{1, a\}$ to $a$ and $\{2, 3, b\}$ to $b$.

Let $K = \alpha$, $B = \beta$ with $\rho \beta = R$ and $f = \id_R$.
Instantiating the above diagram, we get that it commutes if and only if $(\Xi_1, \Xi_2) \circ (q \circ \SumIn_1, q \circ \SumIn_2) = \id_R$.
But $\Xi_1 \circ q \circ \SumIn_1$ is not the identity (as it is not injective, because $\Xi_1 ( q ( \SumIn_1 2 ) ) = \Xi_1 \{ 2, 3, b \} = \Xi_1 ( q ( \SumIn_1 3 ) )$) and thus the diagram does not commute.

This still does not exclude that Equation~\eqref{eq:PsiPhi} might hold.
Let $F = \alpha$.
Then Equation~\eqref{eq:PsiPhi} becomes $j_{\ol{\Eq_{\Sigma{R}}}, \ol{(\Xi_1, \Xi_2)}} \circ q \circ \SumIn_1 = j_{\ol{R}, \ol{\id}}$, as shown in the following diagram:
\[
\begin{tikzcd}
\pi_1 R
\ar[r, "{j_{\ol{R}, \ol{\id}}}"]
\ar[d, "{q \circ \SumIn_1}"']
&\colim{Q: \rel_0, f: Q \to R}{\pi_1 Q} \\
\Sigma{R}
\ar[ur, "{j_{\ol{\Eq_{\Sigma{R}}}, \ol{(\Xi_1, \Xi_2)}}}"']
\end{tikzcd}
\]
Notice that $\colim{Q: \rel_0, f: Q \to R}{\pi_1 Q} = \pi_1 \colim{Q: \rel_0, f: Q \to R}{Q} = \pi_1 R$, and thus $j_{\ol{R}, \ol{\id}} = \id_{\pi_1 R}$ and $j_{\ol{\Eq_{\Sigma{R}}}, \ol{(\Xi_1, \Xi_2)}} = \Xi_1$.
Finally, Equation~\eqref{eq:PsiPhi} becomes $\Xi_1 \circ q \circ \SumIn_1 = \id_{\pi_1 R}$, which we already showed being false.
\end{exa}

Observe that, for the choice of $K$, $F$ and $B$ as in the example above, it is actually the case that $\eqref{eq:LHS} \cong \eqref{eq:RHS}$, as they are both equal to $\pi_1 R$.
Indeed, by instantiating, we have that $\eqref{eq:LHS} = \colim{\ol{S : \set_0}, \ol{g : S \to \pi_1 R}}{S} = \pi_1 R$ and
$\eqref{eq:RHS} = \colim{\ol{Q : \rel_0}, \ol{f : Q \to R}}{\pi_1 Q} = \pi_1 \colim{\ol{Q : \rel_0}, \ol{f : Q \to R}}{Q} = \pi_1 R$.
So, the above is not a counterexample to $\eqref{eq:LHS} \cong \eqref{eq:RHS}$, but just to the commutativity of Diagram~\ref{eq:index-comm}.



\bibliographystyle{alpha}
\bibliography{references}
  
\end{document}



