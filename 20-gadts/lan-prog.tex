% double-blind review submission, w/o CCS and ACM Reference (max
%
% submission space)
\documentclass[acmsmall,review,anonymous]{acmart}
\settopmatter{printfolios=true,printccs=false,printacmref=false}
%% For double-blind review submission, w/ CCS and ACM Reference
%\documentclass[acmsmall,review,anonymous]{acmart}\settopmatter{printfolios=true}
%% For single-blind review submission, w/o CCS and ACM Reference (max submission space)
%\documentclass[acmsmall,review]{acmart}\settopmatter{printfolios=true,printccs=false,printacmref=false}
%% For single-blind review submission, w/ CCS and ACM Reference
%\documentclass[acmsmall,review]{acmart}\settopmatter{printfolios=true}
%% For final camera-ready submission, w/ required CCS and ACM Reference
%\documentclass[acmsmall]{acmart}\settopmatter{}


%% Journal information
%% Supplied to authors by publisher for camera-ready submission;
%% use defaults for review submission.
\acmJournal{PACMPL}
\acmVolume{1}
\acmNumber{POPL} % CONF = POPL or ICFP or OOPSLA
\acmArticle{1}
\acmYear{2020}
\acmMonth{1}
\acmDOI{} % \acmDOI{10.1145/nnnnnnn.nnnnnnn}
\startPage{1}

%% Copyright information
%% Supplied to authors (based on authors' rights management selection;
%% see authors.acm.org) by publisher for camera-ready submission;
%% use 'none' for review submission.
\setcopyright{none}
%\setcopyright{acmcopyright}
%\setcopyright{acmlicensed}
%\setcopyright{rightsretained}
%\copyrightyear{2018}           %% If different from \acmYear

%% Bibliography style
\bibliographystyle{ACM-Reference-Format}
%% Citation style
%% Note: author/year citations are required for papers published as an
%% issue of PACMPL.
\citestyle{acmauthoryear}   %% For author/year citations
%\citestyle{acmnumeric}

%%%%%%%%%%%%%%%%%%%%%%%%%%%%%%%%%%%%%%%%%%%%%%%%%%%%%%%%%%%%%%%%%%%%%%
%% Note: Authors migrating a paper from PACMPL format to traditional
%% SIGPLAN proceedings format must update the '\documentclass' and
%% topmatter commands above; see 'acmart-sigplanproc-template.tex'.
%%%%%%%%%%%%%%%%%%%%%%%%%%%%%%%%%%%%%%%%%%%%%%%%%%%%%%%%%%%%%%%%%%%%%%



\usepackage[utf8]{inputenc}
\usepackage{ccicons}
\usepackage{verbatim}

\usepackage{amsmath}
\usepackage{amsthm}
\usepackage{amscd}
%\usepackage{MnSymbol}
\usepackage{xcolor}

\usepackage{bbold}
\usepackage{url}
\usepackage{upgreek}
%\usepackage{stmaryrd}

\usepackage{lipsum}
\usepackage{tikz-cd}
\usetikzlibrary{cd}
\usetikzlibrary{calc}
\usetikzlibrary{arrows}

\usepackage{bussproofs}
\EnableBpAbbreviations

\DeclareMathAlphabet{\mathpzc}{OT1}{pzc}{m}{it}

%\usepackage[amsmath]{ntheorem}

\newcommand{\lam}{\lambda}
\newcommand{\eps}{\varepsilon}
\newcommand{\ups}{\upsilon}
\newcommand{\mcB}{\mathcal{B}}
\newcommand{\mcD}{\mathcal{D}}
\newcommand{\mcE}{\mathcal{E}}
\newcommand{\mcF}{\mathcal{F}}
\newcommand{\mcP}{\mathcal{P}}
\newcommand{\mcI}{\mathcal{I}}
\newcommand{\mcJ}{\mathcal{J}}
\newcommand{\mcK}{\mathcal{K}}
\newcommand{\mcL}{\mathcal{L}}
\newcommand{\WW}{\mathcal{W}}

\newcommand{\ex}{\mcE_x}
\newcommand{\ey}{\mcE_y}
\newcommand{\bzero}{\boldsymbol{0}}
\newcommand{\bone}{{\boldsymbol{1}}}
\newcommand{\tB}{{\bone_\mcB}}
\newcommand{\tE}{{\bone_\mcE}}
\newcommand{\bt}{\mathbf{t}}
\newcommand{\bp}{\mathbf{p}}
\newcommand{\bsig}{\mathbf{\Sigma}}
\newcommand{\bpi}{\boldsymbol{\pi}}
\newcommand{\Empty}{\mathtt{Empty}}
\newcommand{\truthf}{\mathtt{t}}
\newcommand{\id}{id}
\newcommand{\coo}{\mathtt{coo\ }}
\newcommand{\mcC}{\mathcal{C}}
\newcommand{\Rec}{\mathpzc{Rec}}
\newcommand{\types}{\mathcal{T}}



%\newcommand{\semof}[1]{\llbracket{#1}\rrbracket^\rel}
\newcommand{\sem}[1]{\llbracket{#1}\rrbracket}
\newcommand{\setsem}[1]{\llbracket{#1}\rrbracket^\set}
\newcommand{\relsem}[1]{\llbracket{#1}\rrbracket^\rel}
\newcommand{\dsem}[1]{\llbracket{#1}\rrbracket^{\mathsf D}}
\newcommand{\setenv}{\mathsf{SetEnv}}
\newcommand{\relenv}{\mathsf{RelEnv}}
\newcommand{\denv}{\mathsf{DEnv}}

\newcommand{\rel}{\mathsf{Rel}}
\newcommand{\setof}[1]{\{#1\}}
\newcommand{\letin}[1]{\texttt{let }#1\texttt{ in }}
\newcommand{\comp}[1]{{\{#1\}}}
\newcommand{\bcomp}[1]{\{\![#1]\!\}}
\newcommand{\beps}{\boldsymbol{\eps}}
%\newcommand{\B}{\mcB}
%\newcommand{\Bo}{{|\mcB|}}

\newcommand{\lmt}{\longmapsto}
\newcommand{\RA}{\Rightarrow}
\newcommand{\LA}{\Leftarrow}
\newcommand{\rras}{\rightrightarrows}
\newcommand{\colim}[2]{{{\underrightarrow{\lim}}_{#1}{#2}}}
\newcommand{\lift}[1]{{#1}\,{\hat{} \; \hat{}}}
\newcommand{\graph}[1]{\langle {#1} \rangle}

\newcommand{\carAT}{\mathsf{car}({\mathcal A}^T)}
\newcommand{\isoAto}{\mathsf{Iso}({\mcA^\to})}
\newcommand{\falg}{\mathsf{Alg}_F}
\newcommand{\CC}{\mathsf{Pres}(\mathcal{A})}
\newcommand{\PP}{\mathcal{P}}
\newcommand{\DD}{D_{(A,B,f)}}
\newcommand{\from}{\leftarrow}
\newcommand{\upset}[1]{{#1}{\uparrow}}
\newcommand{\smupset}[1]{{#1}\!\uparrow}

\newcommand{\Coo}{\mathpzc{Coo}}
\newcommand{\code}{\#}
\newcommand{\nat}{\mathpzc{Nat}}

\newcommand{\eq}{\; = \;}
\newcommand{\of}{\; : \;}
\newcommand{\df}{\; := \;}
\newcommand{\bnf}{\; ::= \;}

\newcommand{\zmap}[1]{{\!{\between\!\!}_{#1}\!}}
\newcommand{\bSet}{\mathbf{Set}}

\newcommand{\dom}{\mathsf{dom}}
\newcommand{\cod}{\mathsf{cod}}
\newcommand{\adjoint}[2]{\mathrel{\mathop{\leftrightarrows}^{#1}_{#2}}}
\newcommand{\isofunc}{\mathpzc{Iso}}
\newcommand{\ebang}{{\eta_!}}
\newcommand{\lras}{\leftrightarrows}
\newcommand{\rlas}{\rightleftarrows}
\newcommand{\then}{\quad\Longrightarrow\quad}
\newcommand{\hookup}{\hookrightarrow}

\newcommand{\spanme}[5]{\begin{CD} #1 @<#2<< #3 @>#4>> #5 \end{CD}}
\newcommand{\spanm}[3]{\begin{CD} #1 @>#2>> #3\end{CD}}
\newcommand{\pushout}{\textsf{Pushout}}
\newcommand{\mospace}{\qquad\qquad\!\!\!\!}

\newcommand{\natur}[2]{#1 \propto #2}

\newcommand{\Tree}{\mathsf{Tree}\,}
\newcommand{\GRose}{\mathsf{GRose}\,}
\newcommand{\List}{\mathsf{List}\,}
\newcommand{\PTree}{\mathsf{PTree}\,}
\newcommand{\Bush}{\mathsf{Bush}\,}
\newcommand{\Forest}{\mathsf{Forest}\,}
\newcommand{\Lam}{\mathsf{Lam}\,}
\newcommand{\LamES}{\mathsf{Lam}^+}
\newcommand{\Expr}{\mathsf{Expr}\,}

\newcommand{\ListNil}{\mathsf{Nil}}
\newcommand{\ListCons}{\mathsf{Cons}}
\newcommand{\LamVar}{\mathsf{Var}}
\newcommand{\LamApp}{\mathsf{App}}
\newcommand{\LamAbs}{\mathsf{Abs}}
\newcommand{\LamSub}{\mathsf{Sub}}
\newcommand{\ExprConst}{\mathsf{Const}}
\newcommand{\ExprPair}{\mathsf{Pair}}
\newcommand{\ExprProj}{\mathsf{Proj}}
\newcommand{\ExprAbs}{\mathsf{Abs}}
\newcommand{\ExprApp}{\mathsf{App}}
\newcommand{\Ptree}{\mathsf{Ptree}}

\newcommand{\kinds}{\mathpzc{K}}
\newcommand{\tvars}{\mathbb{T}}
\newcommand{\fvars}{\mathbb{F}}
\newcommand{\consts}{\mathpzc{C}}
\newcommand{\Lan}{\mathsf{Lan}}
\newcommand{\zerot}{\mathbb{0}}
\newcommand{\onet}{\mathbb{1}}
\newcommand{\bool}{\mathbb{2}}
\renewcommand{\nat}{\mathbb{N}}
%\newcommand{\semof}[1]{[\![#1]\!]}
%\newcommand{\setsem}[1]{\llbracket{#1}\rrbracket^\set}
\newcommand{\predsem}[1]{\llbracket{#1}\rrbracket^\pred}
%\newcommand{\todot}{\stackrel{.}{\to}}
\newcommand{\todot}{\Rightarrow}
\newcommand{\bphi}{{\bm \phi}}

\newcommand{\bm}[1]{\boldsymbol{#1}}

\newcommand{\cL}{\mathcal{L}}
\newcommand{\T}{\mathcal{T}}
\newcommand{\Pos}{P\!}
%\newcommand{\Pos}{\mathcal{P}\!}
\newcommand{\Neg}{\mathcal{N}}
\newcommand{\Hf}{\mathcal{H}}
\newcommand{\V}{\mathbb{V}}
\newcommand{\I}{\mathcal{I}}
\newcommand{\Set}{\mathsf{Set}}
\newcommand{\Nat}{\mathsf{Nat}}
\newcommand{\Homrel}{\mathsf{Hom_{Rel}}}
\newcommand{\CV}{\mathcal{CV}}
\newcommand{\lan}{\mathsf{Lan}}
\newcommand{\Id}{\mathit{Id}}
\newcommand{\mcA}{\mathcal{A}}
\newcommand{\inl}{\mathsf{inl}}
\newcommand{\inr}{\mathsf{inr}}
\newcommand{\case}[3]{\mathsf{case}\,{#1}\,\mathsf{of}\,\{{#2};\,{#3}\}}
\newcommand{\tin}{\mathsf{in}}
\newcommand{\fold}{\mathsf{fold}}
\newcommand{\Eq}{\mathsf{Eq}}
\newcommand{\Hom}{\mathsf{Hom}}
\newcommand{\curry}{\mathsf{curry}}
\newcommand{\uncurry}{\mathsf{uncurry}}
\newcommand{\eval}{\mathsf{eval}}
\newcommand{\apply}{\mathsf{apply}}

\newcommand{\ar}[1]{\##1}
\newcommand{\mcG}{\mathcal{G}}
\newcommand{\mcH}{\mathcal{H}}
\newcommand{\TV}{\mathpzc{V}}

\newcommand{\essim}[1]{\mathsf{EssIm}(#1)}
\newcommand{\hra}{\hookrightarrow}

\newcommand{\ol}[1]{\overline{#1}}
\newcommand{\ul}[1]{\underline{#1}}
\newcommand{\op}{\mathsf{op}}
\newcommand{\trige}{\trianglerighteq}
\newcommand{\trile}{\trianglelefteq}
\newcommand{\LFP}{\mathsf{LFP}}
\newcommand{\LAN}{\mathsf{Lan}}
%\newcommand{\Mu}{{\mu\hskip-4pt\mu}}
\newcommand{\Mu}{{\mu\hskip-5.5pt\mu}}
%\newcommand{\Mu}{\boldsymbol{\upmu}}
\newcommand{\Terms}{\mathpzc{Terms}}
\newcommand{\Ord}{\mathpzc{Ord}}
\newcommand{\Anote}[1]{{\color{blue} {#1}}}
\newcommand{\Pnote}[1]{{\color{red} {#1}}}

\newcommand{\greyout}[1]{{\color{gray} {#1}}}
\newcommand{\ora}[1]{\overrightarrow{#1}}

%\newcommand{\?}{{.\ }}
%\theoremheaderfont{\scshape}
%\theorembodyfont{\normalfont}
%\theoremseparator{.\ \ }
\newtheorem{thm}{Theorem}
\newtheorem{dfn}[thm]{Definition}
\newtheorem{prop}[thm]{Proposition}
\newtheorem{cor}[thm]{Corollary}
\newtheorem{lemma}[thm]{Lemma}
\newtheorem{rmk}[thm]{Remark}
\newtheorem{expl}[thm]{Example}
\newtheorem{notn}[thm]{Notation}
%\theoremstyle{nonumberplain}
%\theoremsymbol{\Box}


\theoremstyle{definition}
\newtheorem{exmpl}{Example}

\renewcommand{\greyout}[1]{} %{{\color{gray} {#1}}} -- toggle to remove greyed text

\newcommand{\ininv}[2]{(\tin^{-1}_{\onet +
  \beta \times \phi (\phi\beta)})_{#1}\, #2}
\newcommand{\emptyfun}{{[]}}
\newcommand{\cal}{\mathcal}
%\newcommand{\fold}{\mathit{fold}}
\newcommand{\F}{\mathcal{F}}
\renewcommand{\G}{\mathcal{G}}
\newcommand{\N}{\mathcal{N}}
\newcommand{\E}{\mathcal{E}}
\newcommand{\B}{\mathcal{B}}
\renewcommand{\P}{\mathcal{A}}
\newcommand{\pred}{\mathsf{Fam}}
\newcommand{\env}{\mathsf{Env}}
\newcommand{\set}{\mathsf{Set}}
\renewcommand{\S}{\mathcal S}
\renewcommand{\C}{\mathcal{C}}
\newcommand{\D}{\mathcal{D}}
\newcommand{\A}{\mathcal{A}}
\renewcommand{\id}{\mathit{id}}
\newcommand{\map}{\mathsf{map}}
\newcommand{\pid}{\underline{\mathit{id}}}
\newcommand{\pcirc}{\,\underline{\circ}\,}
\newcommand{\pzero}{\underline{0}}
\newcommand{\pone}{\underline{1}}
\newcommand{\psum}{\,\underline{+}\,}
\newcommand{\pinl}{\underline{\mathit{inL}}\,}
\newcommand{\pinr}{\underline{\mathit{inR}}\,}
\newcommand{\ptimes}{\,\underline{\times}\,}
\newcommand{\ppi}{\underline{\pi_1}}
\newcommand{\pppi}{\underline{\pi_2}}
\newcommand{\pmu}{\underline{\mu}}
\newcommand{\semmap}{\mathit{map}}
\newcommand{\subst}{\mathit{subst}}

\newcommand{\tb}[1]{~~ \mbox{#1} ~~}
\newcommand{\listt}[1]{(\mu \phi. \lambda \beta . \onet + \beta \times
  \phi \beta) #1} 
\newcommand{\filtype}{\Nat^\emptyset 
 (\Nat^\emptyset \, \alpha \, \mathit{Bool})\, (\Nat^\emptyset 
  (List \, \alpha) \, (List \, \alpha))} 
\newcommand{\filtypeGRose}{\Nat^\emptyset 
 (\Nat^\emptyset \, \alpha \, \mathit{Bool})\, (\Nat^\emptyset 
  (\mathit{GRose}\,\psi \, \alpha) \, (\mathit{GRose}\,\psi \, (\alpha
  + \onet)))} 
\newcommand{\maplist}{\mathit{map}_{\lambda A. \setsem{\emptyset; \alpha
      \vdash \mathit{List} \, \alpha} \rho[\alpha := A]}} 
\newcommand{\PLeaves}{\mathsf{PLeaves}}
\newcommand{\swap}{\mathsf{swap}}
\newcommand{\reverse}{\mathsf{reverse}}
\newcommand{\Bcons}{\mathit{Bcons}}
\newcommand{\Bnil}{\mathit{Bnil}}

\title[Free Theorems for Nested Types]{Free Theorems for
  %Primitive
  Nested Types} %% [Short Title] is optional; when present,
                         %% will be used in header instead of Full
                         %% Title.
%\titlenote{with title note}             %% \titlenote is optional;
                                        %% can be repeated if necessary;
                                        %% contents suppressed with 'anonymous'
%\subtitle{Subtitle}                     %% \subtitle is optional
%\subtitlenote{with subtitle note}       %% \subtitlenote is optional;
                                        %% can be repeated if necessary;
                                        %% contents suppressed with 'anonymous'


%% Author information
%% Contents and number of authors suppressed with 'anonymous'.
%% Each author should be introduced by \author, followed by
%% \authornote (optional), \orcid (optional), \affiliation, and
%% \email.
%% An author may have multiple affiliations and/or emails; repeat the
%% appropriate command.
%% Many elements are not rendered, but should be provided for metadata
%% extraction tools.

%% Author with single affiliation.
\author{Patricia Johann, Enrico Ghiorzi, and Daniel Jeffries}
%\authornote{with author1 note}          %% \authornote is optional;
%                                        %% can be repeated if necessary
%\orcid{nnnn-nnnn-nnnn-nnnn}             %% \orcid is optional
\affiliation{
%  \position{Position1}
%  \department{Department1}              %% \department is recommended
  \institution{Appalachian State University}            %% \institution is required
%  \streetaddress{Street1 Address1}
%  \city{City1}
%  \state{State1}
%  \postcode{Post-Code1}
%  \country{Country1}                    %% \country is recommended
}
\email{johannp@appstate.edu, ghiorzie@appstate.edu, jeffriesd@appstate.edu}          %% \email is recommended


%% macros
\newcommand{\nats}{\mu \beta. \onet + \beta}

\begin{document}

% Equality type GADT

\begin{comment}

    \[\begin{array}{l}
      \mathtt{data\; Eql \;(A : Set)\;(B : Set)\;:\;Set\;where}\\
    \hspace*{0.4in}\mathtt{refl\;:\; \;Eql \;A\;A}
    \end{array}\]

    % singleton Z - not a GADT
    \[\begin{array}{l}
    \mathtt{data\; Z \;:\;Set\;where}\\
    \hspace*{0.4in}\mathtt{szero\;:\; \;Z}
    \end{array}\]

    % singleton S - not a GADT
    \[\begin{array}{l}
      \mathtt{data\; S \;(A : Set)\;:\;Set\;where}\\
    \hspace*{0.4in}\mathtt{ssucc\;:\; A \rightarrow S\;A}
    \end{array}\]

    % GNat  - GADT
    \[\begin{array}{l}
    \mathtt{data\; GNat\;(A : Set)\;:\;Set\;where}\\
    \hspace*{0.4in}\mathtt{gzero\;:\; GNat\;Z}\\
      \hspace*{0.4in}\mathtt{gsucc\;:\;GNat\;A \rightarrow GNat\;(S\;A)}
    \end{array}\]


    % GVec  - GADT
    % --- uses GADT (GNat) in subscript!
    \[\begin{array}{l}
      \mathtt{data\; GVec\;(A : Set)\;(N : Set)\;:\;Set\;where}\\
      \hspace*{0.4in}\mathtt{gnil\;:\; GVec\;A\;(GNat\;Z)}\\
      \hspace*{0.4in}\mathtt{gcons\;:\; A \rightarrow 
          GVec\;A\;(GNat\;N) \rightarrow GVec\;A\;(GNat\;(S\;N))} 
    \end{array}\]

    % PTree 
    \[\begin{array}{l}
      \mathtt{data\; PTree\;(A : Set)\;:\;Set\;where}\\
      \hspace*{0.4in}\mathtt{pleaf\;:\; A \rightarrow PTree\;A}\\
      \hspace*{0.4in}\mathtt{pnode\;:\; PTree\;(A \times A) \rightarrow PTree\;A} \\
    \end{array}\]


    % GTree
    \[\begin{array}{l}
      \mathtt{data\; GTree\;(A : Set)\;(N : Set)\;:\;Set\;where}\\
      \hspace*{0.4in}\mathtt{gleaf\;:\; A \rightarrow GTree\;A\;Z}\\
      \hspace*{0.4in}\mathtt{gnode\;:\; GTree\;A\;N \rightarrow GTree\;A\;N 
            \rightarrow GTree\;A\;(S\;N)} \\
    \end{array}\]

    % ETree
    \[\begin{array}{l}
      \mathtt{data\; ETree\;(A : Set)\;(N : Set)\;:\;Set\;where}\\
      \hspace*{0.4in}\mathtt{eleaf\;:\; A \rightarrow Eql\;N\;Z \rightarrow ETree\;A\;N}\\
      \hspace*{0.4in}\mathtt{enode\;:\; Eql\;N\;(S\;M) \rightarrow ETree\;A\;M \rightarrow ETree\;A\;M 
            \rightarrow ETree\;A\;N} \\
    \end{array}\]


    % lambda terms from wobbly types (minus a few constructors)
    \[\begin{array}{l}
      \mathtt{data\; Term\;(A : Set)\;:\;Set\;where}\\
      \hspace*{0.4in}\mathtt{lit\;:\; \nat \rightarrow Term\;\nat}\\
      \hspace*{0.4in}\mathtt{inc\;:\; Term\;\nat \rightarrow Term\;\nat}\\
      \hspace*{0.4in}\mathtt{isZ\;:\; Term\;\nat \rightarrow Term\;Bool}\\
      \hspace*{0.4in}\mathtt{pair\;:\; Term\;A \rightarrow Term\;B \rightarrow Term\;(A \times B)}
    \end{array}\]


    % 
    \[\begin{array}{l}
    % Natural numbers in our calculus
    \nat = (\mu \beta. \onet + \beta) \\
    % Z in our calculus
    Z = \onet \\
    % S in our calculus
    S \, \beta = (\mu \phi. \lambda \alpha. \alpha) \beta \\
    % GNat in our calculus
    \mathit{GNat} \, \beta = 
    (\mu \phi. \lambda \alpha. 
      (\Lan^\emptyset_Z \, 1)\alpha 
      + (\Lan^\gamma_{S\,\gamma} \, \phi \gamma)\alpha)\beta \\
    % GVec in our calculus
    \mathit{GVec} \, \beta_1 \, \beta_2 = 
      (\mu \phi. \lambda \alpha_1, \alpha_2. 
        (\Lan^\gamma_{\gamma, \mathit{GNat}\,Z} \, 1)(\alpha_1, \alpha_2)
      + (\Lan^{\gamma_1,\gamma_2}_{\gamma_1, \mathit{GNat} \, (S\,\gamma_2)}
            \,\gamma_1 \times \phi (\gamma_1, \mathit{GNat} \, \gamma_2))(\alpha_1, \alpha_2)
            )(\beta_1, \beta_2) \\
      \color{red} \mbox{Note: GVec uses GNat as a subscript}\\
    % GTree in our calculus
      \mathit{GTree} \, \beta_1 \, \beta_2 =
        (\mu \phi. \lambda \alpha_1, \alpha_2.
          (\Lan^\gamma_{\gamma, Z} \, \gamma)(\alpha_1, \alpha_2)
        + (\Lan^{\gamma_1, \gamma_2}_{\gamma, S\,\gamma_2} \, \phi\,(\gamma_1,\gamma_2) \times \phi\,(\gamma_1,\gamma_2))(\alpha_1, \alpha_2)
        )(\beta_1, \beta_2) \\
    % PTree in our calculus
      \mathit{PTree} \, \beta = 
        (\mu \phi. \lambda \alpha. \alpha + \phi\,(\alpha \times \alpha))\beta \\

    % Term data type in our calculus
    \mathit{Term} \, \beta = (\mu \phi. \lambda \alpha. 
      (\Lan^\emptyset_\nat \, \nat)\alpha
    + (\Lan^\emptyset_\nat \, \phi \nat)\alpha
    + (\Lan^\emptyset_{\mathit{Bool}} \, \phi \nat)\alpha
    + (\Lan^{\gamma_1, \gamma_2}_{\gamma_1 \times \gamma_2} \, \phi \gamma_1 \times \phi \gamma_2)\alpha)\beta\\
    % Eql data type 
      \mathit{Eql} \, \beta_1 \, \beta_2 = (\Lan^\gamma_{\gamma, \gamma} 1)(\beta_1, \beta_2)
    \end{array}\]
  \end{comment}


\section{Programming with $\Lan$ types}
We can use $\Lan$ types in several different ways in our calculus. In this note
we will explore how to use $\Lan$ types to encode GADTs in our calculus. 
These $\Lan$ types / GADTs have several useful applications in our calculus.
In addition to allowing for more general return types in data type constructors,
$\Lan$ can be combined with singleton types to mimick some dependent types.

We can also define existential types and an equality type using $\Lan$,
but the latter is not really useful in the absence of type refinement(?).

The usefulness of existential types is also somewhat limited depending on 
what is allowed in the context of $G$ in the $\partial$ rule. 







\subsection{Using Lan in a constructor}
Our calculus does not include "GADT-syntax" (like Agda, Haskell + GADTs) in which 
one can make the "return type" explicit.  Instead,
we write a data type as the fixed point of the sum of its constructors. 
To produce a term of some data type with $n$ constructors $(\mu \phi. \lambda \ol{\alpha}. F_1 + ... + F_n)\ol{G}$,
we first construct a term 
$$t : F_i [\phi :=_\beta (\mu \phi. \lambda \ol{\alpha}. F_1 + ... + F_n)\ol{\beta}][\ol{\alpha := G}]$$
and invoke $\tin$ and some combination of $\inl/\inr$ to get 
$$(\tin_{F_1 + ... + F_n})_{\ol{G}} \, (..\inl/\inr..  \,t) : (\mu \phi. \lambda \ol{\alpha}. F_1 + ... + F_n)\ol{G}$$

So how are
these restrictions on the return types enforced in our calculus?
Although $\tin_{F_1 + ... + F_n}$ is a natural transformation from 
$$(F_1 + ... + F_n)[\phi :=_\beta (\mu \phi. \lambda \ol{\alpha}. F_1 + ... + F_n)\ol{\beta}][\ol{\alpha := \beta}]$$
to 
$$(\mu \phi.\ol\alpha. F_1 + ... + F_n)\ol\beta$$
we can only use the component $(\tin_{F_1 + ... + F_n})_{\ol{G}}$ when
$F_i [\phi :=_\beta (\mu \phi. \lambda \ol{\alpha}. F_1 + ... + F_n)\ol{\beta}][\ol{\alpha := G}]$
is inhabited for some $1 \le i \le n$. 

So when $F_i = (\Lan^{\ol\gamma}_{\ol K} \,H)\ol\alpha$, we know from the $\int$ rule
that we can only introduce terms of type $(\Lan^{\ol\gamma}_{\ol K} \,H)\ol{K[\ol{\gamma := A}]}$
for some types $\ol{A}$, so we can never use a component of $\tin_{F_1 + ... F_n}$ unless it has the form
$(\tin_{F_1 + ... F_n})_{\ol{K[\ol{\gamma := A}]}}$


% $\alpha_j = K_j[\ol{\gamma := A}]$ for $j = 1,...,|\ol{K}|$ and types $\ol{A}$.
For example, consider the third constructor in
the type of simply-typed lambda terms:
\[\begin{array}{l}
  \mathtt{data\; Term\;(A : Set)\;:\;Set\;where}\\
  \hspace*{0.4in}\mathtt{lit\;:\; \nat \rightarrow Term\;\nat}\\
  \hspace*{0.4in}\mathtt{inc\;:\; Term\;\nat \rightarrow Term\;\nat}\\
  \hspace*{0.4in}\mathtt{isZ\;:\; Term\;\nat \rightarrow Term\;Bool}\\
  \hspace*{0.4in}\mathtt{pair\;:\; Term\;A \rightarrow Term\;B \rightarrow Term\;(A \times B)} \\ \\

\mathit{Term} \, \beta = (\mu \phi. \lambda \alpha. 
  (\Lan^\emptyset_\nat \, \nat)\alpha
+ (\Lan^\emptyset_\nat \, \phi \nat)\alpha
+ (\Lan^\emptyset_{\mathit{Bool}} \, \phi \nat)\alpha
+ (\Lan^{\gamma_1, \gamma_2}_{\gamma_1 \times \gamma_2} \, \phi \gamma_1 \times \phi \gamma_2)\alpha)\beta
\end{array}\]

In this case, $F_i = (\Lan^{\emptyset}_{\mathit{Bool}} \, \phi\,\nat)\alpha$, which is only inhabited when $\alpha = \mathit{Bool}$.
The only component of $\tin_{F_1 + ... F_n}$ we can hope to use with this constructor is $(\tin_{F_1 + ... F_n})_\mathit{Bool}$,
and this restriction corresponds exactly with the return type of \texttt{isZ} in the Agda definition. 

% This restriction in the $\int$ rule also restricts the choice of $\ol{B}$ in the $\partial$ rule,
% in practice.  



\section{Singleton Types and Fake Dependent Types}
Since the return types of constructors in a GADT can vary, 
the type of a term can now potentially give us information about its value.

We have already seen the $\mathit{Term}$ datatype, which ensures that 
the lambda terms which inhabit the datatype are well-typed. This additional
structure can also make programs more efficient in a practical setting
by eliminating some pattern matching cases. 

It turns out that we can mimick dependent types by using GADTs with "singleton types", 
which are type-level representations of terms. For instance, we can 
represent type-level natural numbers using $Z = \onet$ and 
$S \, \beta = (\mu \phi. \lambda \alpha. \alpha) \beta$.

Using $Z$ and $S$ we can define a type of length-indexed lists:
\[\begin{array}{l}
  \mathtt{data\; Vec\;(A : Set)\;(N : Set)\;:\;Set\;where}\\
  \hspace*{0.4in}\mathtt{gnil\;:\; Vec\;A\;\;Z}\\
  \hspace*{0.4in}\mathtt{gcons\;:\; A \rightarrow 
      Vec\;A\;N \rightarrow Vec\;A\;(S\;N)}  \\ \\

\mathit{Vec} \, \beta_1 \beta_2 = 
  (\mu \phi. \lambda \alpha_1, \alpha_2. 
    (\Lan^\gamma_{\gamma, Z} \, 1)(\alpha_1, \alpha_2)
  + (\Lan^{\gamma_1,\gamma_2}_{\gamma_1, S\,\gamma_2}
        \,\gamma_1 \times \phi (\gamma_1, \gamma_2))(\alpha_1, \alpha_2)
        )(\beta_1, \beta_2)
\end{array}\]
and for convenience we can define some constructors \\(abusing notation to write
$\tin_{\mathit{Vec}}$ for $\tin_{
    (\Lan^\gamma_{\gamma, Z} \, 1)(\alpha_1, \alpha_2)
  + (\Lan^{\gamma_1,\gamma_2}_{\gamma_1, S\,\gamma_2}
        \,\gamma_1 \times \phi (\gamma_1, \gamma_2))(\alpha_1, \alpha_2)}$ and similar below):

\[\begin{array}{l}
  \mathit{nil}: \Nat^\alpha \, \onet \, (\mathit{List}\,\alpha) \\
  \vspace{0.05in}
  \mathit{nil} = L_\alpha \ast. (\tin_{\onet + \alpha \times \phi\alpha})_\alpha (\inl \, \ast) \\
  \mathit{cons} : \Nat^\alpha \, (\alpha \times \mathit{List}\,\alpha)\, (\mathit{List} \,\alpha) \\
  \vspace{0.05in}
  \mathit{cons} = L_\alpha t. (\tin_{\onet + \alpha \times \phi\alpha})_\alpha (\inr \, t) \\
  \mathit{vnil} : \Nat^\alpha \, \onet \, (\mathit{Vec} \,\alpha \,Z) \\
  \vspace{0.05in}
  \mathit{vnil} = L_\alpha \ast. (\tin_{\mathit{Vec}})_{\alpha, Z} (\inl \,(\int_{(\gamma, Z), \onet} \ast)) \\
  \mathit{vcons} : \Nat^{\alpha, n} \, (\alpha \times \mathit{Vec} \, \alpha \, n)\, (\mathit{Vec} \, \alpha\,(S\,n)) \\
  \mathit{vcons} = L_{\alpha, n} t. (\tin_{\mathit{Vec}})_{\alpha, (S \,n)} (\inr\, (\int_{(\gamma_1, S\,\gamma_2), \gamma_1 \times \mathit{Vec} \, \gamma_1\,\gamma_2} t))
\end{array}\]

We can write a function to convert a vector into a list: 
\[\begin{array}{l}
  \mathit{vecToList} : \Nat^{\alpha, n} \, (\mathit{Vec} \, \alpha \, n) \, (\mathit{List} \, \alpha) \\
  \vspace{0.05in}
  \mathit{vecToList} = (\fold_{\mathit{Vec}})_\emptyset \mathit{alg} \\
  \mathit{alg} : \Nat^{\alpha, n} \, 
  ((\Lan^\gamma_{\gamma, Z} \, \onet)(\alpha, n) + (\Lan^{\gamma_1, \gamma_2}_{\gamma_1, S\,\gamma_2} \, \gamma_1 \times \mathit{List}\, \gamma_1)(\alpha, n))
  \; (\mathit{List}\,\alpha)\\
  \mathit{alg} = L_{\alpha, n} \, t. 
    \mathsf{case} \; t \; \mathsf{of} \; \{ \\ 
    \hspace{0.5in} vn \mapsto \mathit{nil}_\alpha \ast; \\
    \hspace{0.5in} vc \mapsto \mathit{cons}_\alpha 
    (\partial^{\beta \times \mathit{List}\,\beta, (\gamma_1, S\,\gamma_2)}_{\gamma_1 \times \mathit{List}\,\gamma_1} \, (\id_{\gamma_1\times \mathit{List}\,\gamma_1}) \, vc) \\
    \}
\end{array}\]

In $alg$, we use the following instance of the $\Lan$-elimination rule:
\[\begin{array}{c}
  \AXC{$\vdash \id_{\gamma_1 \times \mathit{List}\, \gamma_1} : \Nat^{\alpha,n,\gamma_1,\gamma_2}\,
  (\gamma_1 \times \mathit{List} \, \gamma_1)\;(\gamma_1 \times \mathit{List} \, \gamma_1)$}
  \AXC{$\emptyset;\alpha,n~|~\emptyset \vdash vc : (\Lan^{\gamma_1,\gamma_2}_{\gamma_1, S\,\gamma_2} \gamma_1 \times \mathit{List} \,\gamma_1)\,(\alpha, n) $}
  \BIC{$\emptyset;\alpha,n~|~\emptyset \vdash \partial^{\beta_1 \times \mathit{List}\,\beta_1, (\gamma_1, S\,\gamma_2)}_{\gamma_1 \times \mathit{List}\,\gamma_1} \eta\,vc :
  (\beta_1 \times \mathit{List}\, \beta_1)[\beta_1 := \alpha][\beta_2 := n]$}
\DisplayProof
\end{array}\]
to produce a term of type $\alpha \times \mathit{List}\, \alpha$.
We can also convert lists to vectors, but this is tackled in the next section 
with existential types.


\subsection{Safe(?) head}
Given a the type of length-indexed vectors, one might ask whether we can 
write a $head$ function of type $\Nat^{\alpha, n} (\mathit{Vec}\,\alpha\,(S\,n))\;(\alpha)$ 
in which we only need to give the case for the $\mathit{vcons}$ constructor. 


\[\begin{array}{l}
  \mathit{head} : \Nat^{\alpha, n} (\mathit{Vec}\,\alpha\,(S\,n))\;(\alpha)\\
  \mathit{head} = L_{\alpha, n} t. \mathsf{case} \, ((\tin^{-1}_{\mathit{Vec}})_{\alpha, S\,n} t) \, \mathsf{of} \{ \\
  \hspace{0.5in} vn : (\Lan^\gamma_{\gamma, Z} \, 1)(\alpha, S\,n) \mapsto {\color{red}???}; \\
  \hspace{0.5in} vc : (\Lan^{\gamma_1, \gamma_2}_{\gamma_1, S\,\gamma_2} \, 
  \gamma_1 \times \mathit{Vec}\,\gamma_1\,\gamma_2)(\alpha, S\,n) 
  \mapsto  \partial^{\beta_1, (\gamma_1, \gamma_2)}_{\gamma_1 \times \mathit{Vec}\,\gamma_1\,\gamma_2} \, (L_{\alpha, n, \gamma_1, \gamma_2}\, x. \pi_1 x) \, vc \\
  \}
\end{array}\]

In the second case above, we can use the following instance of the $\Lan$-elimination rule
to extract a term of type $\alpha$ from a term of type $(\Lan^{\gamma_1, \gamma_2}_{\gamma_1, S\,\gamma_2} \, 
  \gamma_1 \times \mathit{Vec}\,\gamma_1\,\gamma_2)(\alpha, n)$:
\[\begin{array}{c}
  \AXC{$\vdash \eta : \Nat^{\alpha,n,\gamma_1,\gamma_2}\,
  (\gamma_1 \times \mathit{Vec} \, \gamma_1 \,\gamma_2)\;(\gamma_1)$}
  \AXC{$\emptyset;\alpha,n~|~ vc \vdash vc : (\Lan^{\gamma_1,\gamma_2}_{\gamma_1, S\,\gamma_2} \gamma_1 \times \mathit{Vec} \,\gamma_1\,\gamma_2)\,(\alpha, S\,n) $}
  \BIC{$\emptyset;\alpha,n~|~ vc \vdash \partial^{\beta_1, (\gamma_1, S\,\gamma_2)}_{\gamma_1 \times \mathit{Vec}\,\gamma_1\,\gamma_2}\, \eta\,vc :
  (\beta_1)[\beta_1 := \alpha][\beta_2 := S\,n]$}
\DisplayProof
\end{array}\]

Unfortunately, we have no way of producing a term of type $\alpha$ in the first case. 
In a practical setting, the type unification algorithm would determine at this point
that the first constructor pattern need not be matched, because $Z$ cannot be unified 
with $S\,n$ for any $n$. It is also true in our calculus that, due to the $\Lan$-introduction rule,
it is impossible to construct a term of type $(\Lan^{\ol{\gamma}}_{\ol{K}}\, F)\ol{B}$ unless
$B_i = K_{i}[\ol{\gamma := A}]$ for $i = 1,...,|\ol{K}|$ and some types $\ol{A}$.
In particular, it is impossible to construct a term of type 
$(\Lan^\gamma_{\gamma, Z} \, 1)(\alpha, S\,n)$, so assuming its existence should
allow us to do something akin to $\bot$-eliimination in the first case above.
{\color{blue} I don't know how we could do this because it seems we would need type-equality
at the meta-level in order to have a rule that can check the arguments of a $\Lan$ type in this way.
This concern is probably orthogonal to the goals of this project, but without it, the
"fake dependent types" are not incredibly useful.}

{\color{red} Note about S and Z }


Note that we cannot write a function 
from the type-level natural numbers to 
the natural numbers data type. We 
can write each instance, e.g. $Zto0 : \Nat^{\emptyset} (Z) \, (\nats)$,
$SZto1 : \Nat^{\emptyset} (S\,Z) \, (\nats)$, but we cannot 
write them all in a single function because S and Z are separate types. 
(What would the type be? $\Nat^{\alpha} (Z + S \,\alpha) (\nats)$? This is too 
generic because it must be defined for all $\alpha$. )
We address this issue with the following GADT:

\[\begin{array}{l}
\mathtt{data\; GNat\;(A : Set)\;:\;Set\;where}\\
\hspace*{0.4in}\mathtt{gzero\;:\; GNat\;Z}\\
  \hspace*{0.4in}\mathtt{gsucc\;:\;GNat\;A \rightarrow GNat\;(S\;A)}
\end{array}\]
Note that using $\mathit{GNat}$ in the definition of $\mathit{Vec}$ 
would required having a $\Lan$ type ($\mathit{GNat}$) in the subscript of another $\Lan$
(a constructor for $\mathit{Vec}$).

\section{Existential Types} 
The following defines an existential type in Agda:
\[\begin{array}{l}
\mathtt{data\; Exists \;(F : Set \rightarrow Set)\;:\;Set_1\;where}\\
\hspace*{0.4in}\mathtt{e\;:\; F \,A \rightarrow Exists\;F}\\
\end{array}\]
The aspect that makes it existential is that the type $\mathtt{A}$ used in the constructor argument is hidden in the return type.
To provide a witness to the fact that there exists a type $\mathtt{A}$ such that $\mathtt{List\, A}$
is inhabited, we just need to provide an inhabitant of $\mathtt{List \, A}$ for some type $\mathtt{A}$.

Unfortunately, we cannot define an exact analogue (with closed type) 
of the $\mathtt{Exists}$ datatype in our calculus because it is parameterized over a $1$-ary variable ($\mathtt{F}$),
and we can only bind 0-ary variables in our types (we can bind other variables with $\mu$, but that isn't what we want here).

However, we can define an existential type for a variable $\psi$ as $\emptyset; \psi \vdash (\Lan^{\ol\alpha}_\emptyset\,\psi\,\ol\alpha)$,
or given some type $\Gamma; \Phi, \ol\gamma \vdash F$ we can existentially quantify $\ol\gamma$ 
in $F$ with $\Gamma; \Phi \vdash (\Lan^{\ol\alpha}_\emptyset \, F[\ol{\gamma := \alpha}])$. 

The introduction and elimination rules for $\Lan$ give these types their "existential" character:

\[\begin{array}{c}
\AXC{$\Gamma; \Phi,\ol{\alpha}\vdash F$}
  \AXC{$\ol{\Gamma;\Phi \vdash A}$} 
  \AXC{$\Gamma;\Phi~|~\Delta \vdash t : F[\ol{\alpha :=A}]$}
\TIC{$\Gamma;\Phi~|~\Delta \vdash \int_{\emptyset,F} t :
  (\Lan^{\ol\alpha}_{\emptyset}\,  F)$}
\DisplayProof
\\\\
\AXC{$\Gamma;\emptyset~|~\Delta \vdash \eta : \Nat^{\Phi,\ol\alpha}\,
  F\;G$}
\AXC{$\Gamma;\Phi~|~\Delta \vdash t : (\Lan^{\ol\alpha}_{\emptyset} F) $}
\BIC{$\Gamma;\Phi~|~\Delta \vdash \partial^{G, \emptyset}_F \eta\,t :
 G$}
\DisplayProof
\end{array}\]

The introduction rule is precisely as we expect:
to introduce an existential type, we just need a term of type $F[\ol{\alpha := A}]$
for some types $\ol{A}$. The elimination rule also behaves as expected: 
if we have a term $t$ of existential type and a natural transformation from $F$ to $G$, 
we don't need to know which instance of $F$ produced $t$ in order to reach $G$, because 
the natural transformation is defined for all instances of $F$.
For example, if we have a list whose elements are of an unknown type, we can still
compute its length given a natural transformation of type $\Nat^\alpha (\mathit{List}\,\alpha) (\nat)$.



\subsection{From List to Vec (PTree to GTree)}
In order to write a natural transformation from lists to length-indexed vectors,
we must existentially quantify the length variable in the return type. Otherwise,
we must define the natural transformation for all possible values of $n$.
The function is defined as follows:

{\color{red} I am having issues defining the list to vector function at the moment, but I will define an analogous
embedding function for $\mathit{PTree}$ and $\mathit{GTree}$:}
\[\begin{array}{l}
  \mathtt{data\; PTree\;(A : Set)\;:\;Set\;where}\\
  \hspace*{0.4in}\mathtt{pleaf\;:\; A \rightarrow PTree\;A}\\
  \hspace*{0.4in}\mathtt{pnode\;:\; PTree\;(A \times A) \rightarrow PTree\;A} \\ \\
  \mathit{PTree} \, \beta = 
    (\mu \phi. \lambda \alpha. \alpha + \phi\,(\alpha \times \alpha))\beta \\ \\ \\

  \mathtt{data\; GTree\;(A : Set)\;(N : Set)\;:\;Set\;where}\\
  \hspace*{0.4in}\mathtt{gleaf\;:\; A \rightarrow GTree\;A\;Z}\\
  \hspace*{0.4in}\mathtt{gnode\;:\; GTree\;A\;N \rightarrow GTree\;A\;N 
        \rightarrow GTree\;A\;(S\;N)} \\ \\

  \mathit{GTree} \, \beta_1 \, \beta_2 =
    (\mu \phi. \lambda \alpha_1, \alpha_2.
      (\Lan^\gamma_{\gamma, Z} \, \gamma)(\alpha_1, \alpha_2)
    + (\Lan^{\gamma_1, \gamma_2}_{\gamma, S\,\gamma_2} \, \phi\,(\gamma_1,\gamma_2) \times \phi\,(\gamma_1,\gamma_2))(\alpha_1, \alpha_2)
    )(\beta_1, \beta_2) 
\end{array}\]

%%%%% haven't quite figured this out 
%%%%%  
%% \[\begin{array}{l}
%%   \mathit{listToVec} : \Nat^\alpha \, (\mathit{List}\,\alpha)\,(\Lan^n_\emptyset \, \mathit{Vec}\,\alpha\,n) \\
%%   \vspace{0.05in}
%%   \mathit{listToVec} = (\fold_{\onet + \beta \times \phi\,\beta})_\emptyset \, \mathit{alg} \\
%%   \mathit{alg} : \Nat^\alpha (\onet + \alpha \times (\Lan^n_\emptyset \, \mathit{Vec}\,\alpha\,n)) 
%%     \; (\Lan^n_\emptyset \,\mathit{Vec}\,\alpha\,n) \\
%%   \mathit{alg} = L_\alpha \,s. \mathsf{case} \;s \; \mathsf{of} \; \{ \\
%%     \hspace{0.5in} \ast \mapsto \int_{\emptyset, \mathit{Vec}\,\alpha\,n} (\mathit{vnil}_\alpha \ast)  ; \\
%%     \hspace{0.5in} t 
%%     : \alpha \times (\Lan^n_\emptyset \,\mathit{Vec}\,\alpha\,n) 
%%     \mapsto ??? \\
%%     \}
%% \end{array}\]

We first define some constructors for $\mathit{GTree}$ and some helper functions for 
$\mathit{ptreeToGTree}$. There is some trickiness with variable names in the $\Lan$ types, and so we define
$\mathit{alphaRename}$ to deal with this. 

The origin of this issue is in the typing of the 
$\Lan$-elimination rule used in the second case of $alg$ below. In particular, it does
not seem possible to let the codomain of $\eta$ be typed as $\emptyset; \alpha, n \vdash (\Lan^n_\emptyset \, \mathit{GTree}\,\alpha\,n)$,
even though it is legal to form $\emptyset; \alpha \vdash (\Lan^n_\emptyset \,\mathit{GTree}\,\alpha\,n)$ and 
then weaken to get  $\emptyset; \alpha, n \vdash (\Lan^n_\emptyset \, \mathit{GTree}\,\alpha\,n)$. The issue
is that a natural transformation of type 
$\Nat^{\alpha,n}\,
  (\mathit{GTree}\,(\alpha\times\alpha)\,n) \;(\Lan^n_\emptyset\,\mathit{GTree}\,\alpha\,n)
  $ 
apparently cannot be defined directly.
To define such a natural transformation, we would need to change $\mathit{helper}$'s type and definition:  \\ \\
  $\mathit{helper} : \Nat^{\alpha,n}\,
  (\mathit{GTree}\,(\alpha\times\alpha)\,n) \;(\Lan^{\color{red}n}_\emptyset\,\mathit{GTree}\,\alpha\,{\color{red}n})$ \\
  $\mathit{helper} = L_{\alpha, n} \, t. \, \int_{\emptyset, \mathit{GTree}\,\alpha\,{\color{red}n}}  \,  (\mathit{gnode}_{\alpha, n}  \,  (\mathit{unzipGTree}_{\alpha, n} \, t)) $ \\

It turns out that the use of $\int_{\emptyset, \mathit{GTree}\,\alpha\,n}$ is not well-typed in this 
version (specifically because it requires $n$ to be in both $\Phi$ and $\ol\alpha$ in the context of $F$
in the $\Lan$-introduction rule), and so we must define $\mathit{alphaRename}$.


\[\begin{array}{l}
  % constructors for GTree
  \mathit{gleaf} : \Nat^{\alpha}\, (\alpha) \, (\mathit{GTree}\,\alpha\,Z) \\
  \mathit{gleaf} = L_\alpha \, t. (\tin_{\mathit{GTree}})_{\alpha,Z} \, (\inl \, (\int_{(\gamma, Z), \gamma} \, t)) \\ \\ 

  \mathit{gnode} : \Nat^{\alpha, n}\, (\mathit{GTree}\,\alpha\,n \times \mathit{GTree}\,\alpha\,n) \, (\mathit{GTree}\,\alpha\, (S\,n)) \\
  \mathit{gnode} = L_{\alpha, n} \, t. (\tin_{\mathit{GTree}})_{\alpha,(S\,n)} \, (\inr \, (\int_{(\gamma_1, S\,\gamma_2), \mathit{GTree}\,\gamma_1\,\gamma_2\times\mathit{GTree}\,\gamma_1\,\gamma_2} \, t)) \\ \\ 

  \mathit{ptreeToGTree} : \Nat^\alpha \, (\mathit{PTree}\,\alpha)\,(\Lan^n_\emptyset \, \mathit{GTree}\,\alpha\,n) \\
  \vspace{0.05in}
  \mathit{ptreeToGTree} = (\fold_{\beta + \phi\,(\beta \times \beta)})_\emptyset \, \mathit{alg} \\ \\ 
  \mathit{alg} : \Nat^\alpha (\alpha + (\Lan^n_\emptyset \, \mathit{GTree}\,(\alpha \times \alpha) \,n)) 
    \; (\Lan^n_\emptyset \,\mathit{GTree}\,\alpha\,n) \\
  \mathit{alg} = L_\alpha \,s. \mathsf{case} \;s \; \mathsf{of} \; \{ \\
    \hspace{0.5in} \mathit{lf} : \alpha \mapsto \int_{\emptyset, \mathit{GTree}\,\alpha\,n} (\mathit{gleaf}_\alpha \mathit{lf})  ; \\
    \hspace{0.5in} \mathit{nd}
    : (\Lan^n_\emptyset \,\mathit{GTree}\,(\alpha \times \alpha)\,n) 
    \mapsto \mathit{alphaRename}_{\mathit{GTree}\,\alpha\,\beta}\, ( \partial^{(\Lan^m_\emptyset \,\mathit{GTree}\,\alpha\,m),\emptyset}_{\mathit{GTree}\,(\alpha \times \alpha)\,n} \mathit{helper} \,t ) \\
    \}\\ \\

  \mathit{helper} : \Nat^{\alpha,n}\,
  (\mathit{GTree}\,(\alpha\times\alpha)\,n) \;(\Lan^m_\emptyset\,\mathit{GTree}\,\alpha\,m) \\
  \mathit{helper} = L_{\alpha, n} \, t. \, \int_{\emptyset, \mathit{GTree}\,\alpha\,m}  \,  (\mathit{gnode}_{\alpha, n}  \,  (\mathit{unzipGTree}_{\alpha, n} \, t)) \\ \\

  \mathit{unzipGTree} : \Nat^{\alpha, n} \, (\mathit{GTree}\,(\alpha \times \alpha)\,n) \, (\mathit{GTree}\,\alpha\,n \times \mathit{GTree}\,\alpha\,n) \\
  \vspace{0.05in}
  \mathit{unzipGTree} = L_{\alpha, n}\, t. (     
    ((\map^{\alpha \times \alpha, \alpha}_{\mathit{GTree}\,\beta\,n})_\emptyset \, (L_{\alpha, n} x. \pi_1 x))_{\alpha, n} \, t ,\\
    \hspace{1.3in}       ((\map^{\alpha \times \alpha, \alpha}_{\mathit{GTree}\,\beta\,n})_\emptyset \, (L_{\alpha, n} x. \pi_2 x))_{\alpha, n} \, t)\\ \\
   


  \mathit{existsGTree} : \Nat^{\alpha, n} (\mathit{GTree}\,\alpha\,n) \; (\Lan^m_\emptyset \,\mathit{GTree}\,\alpha\,m) \\
    \vspace{0.15in}
    % F = GTree alpha n 
    % t : (GTree alpha n)[n := S n] -- ! cant do this because \Phi and \ol\alpha 
    % must be disjoint for F in \int rule 
    % -- but we can let Phi := alpha, n ; \ol\alpha := m ; A := S n
    % -- F := \alpha, n, m \vdash GTree \alpha m 
    % -- F[\alpha := A] := GTree \alpha (S n)
    \mathit{existsGTree} = L_{\alpha, n} \, t. \int_{\emptyset,\mathit{GTree}\,\alpha\,m} t \\ 
    \vspace{0.05in}
    % we cant replace along m, n here can we ? yes we can , e.g. 
    % dfn 3 
    % "H" - phi \vdash (Lan^m phi m ) 
    % "F" - alpha, n \vdash Vec alpha n 
    % --------
    % \alpha \vdash (Lan^m phi m)[phi :=_n Vec alpha n]
    % =
    % \alpha \vdash (Lan^m Vec alpha m)
    %
    %
  \mathit{alphaRename} : \Nat^{\phi} (\Lan^m_\emptyset \,\phi\,m) \; (\Lan^n_\emptyset \,\phi\,n)\\
    \vspace{0.05in}
    \mathit{alphaRename} = L_{\phi}\, t.  \partial^{(\Lan^n_\emptyset \phi\,n), \emptyset}_{(\phi\,m)} \mathit{exists}  \;t \\
    \\
    \mathit{exists} : \Nat^{\phi, m} (\phi\,m) \; (\Lan^n_\emptyset \, \phi \,n) \\
    \mathit{exists} = L_{\phi, m}\, t . \int_{\emptyset, \phi\,n} t \\ \\


\end{array}\]


{\color{red} Let $\emptyset; \alpha \vdash G = (\Lan^n_\emptyset\,\mathit{GTree}\,\alpha\,n)$.
Is this allowed?
}

\[\begin{array}{c}
  \vspace{0.05in}
  \mbox{\textbf{Typing of $\Lan$-elimination in \textit{nd} case of \textit{alg} } }\\
\AXC{$\emptyset;\emptyset~|~\Delta \vdash \eta : \Nat^{\alpha,n}\,
  (\mathit{GTree}\,(\alpha\times\alpha)\,n) \;(\Lan^m_\emptyset\,\mathit{GTree}\,\alpha\,m)
  $}
  \AXC{$\emptyset;\alpha~|~\Delta \vdash t : (\Lan^n_\emptyset \,\mathit{GTree}\,(\alpha \times \alpha) \,n) $}
  \BIC{$\emptyset;\alpha~|~\Delta \vdash \partial^{(\Lan^m_\emptyset \,\mathit{GTree}\,\alpha\,m),\emptyset}_{\mathit{GTree}\,(\alpha \times \alpha)\,n} \eta\,t :
  (\Lan^m_\emptyset \,\mathit{GTree}\,\alpha\,m)  
  $}
\DisplayProof \\ \\

  \vspace{0.05in}
  \mbox{\textbf{Typing of $\Lan$-elimination in \textit{alphaRename} } }\\
  % typing of alpha conversion for Lan
  \AXC{$\emptyset;\emptyset~|~t\vdash \eta : \Nat^{\phi, m}\,
    (\phi\,m)\; (\Lan^n_\emptyset \,\phi\,n)$}
  \AXC{$\emptyset;\phi~|~ t \vdash t : (\Lan^{m}_{\emptyset} (\phi\,m))$}
  \BIC{$\emptyset;\phi~|~ t \vdash \partial^{(\Lan^n_\emptyset \phi\,n), \emptyset}_{(\phi\,m)} \eta\,t :
   (\Lan^n_\emptyset \,\phi\,n)$}
  \DP \\ \\

  \vspace{0.05in}
  \mbox{\textbf{Typing of $\Lan$-introduction in \textit{exists} } }\\
  % typing of alpha conversion helper function for Lan (exists function)
  \AXC{$\emptyset; \phi, m, n \vdash \phi \,n$}
  \AXC{$\emptyset; \phi, m \vdash m $}
  \AXC{$\emptyset; \phi, m \,|\, t \vdash t : \phi n [n := m]$}
  \TIC{$\emptyset; \phi, m \,|\, t \vdash \int_{\emptyset, \phi \,n} t : (\Lan^n_\emptyset \phi \,n)$}
  \DP
\end{array}\]


\subsection{Varying $K$ in Existential Types}
We can also represent existential types with $(\Lan^{\ol\alpha}_\onet \, F)\onet$, or
by using any other constant functor as a subscript. This is because using
a constant functor in the subscript restricts the introduction rule 
so that the instantiating types $\ol{A}$ are hidden in the resulting $\Lan$-type. 

\[\begin{array}{c}
  \mbox{\textbf{$\Lan$-introduction rule with constant subscript} }\\
\AXC{$\Gamma; \Phi,\ol{\alpha}\vdash F$}
\AXC{$\ol{\Gamma; \ol{\alpha} \vdash K}$}
\AXC{$\ol{\Gamma;\Phi \vdash A} \hspace*{0.3in} \Gamma;\Phi~|~\Delta \vdash t : F[\ol{\alpha :=A}]$}
\TIC{$\Gamma;\Phi~|~\Delta \vdash \int_{\ol K,F} t :
  (\Lan^{\ol\alpha}_{\ol K}\,  F)\,\ol{K}$}  
\DisplayProof
\end{array}\]

Varying $K$ changes the elimination rule as well. In particular,
the types $\ol{B}$ must be $\ol{K}$,  

\[\begin{array}{c}
  \mbox{\textbf{$\Lan$-elimination rule with constant subscript} }\\
\AXC{$\Gamma;\emptyset~|~\Delta \vdash \eta : \Nat^{\Phi,\ol\alpha}\,
  F\;G[\ol{\beta := K}]$}
\AXC{$\Gamma;\Phi~|~\Delta \vdash t : (\Lan^{\ol\alpha}_{\ol K} F)\,\ol K $}
\BIC{$\Gamma;\Phi~|~\Delta \vdash \partial^{G, \ol K}_F \eta\,t :
 G[\ol{\beta := K}]$}
\DisplayProof 
\end{array}\]

The most general existential type is $(\Lan^{\ol\alpha}_\emptyset \,F)$,
or equivalently $(\Lan^{\ol\alpha}_\onet\, F)\onet$, but if we 
change the subscript to $\nat$, for example, then we are somewhat
restricted in what we can compute with this type due to the 
elimination rule. 

% % \subsection{Computing with Existential Types}
% % {\color{blue} ($\Lan^\alpha_{\nat}\,\mathit{List}\,\alpha$) example}


\section{Equality Types}
A common use of GADTs is to define a datatype representing 
type equality: 

\[\begin{array}{l}
  \mathtt{data\; Eql \;(A : Set)\;(B : Set)\;:\;Set\;where}\\
\hspace*{0.4in}\mathtt{refl\;:\; \;Eql \;A\;A}
\end{array}\]
We can define this in our calculus as 
  $\mathit{Eql} \, \beta_1 \, \beta_2 = (\Lan^\gamma_{\gamma, \gamma} 1)(\beta_1, \beta_2)$.
Looking at the $\Lan$-introduction rule tells us that we can only produce a term of 
this type when $\beta_1$ and $\beta_2$ are the same type: 


\[\begin{array}{c}
\AXC{$\Gamma; \Phi, \gamma\vdash \onet$}
\AXC{$\ol{\Gamma; \gamma \vdash \gamma}$}
  \AXC{$\Gamma;\Phi \vdash A $}
  \AXC{$\vdash t : \onet$}
\QIC{$\Gamma;\Phi~|~\emptyset \vdash \int_{\ol K,F} t :
  (\Lan^{\gamma}_{\gamma, \gamma}\,  \onet)\,(A, A)$}
\DisplayProof
\end{array}\]

We can use the elimination rule to show that this equality
is symmetric, but so far, I have not found many useful applications
of this type. It seems these types are mostly useful in practical settings
with type refinement/unification.

\[\begin{array}{c}
  \mbox{\textbf{Generic $\Lan$-elimination rule for Eql } }\\
\AXC{$\Gamma;\emptyset~|~\Delta \vdash \eta : \Nat^{\Phi,\gamma}\,
  \onet\;G[\beta_1 := \gamma][\beta_2 := \gamma]$}
\AXC{$\Gamma;\Phi \vdash A$}
\AXC{$\Gamma;\Phi \vdash B$}
  \AXC{$\Gamma;\Phi~|~\Delta \vdash t : (\Lan^{\gamma}_{\gamma, \gamma} \onet)\,(A, B) $}
\QIC{$\Gamma;\Phi~|~\Delta \vdash \partial^{G, \ol K}_F \eta\,t :
  G[\beta_1 := A][\beta_2 := B] $}
\DisplayProof \\ \\


\mbox{\textbf{$\Lan$-elimination rule for Eql with G = ($\mathit{Eql} \, \beta_2 \, \beta_1$)} }\\
\AXC{$\Gamma;\emptyset~|~\Delta \vdash \eta : \Nat^{\Phi,\gamma}\,
  \onet\;(\mathit{Eql} \, \gamma \, \gamma)$}
  
  % G[\beta_1 := \gamma][\beta_2 := \gamma]$}
\AXC{$\Gamma;\Phi \vdash A$}
\AXC{$\Gamma;\Phi \vdash B$}
  \AXC{$\Gamma;\Phi~|~\Delta \vdash t : \mathit{Eql}\,A\,B
  % : (\Lan^{\gamma}_{\gamma, \gamma} \onet)\,(A, B)
  $}
\QIC{$\Gamma;\Phi~|~\Delta \vdash \partial^{G, \ol K}_F \eta\,t :
  \mathit{Eql}\,B\,A
  % G[\beta_1 := A][\beta_2 := B] 
  $}
\DisplayProof \\
\end{array}\]


We can also use the $\mathit{Eql}$ type to define GADTs, e.g., an analogue of
$\mathit{GTree}$:

\[\begin{array}{l}
  \mathtt{data\; ETree\;(A : Set)\;(N : Set)\;:\;Set\;where}\\
  \hspace*{0.4in}\mathtt{eleaf\;:\; A \rightarrow Eql\;N\;Z \rightarrow ETree\;A\;N}\\
  \hspace*{0.4in}\mathtt{enode\;:\; Eql\;N\;(S\;M) \rightarrow ETree\;A\;M \rightarrow ETree\;A\;M 
        \rightarrow ETree\;A\;N} \\

  \mathit{ETree} \, \beta_1 \, \beta_2 = \\
     (\mu \phi. \lambda \alpha, n. 
      \alpha \times (\Lan^\gamma_{\gamma, \gamma} \, \onet)(n, Z)
    + (\Lan^{\alpha', n', m}_{\alpha', n'} \,(\Lan^\gamma_{\gamma, \gamma}\,\onet)(n', S\, m) \times \phi\,(\alpha',m) \times \phi\,(\alpha',m))(\alpha, n) 
    )(\beta_1, \beta_2) \\
\end{array}\]

\noindent
I have not yet explored $\mathit{ETree}$ or similar types thoroughly. 

% % \section{Other}
% % \subsection{Case Elimination}
% % {\color{red} head: Vec a Sn -> a example}
% % 
% % \subsection{Properties of Lan}
% % 
% % {\color{red} What is allowed for the context of G in $\partial$ rule??}
% % 
% % 
% % \subsection{Balanced tree variants}

\end{document}
