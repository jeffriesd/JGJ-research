% For double-blind review submission, w/o CCS and ACM Reference (max submission space)
\documentclass[acmsmall,review,anonymous]{acmart}
\settopmatter{printfolios=true,printccs=false,printacmref=false}
%% For double-blind review submission, w/ CCS and ACM Reference
%\documentclass[acmsmall,review,anonymous]{acmart}\settopmatter{printfolios=true}
%% For single-blind review submission, w/o CCS and ACM Reference (max submission space)
%\documentclass[acmsmall,review]{acmart}\settopmatter{printfolios=true,printccs=false,printacmref=false}
%% For single-blind review submission, w/ CCS and ACM Reference
%\documentclass[acmsmall,review]{acmart}\settopmatter{printfolios=true}
%% For final camera-ready submission, w/ required CCS and ACM Reference
%\documentclass[acmsmall]{acmart}\settopmatter{}


%% Journal information
%% Supplied to authors by publisher for camera-ready submission;
%% use defaults for review submission.
\acmJournal{PACMPL}
\acmVolume{1}
\acmNumber{POPL} % CONF = POPL or ICFP or OOPSLA
\acmArticle{1}
\acmYear{2020}
\acmMonth{1}
\acmDOI{} % \acmDOI{10.1145/nnnnnnn.nnnnnnn}
\startPage{1}

%% Copyright information
%% Supplied to authors (based on authors' rights management selection;
%% see authors.acm.org) by publisher for camera-ready submission;
%% use 'none' for review submission.
\setcopyright{none}
%\setcopyright{acmcopyright}
%\setcopyright{acmlicensed}
%\setcopyright{rightsretained}
%\copyrightyear{2018}           %% If different from \acmYear

%% Bibliography style
\bibliographystyle{ACM-Reference-Format}
%% Citation style
%% Note: author/year citations are required for papers published as an
%% issue of PACMPL.
\citestyle{acmauthoryear}   %% For author/year citations
%\citestyle{acmnumeric}

%%%%%%%%%%%%%%%%%%%%%%%%%%%%%%%%%%%%%%%%%%%%%%%%%%%%%%%%%%%%%%%%%%%%%%
%% Note: Authors migrating a paper from PACMPL format to traditional
%% SIGPLAN proceedings format must update the '\documentclass' and
%% topmatter commands above; see 'acmart-sigplanproc-template.tex'.
%%%%%%%%%%%%%%%%%%%%%%%%%%%%%%%%%%%%%%%%%%%%%%%%%%%%%%%%%%%%%%%%%%%%%%



\usepackage[utf8]{inputenc}
\usepackage{ccicons}

\usepackage{amsmath}
\usepackage{amsthm}
\usepackage{amscd}
%\usepackage{MnSymbol}
\usepackage{xcolor}

\usepackage{bbold}
\usepackage{url}
\usepackage{upgreek}
%\usepackage{stmaryrd}

\usepackage{lipsum}
\usepackage{tikz-cd}
\usetikzlibrary{cd}
\usetikzlibrary{calc}
\usetikzlibrary{arrows}

\usepackage{bussproofs}
\EnableBpAbbreviations

\DeclareMathAlphabet{\mathpzc}{OT1}{pzc}{m}{it}

%\usepackage[amsmath]{ntheorem}

\newcommand{\lam}{\lambda}
\newcommand{\eps}{\varepsilon}
\newcommand{\ups}{\upsilon}
\newcommand{\mcB}{\mathcal{B}}
\newcommand{\mcD}{\mathcal{D}}
\newcommand{\mcE}{\mathcal{E}}
\newcommand{\mcF}{\mathcal{F}}
\newcommand{\mcP}{\mathcal{P}}
\newcommand{\mcI}{\mathcal{I}}
\newcommand{\mcJ}{\mathcal{J}}
\newcommand{\mcK}{\mathcal{K}}
\newcommand{\mcL}{\mathcal{L}}
\newcommand{\WW}{\mathcal{W}}

\newcommand{\ex}{\mcE_x}
\newcommand{\ey}{\mcE_y}
\newcommand{\bzero}{\boldsymbol{0}}
\newcommand{\bone}{{\boldsymbol{1}}}
\newcommand{\tB}{{\bone_\mcB}}
\newcommand{\tE}{{\bone_\mcE}}
\newcommand{\bt}{\mathbf{t}}
\newcommand{\bp}{\mathbf{p}}
\newcommand{\bsig}{\mathbf{\Sigma}}
\newcommand{\bpi}{\boldsymbol{\pi}}
\newcommand{\Empty}{\mathtt{Empty}}
\newcommand{\truthf}{\mathtt{t}}
\newcommand{\id}{id}
\newcommand{\coo}{\mathtt{coo\ }}
\newcommand{\mcC}{\mathcal{C}}
\newcommand{\Rec}{\mathpzc{Rec}}
\newcommand{\types}{\mathcal{T}}



%\newcommand{\semof}[1]{\llbracket{#1}\rrbracket^\rel}
\newcommand{\sem}[1]{\llbracket{#1}\rrbracket}
\newcommand{\setsem}[1]{\llbracket{#1}\rrbracket^\set}
\newcommand{\relsem}[1]{\llbracket{#1}\rrbracket^\rel}
\newcommand{\dsem}[1]{\llbracket{#1}\rrbracket^{\mathsf D}}
\newcommand{\setenv}{\mathsf{SetEnv}}
\newcommand{\relenv}{\mathsf{RelEnv}}
\newcommand{\denv}{\mathsf{DEnv}}

\newcommand{\rel}{\mathsf{Rel}}
\newcommand{\setof}[1]{\{#1\}}
\newcommand{\letin}[1]{\texttt{let }#1\texttt{ in }}
\newcommand{\comp}[1]{{\{#1\}}}
\newcommand{\bcomp}[1]{\{\![#1]\!\}}
\newcommand{\beps}{\boldsymbol{\eps}}
%\newcommand{\B}{\mcB}
%\newcommand{\Bo}{{|\mcB|}}

\newcommand{\lmt}{\longmapsto}
\newcommand{\RA}{\Rightarrow}
\newcommand{\LA}{\Leftarrow}
\newcommand{\rras}{\rightrightarrows}
\newcommand{\colim}[2]{{{\underrightarrow{\lim}}_{#1}{#2}}}
\newcommand{\lift}[1]{{#1}\,{\hat{} \; \hat{}}}
\newcommand{\graph}[1]{\langle {#1} \rangle}

\newcommand{\carAT}{\mathsf{car}({\mathcal A}^T)}
\newcommand{\isoAto}{\mathsf{Iso}({\mcA^\to})}
\newcommand{\falg}{\mathsf{Alg}_F}
\newcommand{\CC}{\mathsf{Pres}(\mathcal{A})}
\newcommand{\PP}{\mathcal{P}}
\newcommand{\DD}{D_{(A,B,f)}}
\newcommand{\from}{\leftarrow}
\newcommand{\upset}[1]{{#1}{\uparrow}}
\newcommand{\smupset}[1]{{#1}\!\uparrow}

\newcommand{\Coo}{\mathpzc{Coo}}
\newcommand{\code}{\#}
\newcommand{\nat}{\mathpzc{Nat}}

\newcommand{\eq}{\; = \;}
\newcommand{\of}{\; : \;}
\newcommand{\df}{\; := \;}
\newcommand{\bnf}{\; ::= \;}

\newcommand{\zmap}[1]{{\!{\between\!\!}_{#1}\!}}
\newcommand{\bSet}{\mathbf{Set}}

\newcommand{\dom}{\mathsf{dom}}
\newcommand{\cod}{\mathsf{cod}}
\newcommand{\adjoint}[2]{\mathrel{\mathop{\leftrightarrows}^{#1}_{#2}}}
\newcommand{\isofunc}{\mathpzc{Iso}}
\newcommand{\ebang}{{\eta_!}}
\newcommand{\lras}{\leftrightarrows}
\newcommand{\rlas}{\rightleftarrows}
\newcommand{\then}{\quad\Longrightarrow\quad}
\newcommand{\hookup}{\hookrightarrow}

\newcommand{\spanme}[5]{\begin{CD} #1 @<#2<< #3 @>#4>> #5 \end{CD}}
\newcommand{\spanm}[3]{\begin{CD} #1 @>#2>> #3\end{CD}}
\newcommand{\pushout}{\textsf{Pushout}}
\newcommand{\mospace}{\qquad\qquad\!\!\!\!}

\newcommand{\natur}[2]{#1 \propto #2}

\newcommand{\Tree}{\mathsf{Tree}\,}
\newcommand{\GRose}{\mathsf{GRose}\,}
\newcommand{\List}{\mathsf{List}\,}
\newcommand{\PTree}{\mathsf{PTree}\,}
\newcommand{\Bush}{\mathsf{Bush}\,}
\newcommand{\Forest}{\mathsf{Forest}\,}
\newcommand{\Lam}{\mathsf{Lam}\,}
\newcommand{\LamES}{\mathsf{Lam}^+}
\newcommand{\Expr}{\mathsf{Expr}\,}

\newcommand{\ListNil}{\mathsf{Nil}}
\newcommand{\ListCons}{\mathsf{Cons}}
\newcommand{\LamVar}{\mathsf{Var}}
\newcommand{\LamApp}{\mathsf{App}}
\newcommand{\LamAbs}{\mathsf{Abs}}
\newcommand{\LamSub}{\mathsf{Sub}}
\newcommand{\ExprConst}{\mathsf{Const}}
\newcommand{\ExprPair}{\mathsf{Pair}}
\newcommand{\ExprProj}{\mathsf{Proj}}
\newcommand{\ExprAbs}{\mathsf{Abs}}
\newcommand{\ExprApp}{\mathsf{App}}
\newcommand{\Ptree}{\mathsf{Ptree}}

\newcommand{\kinds}{\mathpzc{K}}
\newcommand{\tvars}{\mathbb{T}}
\newcommand{\fvars}{\mathbb{F}}
\newcommand{\consts}{\mathpzc{C}}
\newcommand{\Lan}{\mathsf{Lan}}
\newcommand{\zerot}{\mathbb{0}}
\newcommand{\onet}{\mathbb{1}}
\newcommand{\bool}{\mathbb{2}}
\renewcommand{\nat}{\mathbb{N}}
%\newcommand{\semof}[1]{[\![#1]\!]}
%\newcommand{\setsem}[1]{\llbracket{#1}\rrbracket^\set}
\newcommand{\predsem}[1]{\llbracket{#1}\rrbracket^\pred}
%\newcommand{\todot}{\stackrel{.}{\to}}
\newcommand{\todot}{\Rightarrow}
\newcommand{\bphi}{{\bm \phi}}

\newcommand{\bm}[1]{\boldsymbol{#1}}

\newcommand{\cL}{\mathcal{L}}
\newcommand{\T}{\mathcal{T}}
\newcommand{\Pos}{P\!}
%\newcommand{\Pos}{\mathcal{P}\!}
\newcommand{\Neg}{\mathcal{N}}
\newcommand{\Hf}{\mathcal{H}}
\newcommand{\V}{\mathbb{V}}
\newcommand{\I}{\mathcal{I}}
\newcommand{\Set}{\mathsf{Set}}
\newcommand{\Nat}{\mathsf{Nat}}
\newcommand{\Homrel}{\mathsf{Hom_{Rel}}}
\newcommand{\CV}{\mathcal{CV}}
\newcommand{\lan}{\mathsf{Lan}}
\newcommand{\Id}{\mathit{Id}}
\newcommand{\mcA}{\mathcal{A}}
\newcommand{\inl}{\mathsf{inl}}
\newcommand{\inr}{\mathsf{inr}}
\newcommand{\case}[3]{\mathsf{case}\,{#1}\,\mathsf{of}\,\{{#2};\,{#3}\}}
\newcommand{\tin}{\mathsf{in}}
\newcommand{\fold}{\mathsf{fold}}
\newcommand{\Eq}{\mathsf{Eq}}
\newcommand{\Hom}{\mathsf{Hom}}
\newcommand{\curry}{\mathsf{curry}}
\newcommand{\uncurry}{\mathsf{uncurry}}
\newcommand{\eval}{\mathsf{eval}}
\newcommand{\apply}{\mathsf{apply}}

\newcommand{\ar}[1]{\##1}
\newcommand{\mcG}{\mathcal{G}}
\newcommand{\mcH}{\mathcal{H}}
\newcommand{\TV}{\mathpzc{V}}

\newcommand{\essim}[1]{\mathsf{EssIm}(#1)}
\newcommand{\hra}{\hookrightarrow}

\newcommand{\ol}[1]{\overline{#1}}
\newcommand{\ul}[1]{\underline{#1}}
\newcommand{\op}{\mathsf{op}}
\newcommand{\trige}{\trianglerighteq}
\newcommand{\trile}{\trianglelefteq}
\newcommand{\LFP}{\mathsf{LFP}}
\newcommand{\LAN}{\mathsf{Lan}}
%\newcommand{\Mu}{{\mu\hskip-4pt\mu}}
\newcommand{\Mu}{{\mu\hskip-5.5pt\mu}}
%\newcommand{\Mu}{\boldsymbol{\upmu}}
\newcommand{\Terms}{\mathpzc{Terms}}
\newcommand{\Ord}{\mathpzc{Ord}}
\newcommand{\Anote}[1]{{\color{blue} {#1}}}
\newcommand{\Pnote}[1]{{\color{red} {#1}}}

\newcommand{\greyout}[1]{{\color{gray} {#1}}}
\newcommand{\ora}[1]{\overrightarrow{#1}}

%\newcommand{\?}{{.\ }}
%\theoremheaderfont{\scshape}
%\theorembodyfont{\normalfont}
%\theoremseparator{.\ \ }
\newtheorem{thm}{Theorem}
\newtheorem{dfn}[thm]{Definition}
\newtheorem{prop}[thm]{Proposition}
\newtheorem{cor}[thm]{Corollary}
\newtheorem{lemma}[thm]{Lemma}
\newtheorem{rmk}[thm]{Remark}
\newtheorem{expl}[thm]{Example}
\newtheorem{notn}[thm]{Notation}
%\theoremstyle{nonumberplain}
%\theoremsymbol{\Box}


\theoremstyle{definition}
\newtheorem{exmpl}{Example}

\renewcommand{\greyout}[1]{} %{{\color{gray} {#1}}} -- toggle to remove greyed text

\newcommand{\emptyfun}{{[]}}
\newcommand{\cal}{\mathcal}
%\newcommand{\fold}{\mathit{fold}}
\newcommand{\F}{\mathcal{F}}
\renewcommand{\G}{\mathcal{G}}
\newcommand{\N}{\mathcal{N}}
\newcommand{\E}{\mathcal{E}}
\newcommand{\B}{\mathcal{B}}
\renewcommand{\P}{\mathcal{A}}
\newcommand{\pred}{\mathsf{Fam}}
\newcommand{\env}{\mathsf{Env}}
\newcommand{\set}{\mathsf{Set}}
\renewcommand{\S}{\mathcal S}
\renewcommand{\C}{\mathcal{C}}
\newcommand{\D}{\mathcal{D}}
\newcommand{\A}{\mathcal{A}}
\renewcommand{\id}{\mathit{id}}
\newcommand{\map}{\mathsf{map}}
\newcommand{\pid}{\underline{\mathit{id}}}
\newcommand{\pcirc}{\,\underline{\circ}\,}
\newcommand{\pzero}{\underline{0}}
\newcommand{\pone}{\underline{1}}
\newcommand{\psum}{\,\underline{+}\,}
%\newcommand{\inl}{\mathit{inL}\,}
%\newcommand{\inr}{\mathit{inR}\,}
\newcommand{\pinl}{\underline{\mathit{inL}}\,}
\newcommand{\pinr}{\underline{\mathit{inR}}\,}
\newcommand{\ptimes}{\,\underline{\times}\,}
\newcommand{\ppi}{\underline{\pi_1}}
\newcommand{\pppi}{\underline{\pi_2}}
\newcommand{\pmu}{\underline{\mu}}

\title[Free Theorems for Nested Types]{Free Theorems for 
Nested Types} %% [Short Title] is optional;
                                        %% when present, will be used in
                                        %% header instead of Full Title.
%\titlenote{with title note}             %% \titlenote is optional;
                                        %% can be repeated if necessary;
                                        %% contents suppressed with 'anonymous'
%\subtitle{Subtitle}                     %% \subtitle is optional
%\subtitlenote{with subtitle note}       %% \subtitlenote is optional;
                                        %% can be repeated if necessary;
                                        %% contents suppressed with 'anonymous'


%% Author information
%% Contents and number of authors suppressed with 'anonymous'.
%% Each author should be introduced by \author, followed by
%% \authornote (optional), \orcid (optional), \affiliation, and
%% \email.
%% An author may have multiple affiliations and/or emails; repeat the
%% appropriate command.
%% Many elements are not rendered, but should be provided for metadata
%% extraction tools.

%% Author with single affiliation.
\author{Patricia Johann and Andrew Polonsky}
%\authornote{with author1 note}          %% \authornote is optional;
%                                        %% can be repeated if necessary
%\orcid{nnnn-nnnn-nnnn-nnnn}             %% \orcid is optional
\affiliation{
%  \position{Position1}
%  \department{Department1}              %% \department is recommended
  \institution{Appalachian State University}            %% \institution is required
%  \streetaddress{Street1 Address1}
%  \city{City1}
%  \state{State1}
%  \postcode{Post-Code1}
%  \country{Country1}                    %% \country is recommended
}
\email{johannp@appstate.edu, andrew.polonsky@gmail.com}          %% \email is recommended


\begin{document}

\begin{abstract}
\end{abstract}

%\begin{CCSXML}
%<ccs2012>
%<concept>
%<concept_id>10011007.10011006.10011008</concept_id>
%<concept_desc>Software and its engineering~General programming languages</concept_desc>
%<concept_significance>500</concept_significance>
%</concept>
%<concept>
%<concept_id>10003456.10003457.10003521.10003525</concept_id>
%<concept_desc>Social and professional topics~History of programming languages</concept_desc>
%<concept_significance>300</concept_significance>
%</concept>
%</ccs2012>
%\end{CCSXML}
%
%\ccsdesc[500]{Software and its engineering~General programming languages}
%\ccsdesc[300]{Social and professional topics~History of programming languages}
%% End of generated code


%% Keywords
%% comma separated list
%\keywords{keyword1, keyword2, keyword3}  %% \keywords is optional


\maketitle

\section{Map}

\begin{dfn}
Two terms $\Gamma; \Phi \,|\, \Delta \vdash t : \tau$ and $\Gamma; \Phi \,|\, \Delta \vdash t' : \tau$
are semantically equivalent if they have the same set interpretation functor and relational interpretation functor,
i.e., for every set environment $\rho$, we have that
\[
 \setsem{\Gamma; \Phi \,|\, \Delta \vdash t : \tau}\rho
 = \setsem{\Gamma; \Phi \,|\, \Delta \vdash t' : \tau}\rho
\]
and, for every relation environment $\rho$, we have that
\[
 \relsem{\Gamma; \Phi \,|\, \Delta \vdash t : \tau}\rho
 = \relsem{\Gamma; \Phi \,|\, \Delta \vdash t' : \tau}\rho
\]
and, moreover, the interpretations coincide on  morphisms of environments as well.
\end{dfn}

When proving two terms semantically equivalent, we shall generally only check that their interpretation functors coincide on objects,
i.e., on set or relation environments.

We define a composition operation between terms of type $\Nat$
and an identity term for functorial types,
as convenient shorthands.

\begin{dfn}
Let $\Gamma; \emptyset \,|\, \Delta \vdash t: \Nat^{\overline{\alpha}} F\,G$
and $\Gamma; \emptyset \,|\, \Delta \vdash s: \Nat^{\overline{\alpha}} G\,H$
be terms.
Then the composition $s \circ t$ of $t$ and $s$ is the term
$\Gamma; \emptyset \,|\, \Delta \vdash L_{\overline{\alpha}} x. s_{\overline{\alpha}}(t_{\overline{\alpha}}x): \Nat^{\overline{\alpha}} F\,H$.
\end{dfn}

\begin{lemma}\label{lem:comp}
Let $\Gamma; \emptyset \,|\, \Delta \vdash t: \Nat^{\overline{\alpha}} F\,G$
and $\Gamma; \emptyset \,|\, \Delta \vdash s: \Nat^{\overline{\alpha}} G\,H$
be terms.
Then for any set environment $\rho$, the semantic interpretation of the composition is
\[
\setsem{\Gamma; \emptyset \,|\, \Delta \vdash s \circ t: \Nat^{\overline{\alpha}} F\,H} \rho
= \setsem{\Gamma; \emptyset \,|\, \Delta \vdash s: \Nat^{\overline{\alpha}} G\,H} \rho
    \circ \setsem{\Gamma; \emptyset \,|\, \Delta \vdash t: \Nat^{\overline{\alpha}} F\,G} \rho
\]
\end{lemma}
\begin{proof}
For any set environment $\rho$ and $d : \setsem{\Gamma; \emptyset \vdash \Delta}\rho$, we have that
\[
\begin{array}{rl}
  & \setsem{\Gamma; \emptyset \,|\, \Delta \vdash s \circ t: \Nat^{\overline{\alpha}} F\,H} \rho d \\
= & \lambda \ol{A}. \lambda x. (\setsem{\Gamma; \emptyset \,|\, \Delta \vdash s \circ t: \Nat^{\overline{\alpha}} F\,H} \rho d)_{\ol{A}} x \\
= & \lambda \ol{A}. \lambda x. (\setsem{\Gamma; \emptyset \,|\, \Delta \vdash L_{\overline{\alpha}} x. s_{\overline{\alpha}}(t_{\overline{\alpha}}x): \Nat^{\overline{\alpha}} F\,H} \rho d)_{\ol{A}} x \\
= & \lambda \ol{A}. \lambda x. \setsem{\Gamma; \ol{\alpha} \,|\, \Delta, x: F \vdash s_{\overline{\alpha}}(t_{\overline{\alpha}}x): H} \rho [\ol{\alpha := A}] d x \\
= & \lambda \ol{A}. \lambda x. (\setsem{\Gamma; \emptyset \,|\, \Delta \vdash s: \Nat^{\overline{\alpha}} G\,H} \rho d)_{\ol{A}}
    (\setsem{\Gamma; \ol{\alpha} \,|\, \Delta, x: F \vdash t_{\overline{\alpha}}x: H} \rho [\ol{\alpha := A}] d x) \\
= & \lambda \ol{A}. \lambda x. (\setsem{\Gamma; \emptyset \,|\, \Delta \vdash s: \Nat^{\overline{\alpha}} G\,H} \rho d)_{\ol{A}}
    ((\setsem{\Gamma; \emptyset \,|\, \Delta \vdash t: \Nat^{\overline{\alpha}} F\,G} \rho d)_{\ol{A}} x) \\
= & \setsem{\Gamma; \emptyset \,|\, \Delta \vdash s: \Nat^{\overline{\alpha}} G\,H} \rho d
    \circ \setsem{\Gamma; \emptyset \,|\, \Delta \vdash t: \Nat^{\overline{\alpha}} F\,G} \rho d \qedhere
\end{array}
\]
\end{proof}

\begin{dfn}
Let $\Gamma; \ol{\alpha} \vdash F$ be a type. Then the identity $\Id_F$ of $F$ is the term
$\Gamma; \emptyset \,|\, \emptyset \vdash L_{\ol{\alpha}} x.x : \Nat^{\ol{\alpha}} F\,F$.
\end{dfn}

\begin{lemma}\label{lem:interpretation-identity}
Let $\Gamma; \ol{\alpha} \vdash F$ be a type.
Then for any set environment $\rho$, the semantic interpretation of the identity is
\[
\setsem{\Gamma; \emptyset \,|\, \emptyset \vdash \Id_{F} : \Nat^{\ol{\alpha}} F\,F} \rho \ast
= \Id_{\lambda \ol{A}. \setsem{\Gamma; \ol{\alpha} \vdash F} \rho [\ol{\alpha := A}]}
\]
\end{lemma}
\begin{proof}
For any set environment $\rho$, we have that
\[
\begin{array}{rl}
  & \setsem{\Gamma; \emptyset \,|\, \emptyset \vdash \Id_{F} : \Nat^{\ol{\alpha}} F\,F} \rho \ast \\
= & \setsem{\Gamma; \emptyset \,|\, \emptyset \vdash L_{\ol{\alpha}} x.x : \Nat^{\ol{\alpha}} F\,F} \rho \ast \\
= & \lambda \ol{A}. \lambda x. \setsem{\Gamma; \ol{\alpha} \,|\, x: F \vdash x : F} \rho [\ol{\alpha := A}] x \\
= & \lambda \ol{A}. \lambda x. x \\
= & \lambda \ol{A}. \Id_{\setsem{\Gamma; \ol{\alpha} \vdash F} \rho [\ol{\alpha := A}]} \\
= & \Id_{\lambda \ol{A}. \setsem{\Gamma; \ol{\alpha} \vdash F} \rho [\ol{\alpha := A}]} \qedhere
\end{array}
\]
\end{proof}

The following result shows that terms of type $\Nat$ behave like actual natural transformations with respect to their source and target functorial types.

\begin{lemma}[Naturality]\label{lem:naturality}
The terms
\[
  \Gamma; \emptyset \,|\, x : \Nat^{\overline{\alpha}, \overline{\gamma}} F\,G, \overline{y : \Nat^{\overline{\gamma}} \sigma\, \tau}
  \vdash ((\map_G^{\overline{\sigma}, \overline{\tau}})_{\emptyset} \overline{y})
  \circ (L_{\overline{\gamma}} z. x_{\overline{\sigma}, \overline{\gamma}} z)
  : \Nat^{\overline{\gamma}} F[\overline{\alpha := \sigma}]\, G[\overline{\alpha := \tau}]
\]
% \begin{prooftree}
%     \AXC{$\Gamma; \overline{\alpha}, \overline{\gamma} \vdash F$}
%     \AXC{$\Gamma; \overline{\alpha}, \overline{\gamma} \vdash G$}
%     \AXC{$\overline{\Gamma; \overline{\gamma} \vdash \sigma}$}
%     \AXC{$\overline{\Gamma; \overline{\gamma} \vdash \tau}$}
%     \QIC{$\Gamma; \emptyset \,|\, \Delta, \eta : \Nat^{\overline{\alpha}, \overline{\gamma}} F G, \overline{f : \Nat^{\overline{\gamma}} \sigma \tau}
%           \vdash ((\map_G^{\overline{\sigma}, \overline{\tau}})_{\emptyset} \overline{f})
%           \circ (L_{\overline{\gamma}} x. \eta_{\overline{\sigma}, \overline{\gamma}} x)
%           : \Nat^{\overline{\gamma}} F[\overline{\alpha := \sigma}] G[\overline{\alpha := \tau}]$}
% \end{prooftree}
and
\[
  \Gamma; \emptyset \,|\, x : \Nat^{\overline{\alpha}, \overline{\gamma}} F\, G, \overline{y : \Nat^{\overline{\gamma}} \sigma\, \tau}
  \vdash (L_{\overline{\gamma}} z. x_{\overline{\tau}, \overline{\gamma}} z)
  \circ ((\map_F^{\overline{\sigma}, \overline{\tau}})_{\emptyset} \overline{y})
  : \Nat^{\overline{\gamma}} F[\overline{\alpha := \sigma}]\, G[\overline{\alpha := \tau}]
\]
% \begin{prooftree}
%     \AXC{$\Gamma; \overline{\alpha}, \overline{\gamma} \vdash F$}
%     \AXC{$\Gamma; \overline{\alpha}, \overline{\gamma} \vdash G$}
%     \AXC{$\overline{\Gamma; \overline{\gamma} \vdash \sigma}$}
%     \AXC{$\overline{\Gamma; \overline{\gamma} \vdash \tau}$}
%     \QIC{$\Gamma; \emptyset \,|\, \Delta, \eta : \Nat^{\overline{\alpha}, \overline{\gamma}} F G, \overline{f : \Nat^{\overline{\gamma}} \sigma \tau}
%           \vdash (L_{\overline{\gamma}} x. \eta_{\overline{\tau}, \overline{\gamma}} x)
%           \circ ((\map_F^{\overline{\sigma}, \overline{\tau}})_{\emptyset} \overline{f})
%           : \Nat^{\overline{\gamma}} F[\overline{\alpha := \sigma}] G[\overline{\alpha := \tau}]$}
% \end{prooftree}
are semantically equivalent.
%: for any set environment $\rho$, we have that
%\[
%\setsem{((\map_G^{\overline{\sigma}, \overline{\tau}})_{\emptyset} \overline{y})
%  \circ (L_{\overline{\gamma}} z. x_{\overline{\sigma}, \overline{\gamma}} z)}\rho
%= \setsem{(L_{\overline{\gamma}} z. x_{\overline{\tau}, \overline{\gamma}} z)
%  \circ ((\map_F^{\overline{\sigma}, \overline{\tau}})_{\emptyset} \overline{y})}\rho
%\]
%and, analogously, for any relation environment $\rho$, we have that
%\[
%\relsem{((\map_G^{\overline{\sigma}, \overline{\tau}})_{\emptyset} \overline{y})
%  \circ (L_{\overline{\gamma}} z. x_{\overline{\sigma}, \overline{\gamma}} z)}\rho
%= \relsem{(L_{\overline{\gamma}} z. x_{\overline{\tau}, \overline{\gamma}} z)
%  \circ ((\map_F^{\overline{\sigma}, \overline{\tau}})_{\emptyset} \overline{y})}\rho
%\]
\end{lemma}
\begin{proof}
% The first term is formed through the following derivation tree.
% \begin{prooftree}
% \AXC{$\Gamma; \overline{\alpha}, \overline{\gamma} \vdash F$}
% \AXC{$\Gamma; \overline{\alpha}, \overline{\gamma} \vdash G$}
% \BIC{$\Gamma; \emptyset \vdash \Nat^{\overline{\alpha}, \overline{\gamma}} F G$}
% \UIC{$\Gamma; \emptyset \,|\, \eta : \Nat^{\overline{\alpha}, \overline{\gamma}} F G
%       \vdash \eta : \Nat^{\overline{\alpha}, \overline{\gamma}} F G$}
% \AXC{$\Gamma; \overline{\alpha}, \overline{\gamma} \vdash F$}
% \AXC{$\overline{\Gamma; \overline{\gamma} \vdash \sigma}$}
% \BIC{$\Gamma; \overline{\gamma} \vdash F[\overline{\alpha := \sigma}]$}
% \BIC{$\Gamma; \overline{\gamma} \,|\, \eta : \Nat^{\overline{\alpha}, \overline{\gamma}} F G, x \colon F[\overline{\alpha := \sigma}]
%       \vdash \eta_{\overline{\sigma}, \overline{\gamma}} x
%       : G[\overline{\alpha := \sigma}]$}
% \AXC{$\Gamma; \overline{\alpha}, \overline{\gamma} \vdash G$}
% \AXC{$\overline{\Gamma; \overline{\gamma} \vdash \sigma}$}
% \AXC{$\overline{\Gamma; \overline{\gamma} \vdash \tau}$}
% \TIC{$\Gamma; \emptyset \,|\, \emptyset \vdash \map_{G}^{\overline{\sigma}, \overline{\tau}}
%       :\Nat^{\emptyset} (\overline{\Nat^{\overline{\gamma}} \sigma \tau}) (\Nat^{\overline{\gamma}} G[\overline{\alpha := \sigma}] G[\overline{\alpha := \tau}])$}
% \UIC{$\Gamma; \emptyset \,|\, \overline{f : \Nat^{\overline{\gamma}} \sigma \tau}
%       \vdash (\map_{G}^{\overline{\sigma}, \overline{\tau}})_{\emptyset} \overline{f}
%       : \Nat^{\overline{\gamma}} G[\overline{\alpha := \sigma}] G[\overline{\alpha := \tau}]$}
% \BIC{$\Gamma; \overline{\gamma} \,|\, \eta : \Nat^{\overline{\alpha}, \overline{\gamma}} F G, x \colon F[\overline{\alpha := \sigma}], \overline{f : \Nat^{\overline{\gamma}} \sigma \tau}
%       \vdash ((\map_{G}^{\overline{\sigma}, \overline{\tau}})_{\emptyset} \overline{f})_{\overline{\gamma}} (\eta_{\overline{\sigma}, \overline{\gamma}} x)
%       : G[\overline{\alpha := \tau}]$}
% \UIC{$\Gamma; \emptyset \,|\, \eta : \Nat^{\overline{\alpha}, \overline{\gamma}} F G, \overline{f : \Nat^{\overline{\gamma}} \sigma \tau}
%       \vdash L_{\overline{\gamma}} x. ((\map_{G}^{\overline{\sigma}, \overline{\tau}})_{\emptyset} \overline{f})_{\overline{\gamma}} (\eta_{\overline{\sigma}, \overline{\gamma}} x)
%       : \Nat^{\overline{\gamma}} F[\overline{\alpha := \sigma}] G[\overline{\alpha := \tau}]$}
% \end{prooftree}
% The second term is formed analogously.
Let $\eta : \setsem{\Gamma; \emptyset \vdash \Nat^{\overline{\alpha}, \overline{\gamma}} F\, G}\rho$
and $\overline{f : \setsem{\Gamma; \emptyset \vdash \Nat^{\overline{\gamma}} \sigma\, \tau}\rho}$.
The semantic interpretation of the first term is
\[
%CAN BE SIMPLIFIED USING LEMMA~\ref{lem:comp}
\begin{array}{cl}
  & \setsem{\Gamma; \emptyset \,|\, x : \Nat^{\overline{\alpha}, \overline{\gamma}} F\, G, \overline{y : \Nat^{\overline{\gamma}} \sigma\, \tau}
      \vdash ((\map_G^{\overline{\sigma}, \overline{\tau}})_{\emptyset} \overline{y}) \circ (L_{\overline{\gamma}} z. x_{\overline{\sigma}, \overline{\gamma}} z)
      : \Nat^{\overline{\gamma}} F[\overline{\alpha := \sigma}]\, G[\overline{\alpha := \tau}]}\rho \eta \overline{f} \\
= & \setsem{\Gamma; \emptyset \,|\, x : \Nat^{\overline{\alpha}, \overline{\gamma}} F\, G, \overline{y : \Nat^{\overline{\gamma}} \sigma\, \tau}
      \vdash L_{\overline{\gamma}} z. ((\map_{G}^{\overline{\sigma}, \overline{\tau}})_{\emptyset} \overline{y})_{\overline{\gamma}} (x_{\overline{\sigma}, \overline{\gamma}} z)
      : \Nat^{\overline{\gamma}} F[\overline{\alpha := \sigma}]\, G[\overline{\alpha := \tau}]}\rho \eta \overline{f} \\
= & \lambda \overline{C}. \lambda z. \setsem{\Gamma; \overline{\gamma}
      \,|\, x : \Nat^{\overline{\alpha}, \overline{\gamma}} F\, G, \overline{y : \Nat^{\overline{\gamma}} \sigma\, \tau}, z \colon F[\overline{\alpha := \sigma}]
      \vdash ((\map_{G}^{\overline{\sigma}, \overline{\tau}})_{\emptyset} \overline{y})_{\overline{\gamma}} (x_{\overline{\sigma}, \overline{\gamma}} z)
      : G[\overline{\alpha := \tau}]}\rho [\overline{\gamma := C}] \eta \overline{f} z \\
= & \lambda \overline{C}. \lambda z. (\setsem{\Gamma; \emptyset \,|\, \overline{y : \Nat^{\overline{\gamma}} \sigma\, \tau}
      \vdash (\map_{G}^{\overline{\sigma}, \overline{\tau}})_{\emptyset} \overline{y}
      : \Nat^{\overline{\gamma}} G[\overline{\alpha := \sigma}]\, G[\overline{\alpha := \tau}]}\rho \overline{f})_{\overline{C}} \\
  &   (\setsem{\Gamma; \overline{\gamma} \,|\, x : \Nat^{\overline{\alpha}, \overline{\gamma}} F\, G, z \colon F[\overline{\alpha := \sigma}]
      \vdash x_{\overline{\sigma}, \overline{\gamma}} z
      : G[\overline{\alpha := \sigma}]} \rho [\overline{\gamma := C}] \eta z) \\
= & \lambda \overline{C}. \lambda z. (
        \setsem{\Gamma; \emptyset \,|\, \emptyset \vdash \map_{G}^{\overline{\sigma}, \overline{\tau}}
        :\Nat^{\emptyset} (\overline{\Nat^{\overline{\gamma}} \sigma\, \tau})\, (\Nat^{\overline{\gamma}} G[\overline{\alpha := \sigma}]\, G[\overline{\alpha := \tau}])}
        \rho \ast \overline{f}
      )_{\overline{C}} \\
  & (
      \setsem{\Gamma; \emptyset \,|\, x : \Nat^{\overline{\alpha}, \overline{\gamma}} F\, G \vdash x}\rho \eta
    )_{\overline{\setsem{\Gamma; \overline{\gamma} \vdash \sigma}\rho [\overline{\gamma := C}]}, \overline{\setsem{\Gamma; \overline{\gamma} \vdash \gamma}\rho [\overline{\gamma := C}]}}
      \setsem{\Gamma; \overline{\gamma} \,|\, z: F[\overline{\alpha := \sigma}] \vdash z} \rho [\overline{\gamma := C}] z
    ) \\
= & \lambda \overline{C}. \lambda z. (
        \setsem{\Gamma; \emptyset \,|\, \emptyset \vdash \map_{G}^{\overline{\sigma}, \overline{\tau}}
        :\Nat^{\emptyset} (\overline{\Nat^{\overline{\gamma}} \sigma\, \tau})\, (\Nat^{\overline{\gamma}} G[\overline{\alpha := \sigma}]\, G[\overline{\alpha := \tau}])}
        \rho \ast \overline{f}
      )_{\overline{C}} \\
  & (\eta_{\overline{\setsem{\Gamma; \overline{\gamma} \vdash \sigma}\rho [\overline{\gamma := C}]}, \overline{C}} z) \\
= & \lambda \overline{C}. \lambda z.
      \setsem{\Gamma; \overline{\alpha}, \overline{\gamma} \vdash G} \Id_{\rho [\overline{\gamma := C}]} [\overline{\alpha := f_{\overline{C}}}]
      (\eta_{\overline{\setsem{\Gamma; \overline{\gamma} \vdash \sigma}\rho [\overline{\gamma := C}]}, \overline{C}} z) \\
= & \lambda \overline{C}.
      \setsem{\Gamma; \overline{\alpha}, \overline{\gamma} \vdash G} \Id_{\rho [\overline{\gamma := C}]} [\overline{\alpha := f_{\overline{C}}}]
      \circ \eta_{\overline{\setsem{\Gamma; \overline{\gamma} \vdash \sigma}\rho [\overline{\gamma := C}]}, \overline{C}}
\end{array}
\]
By naturality of $\eta: \lambda \ol{A}. \lambda \ol{C}. \setsem{\Gamma; \overline{\alpha}, \overline{\gamma} \vdash F} \rho [\overline{\alpha := A}] [\overline{\gamma := C}]
\to \eta: \lambda \ol{A}. \lambda \ol{C}. \setsem{\Gamma; \overline{\alpha}, \overline{\gamma} \vdash G} \rho [\overline{\alpha := A}] [\overline{\gamma := C}]$
we have that
\[
\setsem{\Gamma; \overline{\alpha}, \overline{\gamma} \vdash G} \Id_{\rho [\overline{\gamma := C}]} [\overline{\alpha := f_{\overline{C}}}]
  \circ \eta_{\overline{\setsem{\Gamma; \overline{\gamma} \vdash \sigma}\rho [\overline{\gamma := C}]}, \overline{C}}
= \eta_{\overline{\setsem{\Gamma; \overline{\gamma} \vdash \tau}\rho [\overline{\gamma := C}]}, \overline{C}}
  \circ \setsem{\Gamma; \overline{\alpha}, \overline{\gamma} \vdash F} \Id_{\rho [\overline{\gamma := C}]} [\overline{\alpha := f_{\overline{C}}}]
\]
Proceeding analogously for the second term, we have that
% \begin{multline*}
%   \lambda \overline{C}. \eta_{\overline{\setsem{\Gamma; \overline{\gamma} \vdash \tau}\rho [\overline{\gamma := C}]}, \overline{C}}
%   \circ \setsem{\Gamma; \overline{\alpha}, \overline{\gamma} \vdash F} \Id_{\rho [\overline{\gamma := C}]} [\overline{\alpha := f_{\overline{C}}}] \\
% = \setsem{\Gamma; \emptyset \,|\, x : \Nat^{\overline{\alpha}, \overline{\gamma}} F\, G, \overline{y : \Nat^{\overline{\gamma}} \sigma\, \tau}
%   \vdash (L_{\overline{\gamma}} z. x_{\overline{\tau}, \overline{\gamma}} z) \circ ((\map_F^{\overline{\sigma}, \overline{\tau}})_{\emptyset} \overline{y})
%   : \Nat^{\overline{\gamma}} F[\overline{\alpha := \sigma}]\, G[\overline{\alpha := \tau}]}\rho (\eta, \overline{f})
% \end{multline*}
\[
\begin{array}{cl}
  & \lambda \overline{C}. \eta_{\overline{\setsem{\Gamma; \overline{\gamma} \vdash \tau}\rho [\overline{\gamma := C}]}, \overline{C}}
    \circ \setsem{\Gamma; \overline{\alpha}, \overline{\gamma} \vdash F} \Id_{\rho [\overline{\gamma := C}]} [\overline{\alpha := f_{\overline{C}}}] \\
= & \lambda \overline{C}. \lambda k.
    \eta_{\overline{\setsem{\Gamma; \overline{\gamma} \vdash \tau}\rho [\overline{\gamma := C}]}, \overline{C}}
    (\setsem{\Gamma; \overline{\alpha}, \overline{\gamma} \vdash F} \Id_{\rho [\overline{\gamma := C}]} [\overline{\alpha := f_{\overline{C}}}] k) \\
= & \lambda \overline{C}. \lambda k. \eta_{\overline{\setsem{\Gamma; \overline{\gamma} \vdash \tau}\rho [\overline{\gamma := C}]}, \overline{C}} \\
  & ((\setsem{\Gamma; \emptyset \,|\, \emptyset \vdash \map_{F}^{\overline{\sigma}, \overline{\tau}}
      :\Nat^{\emptyset} (\overline{\Nat^{\overline{\gamma}} \sigma\, \tau})\, (\Nat^{\overline{\gamma}} F[\overline{\alpha := \sigma}]\, F[\overline{\alpha := \tau}])}
      \rho \ast \overline{f}
    )_{\overline{C}} k) \\
= & \lambda \overline{C}. \lambda k. (
      \setsem{\Gamma; \emptyset \,|\, x : \Nat^{\overline{\alpha}, \overline{\gamma}} F\, G \vdash x}\rho \eta
    )_{\overline{\setsem{\Gamma; \overline{\gamma} \vdash \tau}\rho [\overline{\gamma := C}]}, \overline{\setsem{\Gamma; \overline{\gamma} \vdash \gamma}\rho [\overline{\gamma := C}]}} \\
  & \big(
      (\setsem{\Gamma; \emptyset \,|\, \emptyset \vdash \map_{F}^{\overline{\sigma}, \overline{\tau}}
        :\Nat^{\emptyset} (\overline{\Nat^{\overline{\gamma}} \sigma\, \tau})\, (\Nat^{\overline{\gamma}} F[\overline{\alpha := \sigma}]\, F[\overline{\alpha := \tau}])}
        \rho \ast \overline{f}
      )_{\overline{C}} k \big) \\
= & \lambda \overline{C}. \lambda k. (
      \setsem{\Gamma; \emptyset \,|\, x : \Nat^{\overline{\alpha}, \overline{\gamma}} F\, G \vdash x}\rho \eta
    )_{\overline{\setsem{\Gamma; \overline{\gamma} \vdash \tau}\rho [\overline{\gamma := C}]}, \overline{\setsem{\Gamma; \overline{\gamma} \vdash \gamma}\rho [\overline{\gamma := C}]}} \\
  & \big(
      (\setsem{\Gamma; \emptyset \,|\, \overline{y: \Nat^{\overline{\gamma}} \sigma\, \tau} \vdash (\map_{F}^{\overline{\sigma}, \overline{\tau}})_{\emptyset} \ol{y}
        : \Nat^{\overline{\gamma}} F[\overline{\alpha := \sigma}]\, F[\overline{\alpha := \tau}]}
        \rho \overline{f}
      )_{\overline{C}} k \big) \\
= & \lambda \overline{C}. \lambda k. (\setsem{\Gamma; \ol{\gamma} \,|\, x : \Nat^{\overline{\alpha}, \overline{\gamma}} F\, G, z: F[\ol{\alpha := \tau}]
    \vdash x_{\ol{\tau}, \ol{\gamma}} z}\rho [\ol{\gamma := C}] \eta ) \\
  & \big(
      \setsem{\Gamma; \ol{\gamma} \,|\, \overline{y: \Nat^{\overline{\gamma}} \sigma\, \tau}, k: F[\overline{\alpha := \sigma}]
        \vdash ( (\map_{F}^{\overline{\sigma}, \overline{\tau}})_{\emptyset} \ol{y} )_{\ol{\gamma}} k
        : F[\overline{\alpha := \tau}]}
        \rho [\ol{\gamma := C}] \overline{f} k
    \big) \\
= & \lambda \overline{C}. \lambda k. (\setsem{\Gamma; \emptyset \,|\, x : \Nat^{\overline{\alpha}, \overline{\gamma}} F\, G \vdash L_{\ol{\gamma}} z. x_{\ol{\tau}, \ol{\gamma}} z}\rho \eta)_{\ol{C}} \\
  & \big(
      \setsem{\Gamma; \ol{\gamma} \,|\, \overline{y: \Nat^{\overline{\gamma}} \sigma\, \tau}, k: F[\overline{\alpha := \sigma}]
        \vdash ( (\map_{F}^{\overline{\sigma}, \overline{\tau}})_{\emptyset} \ol{y} )_{\ol{\gamma}} k
        : F[\overline{\alpha := \tau}]}
        \rho [\ol{\gamma := C}] \overline{f} k
    \big) \\
= & \lambda \overline{C}. \lambda k.
    \setsem{\Gamma; \ol{\gamma} \,|\, x : \Nat^{\overline{\alpha}, \overline{\gamma}} F\, G, \overline{y: \Nat^{\overline{\gamma}} \sigma\, \tau}, k: F[\overline{\alpha := \sigma}]
      \vdash (L_{\ol{\gamma}} z. x_{\ol{\tau}, \ol{\gamma}} z)_{\ol{\gamma}} ( ( (\map_{F}^{\overline{\sigma}, \overline{\tau}})_{\emptyset} \ol{y} )_{\ol{\gamma}} k )
      : G[\overline{\alpha := \tau}]}
    \rho [\ol{\gamma := C}] \eta \overline{f} k \\
= & \setsem{\Gamma; \emptyset \,|\, x : \Nat^{\overline{\alpha}, \overline{\gamma}} F\, G, \overline{y : \Nat^{\overline{\gamma}} \sigma\, \tau}
  \vdash L_{\overline{\gamma}} k. (L_{\overline{\gamma}} z. x_{\overline{\tau}, \overline{\gamma}} z)_{\ol{\gamma}} (((\map_F^{\overline{\sigma}, \overline{\tau}})_{\emptyset} \overline{y})_{\ol{\gamma}} k)
  : \Nat^{\overline{\gamma}} F[\overline{\alpha := \sigma}]\, G[\overline{\alpha := \tau}]}\rho \eta \overline{f} \\
= & \setsem{\Gamma; \emptyset \,|\, x : \Nat^{\overline{\alpha}, \overline{\gamma}} F\, G, \overline{y : \Nat^{\overline{\gamma}} \sigma\, \tau}
  \vdash (L_{\overline{\gamma}} z. x_{\overline{\tau}, \overline{\gamma}} z) \circ ((\map_F^{\overline{\sigma}, \overline{\tau}})_{\emptyset} \overline{y})
  : \Nat^{\overline{\gamma}} F[\overline{\alpha := \sigma}]\, G[\overline{\alpha := \tau}]}\rho \eta \overline{f}
\end{array}
\]
So, we conclude that
\begin{multline*}
  \setsem{\Gamma; \emptyset \,|\, x : \Nat^{\overline{\alpha}, \overline{\gamma}} F\, G, \overline{y : \Nat^{\overline{\gamma}} \sigma\, \tau}
      \vdash ((\map_{G}^{\overline{\sigma}, \overline{\tau}})_{\emptyset} \overline{y})_{\overline{\gamma}} \circ (L_{\overline{\gamma}} z. x_{\overline{\sigma}, \overline{\gamma}} z)
      : \Nat^{\overline{\gamma}} F[\overline{\alpha := \sigma}]\, G[\overline{\alpha := \tau}]}\rho \\
= \setsem{\Gamma; \emptyset \,|\, x : \Nat^{\overline{\alpha}, \overline{\gamma}} F\, G, \overline{y : \Nat^{\overline{\gamma}} \sigma\, \tau}
  \vdash (L_{\overline{\gamma}} z. x_{\overline{\tau}, \overline{\gamma}} z) \circ ((\map_F^{\overline{\sigma}, \overline{\tau}})_{\emptyset} \overline{y})
  : \Nat^{\overline{\gamma}} F[\overline{\alpha := \sigma}]\, G[\overline{\alpha := \tau}]}\rho
\end{multline*}
% are functors and $f_{\overline{C}}$ is a function
% $\setsem{\Gamma; \overline{\gamma} \vdash \sigma}\rho [\overline{\gamma := C}] \to \setsem{\Gamma; \overline{\gamma} \vdash \tau}\rho [\overline{\gamma := C}]$,
% \[
% \setsem{\Gamma; \overline{\alpha}, \overline{\gamma} \vdash G} \Id_{\rho [\overline{\gamma := C}]} [\overline{\alpha := f_{\overline{C}}}]
%   \circ \eta_{\overline{\setsem{\Gamma; \overline{\gamma} \vdash \sigma}\rho [\overline{\gamma := C}]}, \overline{C}}
% = \eta_{\overline{\setsem{\Gamma; \overline{\gamma} \vdash \tau}\rho [\overline{\gamma := C}]}, \overline{C}}
%   \circ \setsem{\Gamma; \overline{\alpha}, \overline{\gamma} \vdash F} \Id_{\rho [\overline{\gamma := C}]} [\overline{\alpha := f_{\overline{C}}}]
% \]


% The semantic interpretation of the second term is, analogously,
% \[
% \begin{array}{cl}
%   & \setsem{\Gamma; \emptyset \,|\, \eta : \Nat^{\overline{\alpha}, \overline{\gamma}} F G, \overline{f : \Nat^{\overline{\gamma}} \sigma \tau}
%       \vdash L_{\overline{\gamma}} x. \eta_{\overline{\tau}, \overline{\gamma}} ((\map_{F}^{\overline{\sigma}, \overline{\tau}})_{\emptyset} \overline{f})_{\overline{\gamma}} x
%       : \Nat^{\overline{\gamma}} F[\overline{\alpha := \sigma}] G[\overline{\alpha := \tau}]}\rho (\eta, \overline{f}) \\
% = & \lambda \overline{C}.
%       \eta_{\overline{\setsem{\Gamma; \overline{\gamma} \vdash \tau}\rho [\overline{\gamma := C}]}, \overline{C}}
%       \circ \setsem{\Gamma; \overline{\alpha}, \overline{\gamma} \vdash F} \Id_{\rho [\overline{\gamma := C}]} [\overline{\alpha := f_{\overline{C}}}]
% \end{array}
% \]
% By naturality of $\eta$, and the fact that
% $\setsem{\Gamma; \overline{\alpha}, \overline{\gamma} \vdash G} \rho [\overline{\gamma := C}] [\overline{\alpha := \_}]$
% and
% $\setsem{\Gamma; \overline{\alpha}, \overline{\gamma} \vdash G} \rho [\overline{\gamma := C}] [\overline{\alpha := \_}]$
% are functors and $f_{\overline{C}}$ is a function
% $\setsem{\Gamma; \overline{\gamma} \vdash \sigma}\rho [\overline{\gamma := C}] \to \setsem{\Gamma; \overline{\gamma} \vdash \tau}\rho [\overline{\gamma := C}]$,
% \[
% \setsem{\Gamma; \overline{\alpha}, \overline{\gamma} \vdash G} \Id_{\rho [\overline{\gamma := C}]} [\overline{\alpha := f_{\overline{C}}}]
%   \circ \eta_{\overline{\setsem{\Gamma; \overline{\gamma} \vdash \sigma}\rho [\overline{\gamma := C}]}, \overline{C}}
% = \eta_{\overline{\setsem{\Gamma; \overline{\gamma} \vdash \tau}\rho [\overline{\gamma := C}]}, \overline{C}}
%   \circ \setsem{\Gamma; \overline{\alpha}, \overline{\gamma} \vdash F} \Id_{\rho [\overline{\gamma := C}]} [\overline{\alpha := f_{\overline{C}}}]
% \]
% So, we can conclude that
% \[
% \begin{array}{cl}
%   & \setsem{\Gamma; \emptyset \,|\, \eta : \Nat^{\overline{\alpha}, \overline{\gamma}} F G, \overline{f : \Nat^{\overline{\gamma}} \sigma \tau}
%       \vdash L_{\overline{\gamma}} x. ((\map_{G}^{\overline{\sigma}, \overline{\tau}})_{\emptyset} \overline{f})_{\overline{\gamma}} (\eta_{\overline{\sigma}, \overline{\gamma}} x)
%       : \Nat^{\overline{\gamma}} F[\overline{\alpha := \sigma}] G[\overline{\alpha := \tau}]}\rho (\eta, \overline{f}) \\
% = & \setsem{\Gamma; \emptyset \,|\, \eta : \Nat^{\overline{\alpha}, \overline{\gamma}} F G, \overline{f : \Nat^{\overline{\gamma}} \sigma \tau}
%       \vdash L_{\overline{\gamma}} x. \eta_{\overline{\tau}, \overline{\gamma}} ((\map_{F}^{\overline{\sigma}, \overline{\tau}})_{\emptyset} \overline{f})_{\overline{\gamma}} x
%       : \Nat^{\overline{\gamma}} F[\overline{\alpha := \sigma}] G[\overline{\alpha := \tau}]}\rho (\eta, \overline{f})
% \end{array}
% \]
The case for the relational interpretation is analogous.
\end{proof}

%From Lemma~\ref{lemma:naturality} we can deduce that

We have a special case of Lemma~\ref{lem:naturality} following from
the naturality of $\tin_H : \Nat^{\ol{\beta},\ol{\gamma}}\,H[\phi :=
  (\lambda \phi. \lambda \ol{\alpha}. H)\ol{\beta}][\ol{\alpha :=
    \beta}]\; (\mu \phi.\lambda \ol{\alpha}.H)\ol{\beta}$:

\begin{cor}
Let $\Gamma; \phi, \ol{\alpha}, \ol{\gamma} \vdash H$ and  $\ol{\Gamma; \gamma \vdash \sigma}$ and $\ol{\Gamma; \gamma \vdash \tau}$ be types.
Then the terms
\begin{align*}
  \Gamma; \emptyset \,|\, \overline{y : \Nat^{\overline{\gamma}} \sigma\, \tau}
  &\vdash ((\map_{(\mu \phi. \lambda \overline{\alpha}. H) \overline{\beta}}^{\overline{\sigma}, \overline{\tau}})_{\emptyset} \overline{y})
  \circ (L_{\overline{\gamma}} z. (\tin_{H})_{\overline{\sigma}, \overline{\gamma}} z) \\
  &: \Nat^{\overline{\gamma}} H[\phi := (\mu \phi. \lambda \overline{\alpha}. H) \overline{\beta}][\overline{\alpha := \sigma}]\, (\mu \phi. \lambda \overline{\alpha}. H)\overline{\tau}
\end{align*}
and
\begin{align*}
  \Gamma; \emptyset \,|\, \overline{y : \Nat^{\overline{\gamma}} \sigma\, \tau}
  &\vdash (L_{\overline{\gamma}} z. (\tin_H)_{\overline{\tau}, \overline{\gamma}} z)
  \circ ((\map_{H[\phi := (\mu \phi. \lambda \overline{\alpha}. H) \overline{\beta}]}^{\overline{\sigma}, \overline{\tau}})_{\emptyset} \overline{y}) \\
  &: \Nat^{\overline{\gamma}} H[\phi := (\mu \phi. \lambda \overline{\alpha}. H) \overline{\beta}][\overline{\alpha := \sigma}]\, (\mu \phi. \lambda \overline{\alpha}. H)\overline{\tau}
\end{align*}
are semantically equivalent.
\end{cor}
\begin{proof}
The two terms are derived from those in Lemma~\ref{lem:naturality} by instantiating the term variable $x$ with
$\tin_H : \Nat^{\ol{\beta}, \ol{\gamma}} H[\phi := (\mu \phi. \lambda \ol{\alpha}. H)\ol{\beta}][\ol{\alpha := \beta}]\; (\mu \phi. \lambda \ol{\alpha}. H)\ol{\beta}$.
\end{proof}

The next lemma states that the map of the composition of two functors is
(equivalent to) the composition of their maps.

\begin{lemma}\label{lemma:sem-funct-comp}
Let
\[
\begin{array}{cccc}
\Gamma; \ol{\psi}, \ol{\gamma} \vdash H
&\ol{\Gamma; \ol{\alpha}, \ol{\gamma}, \ol{\phi} \vdash K}
&\ol{\Gamma; \ol{\beta}, \ol{\gamma} \vdash F}
&\ol{\Gamma; \ol{\beta}, \ol{\gamma} \vdash G}
\end{array}
\]
be types. Then the terms
\[
\Gamma; \emptyset \,|\, \emptyset
\vdash \map_{H[\ol{\psi := K}]}^{\ol{F}, \ol{G}}
: \Nat^{\emptyset} (\ol{\Nat^{\ol{\alpha}, \ol{\beta}, \ol{\gamma}} F
  G}) (\Nat^{\ol{\gamma}} H[\ol{\psi := K}][\ol{\phi := F}] \; H[\ol{\psi := K}][\ol{\phi := G}])
\]
and
\[
\Gamma; \emptyset \,|\, \emptyset
\vdash \map_H^{\ol{K[\ol{\phi := F}]}, \ol{K[\ol{\phi := G}]}} \circ \ol{\map_K^{\ol{F}, \ol{G}}}
: \Nat^{\emptyset} (\ol{\Nat^{\ol{\alpha}, \ol{\beta}, \ol{\gamma}} F G}) (\Nat^{\ol{\gamma}} H[\ol{\psi := K[\ol{\phi := F}]}] \;H[\ol{\psi := K[\ol{\phi := G}]}])
\]
are semantically equivalent.
(Notice that $F$ and $G$'s context is extended with the $\ol{\alpha}$ variables by weakening).
\end{lemma}
\begin{proof}
Throughout this proof we shall use the fact that the variables
$\ol{\alpha}$ can be added to the environment of $F$ and $G$ by
weakening, even though they do not appear in those types.  As a
consequence, which is true only because $F$ and $G$ contain no
$\ol{\alpha}$'s, $H[\ol{\psi :=_{\ol{\alpha}} K[\ol{\phi
        :=_{\ol{\beta}} F}]}] = H[\ol{\psi :=_{\ol{\alpha}}
    K}][\ol{\phi :=_{\ol{\beta}} F}]$ and $H[\ol{\psi :=_{\ol{\alpha}}
    K[\ol{\phi :=_{\ol{\beta}} G}]}] = H[\ol{\psi :=_{\ol{\alpha}}
    K}][\ol{\phi :=_{\ol{\beta}} G}]$.  Also, observe that a natural
transformation $\eta : \setsem{\Gamma; \emptyset \vdash
  \Nat^{\ol{\beta}, \ol{\gamma}} F G}\rho$ corresponds, by weakening,
to a natural transformation $\eta : \setsem{\Gamma; \emptyset \vdash
  \Nat^{\ol{\alpha}, \ol{\beta}, \ol{\gamma}} F G}\rho$ which is
trivially natural in the $\alpha$'s.  Then for every natural
transformation $\eta : \setsem{\Gamma; \emptyset \vdash
  \Nat^{\ol{\beta}, \ol{\gamma}} F G}\rho$, $\ol{C : \set}$, and
$\ast$ the unique element of the singleton, we have that
\[
%CAN BE SIMPLIFIED USING LEMMA~\ref{lem:comp}
\begin{array}{rl}
& (\setsem{\Gamma; \emptyset \,|\, \emptyset \vdash \map_H^{\ol{K[\ol{\phi := F}]}, \ol{K[\ol{\phi := G}]}} \circ \ol{\map_K^{\ol{F}, \ol{G}}}}\rho \ast \ol{\eta})_{\ol{C}} \\
= & (\setsem{\Gamma; \emptyset \,|\, \emptyset \vdash \map_H^{\ol{K[\ol{\phi := F}]}, \ol{K[\ol{\phi := G}]}}} \rho \ast
	(\ol{\setsem{\Gamma; \emptyset \,|\, \emptyset \vdash \map_K^{\ol{F}, \ol{G}}}\rho \ast \ol{\eta}}))_{\ol{C}} \\
= & \setsem{\Gamma; \ol{\psi}, \ol{\gamma} \vdash H} \Id_{\rho [\ol{\gamma := C}]}
	[\ol{\psi := \lambda \ol{A}. (\setsem{\Gamma; \emptyset \,|\, \emptyset \vdash \map_K^{\ol{F}, \ol{G}}}\rho \ast \ol{\eta})_{\ol{A}, \ol{C}}}] \\
= & \setsem{\Gamma; \ol{\psi}, \ol{\gamma} \vdash H} \Id_{\rho [\ol{\gamma := C}]}
	[\ol{
    		\psi := \lambda \ol{A}. \setsem{\Gamma; \ol{\alpha}, \ol{\gamma}, \ol{\phi} \vdash K}
		\Id_{\rho [\ol{\alpha := A}] [\ol{\gamma := C}]} [\ol{\phi := \lambda \ol{B}. \eta_{\ol{A}, \ol{B}, \ol{C}}}]
	}] \\
= & \setsem{\Gamma; \ol{\psi}, \ol{\gamma} \vdash H} \Id_{\rho [\ol{\gamma := C}]}
	[\ol{
    		\psi := \lambda \ol{A}. \setsem{\Gamma; \ol{\alpha}, \ol{\gamma}, \ol{\phi} \vdash K}
		\Id_{\rho [\ol{\alpha := A}] [\ol{\gamma := C}]} [\ol{\phi := \lambda \ol{B}. \eta_{\ol{B}, \ol{C}}}]
	}] \\
= & \setsem{\Gamma; \ol{\psi}, \ol{\gamma} \vdash H} \Id_{\rho [\ol{\gamma := C}]}[\ol{\phi := \lambda \ol{B}. \eta_{\ol{B}, \ol{C}}}]
	[\ol{
    		\psi := \lambda \ol{A}. \setsem{\Gamma; \ol{\alpha}, \ol{\gamma}, \ol{\phi} \vdash K}
		\Id_{\rho [\ol{\alpha := A}] [\ol{\gamma := C}]} [\ol{\phi := \lambda \ol{B}. \eta_{\ol{B}, \ol{C}}}]
	}] \\
	% \setsem{\Gamma; \emptyset \,|\, \emptyset \vdash \map_K^{\ol{F}, \ol{G}}}\rho \ast \ol{\eta})_{\ol{A}, \ol{C}}}]
	% (\ol{\setsem{\Gamma; \emptyset \,|\, \emptyset \vdash \map_K^{\ol{F}, \ol{G}}}\rho \ast \ol{\eta}}) \\
= & \setsem{\Gamma; \ol{\gamma}, \ol{\phi} \vdash H[\ol{\psi := K}]} Id_{\rho [\ol{\gamma := C}]} [\ol{\phi := \lambda \ol{B}. \eta_{\ol{B}, \ol{C}}}] \\
= & (\setsem{\Gamma; \emptyset \,|\, \emptyset \vdash \map_{H[\ol{\psi := K}]}^{\ol{F}, \ol{G}}} \rho \ast \ol{\eta})_{\ol{C}}
\end{array}
\]
The case for the relational interpretation is analogous.
\end{proof}

Type application $\phi \ol{\tau}$ yields a map acting functorially on both the type constructor variable $\phi$ and its  arguments $\ol{\tau}$.
Such action consists in mapping along the type constructor variable first and the arguments later, or, equivalently,
the arguments first and the type constructor variable later.
That the two ways of describing the action are equivalent is due to naturality.

\begin{lemma}[Map of type application]\label{lem:map_type_application}
Consider the following types
\[
\begin{array}{ccccc}
\overline{\Gamma; \phi, \overline{\psi}, \overline{\gamma} \vdash \tau}
&\Gamma; \overline{\beta}, \overline{\gamma} \vdash H
&\Gamma; \overline{\beta}, \overline{\gamma} \vdash K
&\overline{\Gamma; \overline{\alpha}, \overline{\gamma} \vdash F}
&\overline{\Gamma; \overline{\alpha}, \overline{\gamma} \vdash G}
\end{array}
\]
Let $\overline{I} = \overline{F}, H$
and $\overline{J} = \overline{G}, K$
be lists of types.
Then the terms
\begin{multline}\label{lem:map-type-app-1}
    \Gamma; \emptyset \,|\, \emptyset
    \vdash L_{\emptyset} (x, \overline{y}). L_{\overline{\gamma}} z.
    x_{\overline{\tau [\overline{\psi := G}] [\phi := K]}, \overline{\gamma}}
    \Big(\big(
      (\map_{H}^{\overline{\tau [\overline{\psi := F}] [\phi := H]}, \overline{\tau [\overline{\psi := G}] [\phi := K]}})_{\emptyset}
      (\overline{(\map_{\tau}^{\overline{I}, \overline{J}})_{\emptyset} (x, \overline{y})})
    \big)_{\overline{\gamma}} z \Big) \\
    : \Nat^{\emptyset}
      (\Nat^{\overline{\beta}, \overline{\gamma}} H\;K \times \overline{\Nat^{\overline{\alpha}, \overline{\gamma}} F\, G})\,
      (\Nat^{\overline{\gamma}} H[\overline{\beta:= \tau}] [\overline{\psi := F}] [\phi := H]\; K[\overline{\beta := \tau}] [\overline{\psi := G}] [\phi := K])
\end{multline}
and
\begin{multline}\label{lem:map-type-app-2}
    \Gamma; \emptyset \,|\, \emptyset
    \vdash L_{\emptyset} (x, \overline{y}). L_{\overline{\gamma}} z.
    \big(
      (\map_{K}^{\overline{\tau [\overline{\psi := F}] [\phi := H]}, \overline{\tau [\overline{\psi := G}] [\phi := K]}})_{\emptyset}
      (\overline{(\map_{\tau}^{\overline{I}, \overline{J}})_{\emptyset} (x, \overline{y})})
    \big)_{\overline{\gamma}}
    \Big( x_{\overline{\tau [\overline{\psi := F}] [\phi := H]}, \overline{\gamma}} z \Big) \\
    : \Nat^{\emptyset}
      (\Nat^{\overline{\beta}, \overline{\gamma}} H\, K \times \overline{\Nat^{\overline{\alpha}, \overline{\gamma}} F\, G})\,
      (\Nat^{\overline{\gamma}} H[\overline{\beta:= \tau}] [\overline{\psi := F}] [\phi := H]\; K[\overline{\beta := \tau}] [\overline{\psi := G}] [\phi := K])
\end{multline}
are semantically equivalent to $\map_{\phi \overline{\tau}}^{\overline{I}, \overline{J}}$.
% \begin{prooftree}
% \AXC{$\overline{\Gamma; \phi, \overline{\psi}, \overline{\gamma} \vdash \tau}$}
% \AXC{$\overline{\Gamma; \overline{\alpha}, \overline{\gamma} \vdash F}$}
% \AXC{$\overline{\Gamma; \overline{\alpha}, \overline{\gamma} \vdash G}$}
% \AXC{$\Gamma; \overline{\beta}, \overline{\gamma} \vdash H$}
% \AXC{$\Gamma; \overline{\beta}, \overline{\gamma} \vdash K$}
%   \QuinaryInfC{$\begin{array}{c}
%     \Gamma; \emptyset \,|\, \emptyset
%     \vdash L_{\emptyset} (\overline{\epsilon}, \eta). L_{\overline{\gamma}} x.
%     \eta_{\overline{\tau [\overline{\psi := G}] [\phi := K]}, \overline{\gamma}}
%     \Big(\big(
%       (\map_{H}^{\overline{\tau [\overline{\psi := F}] [\phi := H]}; \overline{\tau [\overline{\psi := G}] [\phi := K]}})_{\emptyset}
%       (\overline{(\map_{\tau}^{\overline{F}, H; \overline{G}, K})_{\emptyset} (\overline{\epsilon}, \eta)})
%     \big)_{\overline{\gamma}} x \Big) \\
%     : \Nat^{\emptyset}
%       (\overline{\Nat^{\overline{\alpha}, \overline{\gamma}} F G} \times \Nat^{\overline{\beta}, \overline{\gamma}} H K)
%       (\Nat^{\overline{\gamma}} H[\overline{\beta:= \tau [\overline{\psi := F}] [\phi := H]}] K[\overline{\beta := \tau [\overline{\psi := G}] [\phi := K]}])
%     \end{array}$}
% \end{prooftree}
% and
% \begin{prooftree}
% \AXC{$\overline{\Gamma; \phi, \overline{\psi}, \overline{\gamma} \vdash \tau}$}
% \AXC{$\overline{\Gamma; \overline{\alpha}, \overline{\gamma} \vdash F}$}
% \AXC{$\overline{\Gamma; \overline{\alpha}, \overline{\gamma} \vdash G}$}
% \AXC{$\Gamma; \overline{\beta}, \overline{\gamma} \vdash H$}
% \AXC{$\Gamma; \overline{\beta}, \overline{\gamma} \vdash K$}
%   \QuinaryInfC{$\begin{array}{c}
%     \Gamma; \emptyset \,|\, \emptyset
%     \vdash L_{\emptyset} (\overline{\epsilon}, \eta). L_{\overline{\gamma}} x.
%     \big(
%       (\map_{K}^{\overline{\tau [\overline{\psi := F}] [\phi := H]}; \overline{\tau [\overline{\psi := G}] [\phi := K]}})_{\emptyset}
%       (\overline{(\map_{\tau}^{\overline{F}, H; \overline{G}, K})_{\emptyset} (\overline{\epsilon}, \eta)})
%     \big)_{\overline{\gamma}}
%     \Big( \eta_{\overline{\tau [\overline{\psi := F}] [\phi := H]}, \overline{\gamma}} x \Big) \\
%     : \Nat^{\emptyset}
%       (\overline{\Nat^{\overline{\alpha}, \overline{\gamma}} F G} \times \Nat^{\overline{\beta}, \overline{\gamma}} H K)
%       (\Nat^{\overline{\gamma}} H[\overline{\beta:= \tau [\overline{\psi := F}] [\phi := H]}] K[\overline{\beta := \tau [\overline{\psi := G}] [\phi := K]}])
%     \end{array}$}
% \end{prooftree}
%meaning that, for any set environment $\rho$, we have that
%\[
%\begin{array}{rl}
%  & \setsem{L_{\emptyset} (x, \overline{y}). L_{\overline{\gamma}} z.
%    x_{\overline{\tau [\overline{\psi := G}] [\phi := K]}, \overline{\gamma}}
%    \Big(\big(
%      (\map_{H}^{\overline{\tau [\overline{\psi := F}] [\phi := H]}, \overline{\tau [\overline{\psi := G}] [\phi := K]}})_{\emptyset}
%      (\overline{(\map_{\tau}^{\overline{I}, \overline{J}})_{\emptyset} (x, \overline{y})})
%    \big)_{\overline{\gamma}} x \Big)}\rho \\
%= & \setsem{L_{\emptyset} (x, \overline{y}). L_{\overline{\gamma}} z.
%    \big(
%      (\map_{K}^{\overline{\tau [\overline{\psi := F}] [\phi := H]}, \overline{\tau [\overline{\psi := G}] [\phi := K]}})_{\emptyset}
%      (\overline{(\map_{\tau}^{\overline{I}, \overline{J}})_{\emptyset} (x, \overline{y})})
%    \big)_{\overline{\gamma}}
%    \Big( x_{\overline{\tau [\overline{\psi := F}] [\phi := H]}, \overline{\gamma}} z \Big)}\rho \\
%= & \setsem{\map_{\phi \overline{\tau}}^{\overline{I}, \overline{J}}}\rho
%\end{array}
%\]
%and, analogously, for any relation environment $\rho$, we have that
%\[
%\begin{array}{rl}
%  & \relsem{L_{\emptyset} (x, \overline{y}). L_{\overline{\gamma}} z.
%    x_{\overline{\tau [\overline{\psi := G}] [\phi := K]}, \overline{\gamma}}
%    \Big(\big(
%      (\map_{H}^{\overline{\tau [\overline{\psi := F}] [\phi := H]}, \overline{\tau [\overline{\psi := G}] [\phi := K]}})_{\emptyset}
%      (\overline{(\map_{\tau}^{\overline{I}, \overline{J}})_{\emptyset} (x, \overline{y})})
%    \big)_{\overline{\gamma}} x \Big)}\rho \\
%= & \relsem{L_{\emptyset} (x, \overline{y}). L_{\overline{\gamma}} z.
%    \big(
%      (\map_{K}^{\overline{\tau [\overline{\psi := F}] [\phi := H]}, \overline{\tau [\overline{\psi := G}] [\phi := K]}})_{\emptyset}
%      (\overline{(\map_{\tau}^{\overline{I}, \overline{J}})_{\emptyset} (x, \overline{y})})
%    \big)_{\overline{\gamma}}
%    \Big( x_{\overline{\tau [\overline{\psi := F}] [\phi := H]}, \overline{\gamma}} z \Big)}\rho \\
%= & \relsem{\map_{\phi \overline{\tau}}^{\overline{I}, \overline{J}}}\rho
%\end{array}
%\]
\end{lemma}
\begin{proof}
% The proof for the second term from the statement is similar.
To begin with, observe that the terms~\ref{lem:map-type-app-1}
and~\ref{lem:map-type-app-2} are semantically equivalent.  This
follows from the fact that the terms
\begin{multline*}
    \Gamma; \emptyset \,|\, x, \ol{y}
    \vdash (L_{\overline{\gamma}} w. x_{\overline{\tau [\overline{\psi := G}] [\phi := K]}, \overline{\gamma}} w )
    \circ
    \big(
      (\map_{H}^{\overline{\tau [\overline{\psi := F}] [\phi := H]}, \overline{\tau [\overline{\psi := G}] [\phi := K]}})_{\emptyset}
      (\overline{(\map_{\tau}^{\overline{I}, \overline{J}})_{\emptyset} (x, \overline{y})})
    \big) \\
    : \Nat^{\overline{\gamma}} H[\overline{\beta:= \tau}] [\overline{\psi := F}] [\phi := H]\, K[\overline{\beta := \tau}] [\overline{\psi := G}] [\phi := K]
\end{multline*}
and
\begin{multline*}
    \Gamma; \emptyset \,|\, x, \ol{y}
    \vdash
    \big(
      (\map_{K}^{\overline{\tau [\overline{\psi := F}] [\phi := H]}, \overline{\tau [\overline{\psi := G}] [\phi := K]}})_{\emptyset}
      (\overline{(\map_{\tau}^{\overline{I}, \overline{J}})_{\emptyset} (x, \overline{y})})
    \big)
    \circ
    \Big( L_{\overline{\gamma}} z. x_{\overline{\tau [\overline{\psi := F}] [\phi := H]}, \overline{\gamma}} z \Big) \\
    : \Nat^{\overline{\gamma}} H[\overline{\beta:= \tau}] [\overline{\psi := F}] [\phi := H]\, K[\overline{\beta := \tau}] [\overline{\psi := G}] [\phi := K]
\end{multline*}
are semantically equivalent by Lemma~\ref{lem:naturality}.
Thus, it will suffice to prove that any of them, say term~\ref{lem:map-type-app-1}, is semantically equivalent to $\map_{\phi \overline{\tau}}^{\overline{I}, \overline{J}}$.

% To begin with, observe that the terms
% \begin{equation}
% (\map_{H}^{\overline{\tau [\overline{\psi := F}] [\phi := H]}, \overline{\tau [\overline{\psi := G}] [\phi := K]}})_{\emptyset}
%   \big( \overline{(\map_{\tau}^{\overline{I}, \overline{J}})_{\emptyset} (x, \overline{y})} \big)
% \end{equation}
% and
% \begin{equation}\label{lem:map-type-app-map}
% (\map_{H[\overline{\beta := \tau}]}^{\overline{I}, \overline{J}})_{\emptyset} (x, \overline{y})
% \end{equation}
% are semantically equivalent because of Lemma~\ref{lemma:sem-funct-comp}.
If $\rho$ is a set environment and $\ast$ is the unique element of the singleton, we have that
the interpretation of the term~\ref{lem:map-type-app-1} is given by
%the semantic interpretation of the first term in the statement is equal to
\[
\begin{array}{cl}
  & \setsem{\Gamma; \emptyset \,|\, \emptyset \vdash L_{\emptyset} (x, \overline{y}). L_{\overline{\gamma}} z.
      x_{\overline{\tau [\overline{\psi := G}] [\phi := K]}, \overline{\gamma}}
      \big(\big(
        (\map_{H[\overline{\beta := \tau}]}^{\overline{I}, \overline{J}})_{\emptyset} (x, \overline{y})
      \big)_{\overline{\gamma}} z \big)}\rho \ast \\
= & \lambda \eta. \lambda \overline{\epsilon}.
      \setsem{\Gamma; \emptyset \,|\, x, \overline{y}
      \vdash L_{\overline{\gamma}} z.
      x_{\overline{\tau [\overline{\psi := G}] [\phi := K]}, \overline{\gamma}}
      \big(\big(
        (\map_{H[\overline{\beta := \tau}]}^{\overline{I}, \overline{J}})_{\emptyset} (x, \overline{y})
      \big)_{\overline{\gamma}} z \big)}\rho \eta \overline{\epsilon} \\
= & \lambda \eta. \lambda \overline{\epsilon}. \lambda \overline{C}. \lambda z.
    \setsem{\Gamma; \overline{\gamma} \,|\, x, \overline{y}, z \vdash x_{\overline{\tau [\overline{\psi := G}] [\phi := K]}, \overline{\gamma}}
      \big(\big(
        (\map_{H[\overline{\beta := \tau}]}^{\overline{I}, \overline{J}})_{\emptyset} (x, \overline{y})
      \big)_{\overline{\gamma}} z \big)}\rho [\overline{\gamma := C}] \eta \overline{\epsilon} z \\
= & \lambda \eta. \lambda \overline{\epsilon}. \lambda \overline{C}. \lambda z.
    \eta_{\overline{\setsem{\Gamma; \overline{\gamma} \vdash \tau [\overline{\psi := G}] [\phi := K]} \rho [\overline{\gamma := C}]}, \overline{C}}
    (
      \setsem{\Gamma; \overline{\gamma} \,|\, x, \overline{y}, z \vdash \big(
        (\map_{H[\overline{\beta := \tau}]}^{\overline{I}, \overline{J}})_{\emptyset} (x, \overline{y})
      \big)_{\overline{\gamma}} z} \rho [\overline{\gamma := C}] \eta \overline{\epsilon} z
    ) \\
= & \lambda \eta. \lambda \overline{\epsilon}. \lambda \overline{C}. \lambda z.
    \eta_{\overline{\setsem{\Gamma; \overline{\gamma} \vdash \tau [\overline{\psi := G}] [\phi := K]} \rho [\overline{\gamma := C}]}, \overline{C}}
    (
      (
      \setsem{\Gamma; \emptyset \,|\, x, \overline{y} \vdash
          (\map_{H[\overline{\beta := \tau}]}^{\overline{I}, \overline{J}})_{\emptyset} (x, \overline{y})
        } \rho \eta \overline{\epsilon}
      )_{\overline{C}} z
    ) \\
= & \lambda \eta. \lambda \overline{\epsilon}. \lambda \overline{C}.
    \eta_{\overline{\setsem{\Gamma; \overline{\gamma} \vdash \tau [\overline{\psi := G}] [\phi := K]} \rho [\overline{\gamma := C}]}, \overline{C}}
    \circ
    (
      \setsem{\Gamma; \emptyset \,|\, x, \overline{y} \vdash
        (\map_{H[\overline{\beta := \tau}]}^{\overline{I}, \overline{J}})_{\emptyset} (x, \overline{y})
      } \rho \eta \overline{\epsilon}
    )_{\overline{C}}
\end{array}
\]
where the term variables are typed as $x : \Nat^{\overline{\beta}, \overline{\gamma}} H\, K$,
$\ol{y : \Nat^{\overline{\alpha}, \overline{\gamma}} F\, G}$,
and $z : H[\overline{\beta:= \tau}] [\overline{\psi := F}] [\phi := H]$.

Observe that the terms
\begin{equation*}
(\map_{H[\overline{\beta := \tau}]}^{\overline{I}, \overline{J}})_{\emptyset} (x, \overline{y})
\end{equation*}
and
\begin{equation*}
(\map_{H}^{\overline{\tau [\overline{\psi := F}] [\phi := H]}, \overline{\tau [\overline{\psi := G}] [\phi := K]}})_{\emptyset}
  \big( \overline{(\map_{\tau}^{\overline{I}, \overline{J}})_{\emptyset} (x, \overline{y})} \big)
\end{equation*}
are semantically equivalent because of
Lemma~\ref{lemma:sem-funct-comp}.
Then for all
$\eta: \setsem{\Gamma; \emptyset \vdash \Nat^{\overline{\beta}, \overline{\gamma}} H\, K}\rho$
and
$\overline{\epsilon: \setsem{\Gamma; \emptyset \vdash \Nat^{\overline{\alpha}, \overline{\gamma}} F\, G}}\rho$,
% the interpretation of
% \[
% \Gamma; \emptyset \,|\, \overline{\epsilon}, \eta \vdash
%   (\map_{H}^{\overline{\tau [\overline{\psi := F}] [\phi := H]}; \overline{\tau [\overline{\psi := G}] [\phi := K]}})_{\emptyset}
%   (\overline{(\map_{\tau}^{\overline{F}, H; \overline{G}, K})_{\emptyset} (\overline{\epsilon}, \eta)})
% \]
% is given by
we have that
\[
\begin{array}{cl}
  & (
      \setsem{\Gamma; \emptyset \,|\, x, \overline{y} \vdash
        (\map_{H[\overline{\beta := \tau}]}^{\overline{I}, \overline{J}})_{\emptyset} (x, \overline{y})
      } \rho \eta \overline{\epsilon}
    )_{\overline{C}} \\
= & \big(
      \setsem{\Gamma; \emptyset \,|\, x, \overline{y} \vdash
    (\map_{H}^{\overline{\tau [\overline{\psi := F}] [\phi := H]}, \overline{\tau [\overline{\psi := G}] [\phi := K]}})_{\emptyset}
    (\overline{(\map_{\tau}^{\overline{I}, \overline{J}})_{\emptyset} (x, \overline{y})})}
    \rho  \eta \overline{\epsilon}
  \big)_{\overline{C}}\\
= & \big(
  \setsem{\Gamma; \emptyset \,|\, \emptyset \vdash\map_{H}^{\overline{\tau [\overline{\psi := F}] [\phi := H]}, \overline{\tau [\overline{\psi := G}] [\phi := K]}}} \rho \ast
  \big(
    \overline{
      \setsem{\Gamma; \emptyset \,|\, x, \overline{y} \vdash (\map_{\tau}^{\overline{I}, \overline{J}})_{\emptyset} (x, \overline{y})}
      \rho \eta \overline{\epsilon}
    }
  \big)
  \big)_{\overline{C}} \\
= & \setsem{\Gamma; \overline{\beta}, \overline{\gamma} \vdash H} \Id_{\rho [\overline{\gamma := C}]} [\overline{
    \beta := (
      \setsem{\Gamma; \emptyset \,|\, x, \overline{y} \vdash (\map_{\tau}^{\overline{I}, \overline{J}})_{\emptyset} (x, \overline{y})}
      \rho \eta \overline{\epsilon}
    )_{\overline{C}}
  }] \\
= & \setsem{\Gamma; \overline{\beta}, \overline{\gamma} \vdash H} \Id_{\rho [\overline{\gamma := C}]} [\overline{
    \beta := \setsem{\Gamma; \phi, \overline{\psi}, \overline{\gamma} \vdash \tau}
      \Id_{\rho [\overline{\gamma := C}]}
      [\overline{\psi := \lambda \overline{A}. \epsilon_{\overline{A}, \overline{C}}}]
      [\phi := \lambda \overline{B}. \eta_{\overline{B}, \overline{C}}]
    }]
\end{array}
\]

Observe that, for each $\tau$,
\[
%\eta_{\overline{
\setsem{\Gamma; \overline{\gamma} \vdash \tau [\overline{\psi := G}] [\phi := K]} \rho [\overline{\gamma := C}]
%}, \overline{C}}
\]
is equal to
\[
%\eta_{
%  \overline{
    \setsem{\Gamma; \phi, \overline{\psi}, \overline{\gamma} \vdash \tau} \rho
    [\overline{\gamma := C}]
    [\overline{\psi := \setsem{\Gamma; \overline{\alpha}, \overline{\gamma} \vdash G} \rho [\overline{\gamma := C}][\overline{\alpha := \_}] }]
    [\phi := \setsem{\Gamma; \overline{\beta}, \overline{\gamma} \vdash K} \rho [\overline{\gamma := C}][\overline{\beta := \_}] ]
%  },
%  \overline{C}
% }
\]
because of Lemma~\ref{lem:substitution}  (references lemma in draft document).
Moreover, observe that $\lambda \ol{B}. \eta_{\ol{B}, \ol{C}}$ is a natural transformation
\[
\lambda \ol{B}. \setsem{\Gamma; \overline{\beta}, \overline{\gamma} \vdash H} \rho [\ol{\beta := B}] [\ol{\gamma := C}]
\Rightarrow
\lambda \ol{B}. \setsem{\Gamma; \overline{\beta}, \overline{\gamma} \vdash K} \rho [\ol{\beta := B}] [\ol{\gamma := C}]
\]
and $\lambda \ol{A}. \epsilon_{\ol{A}, \ol{C}}$ is a natural transformation
\[
\lambda \ol{A}. \setsem{\Gamma; \overline{\alpha}, \overline{\gamma} \vdash F} \rho [\ol{\alpha := A}] [\ol{\gamma := C}]
\Rightarrow
\lambda \ol{A}. \setsem{\Gamma; \overline{\alpha}, \overline{\gamma} \vdash G} \rho [\ol{\alpha := A}] [\ol{\gamma := C}]
\]
for any $\ol{C : \set}$.
Then we have that
\[
\begin{array}{rl}
  & \setsem{\Gamma; \emptyset \,|\, \emptyset \vdash L_{\emptyset} (x, \overline{y}). L_{\overline{\gamma}} z.
      x_{\overline{\tau [\overline{\psi := G}] [\phi := K]}, \overline{\gamma}}
      \big(\big(
        (\map_{H[\overline{\beta := \tau}]}^{\overline{I}, \overline{J}})_{\emptyset} (x, \overline{y})
      \big)_{\overline{\gamma}} z \big)}\rho \ast \\
= & \lambda \eta. \lambda \overline{\epsilon}. \lambda \overline{C}.
    \eta_{\overline{\setsem{\Gamma; \overline{\gamma} \vdash \tau [\overline{\psi := G}] [\phi := K]} \rho [\overline{\gamma := C}]}, \overline{C}}
    \circ
    \big(
      \setsem{\Gamma; \emptyset \,|\, \emptyset \vdash
        \map_{H[\overline{\beta := \tau}]}^{\overline{I}, \overline{J}}
      } \rho \ast \eta \overline{\epsilon}
    \big)_{\overline{C}} \\
= & \lambda \eta. \lambda \overline{\epsilon}. \lambda \overline{C}.
  \eta_{
  \overline{
    \setsem{\Gamma; \phi, \overline{\psi}, \overline{\gamma} \vdash \tau} \rho
    [\overline{\gamma := C}]
    [\overline{\psi := \setsem{\Gamma; \overline{\alpha}, \overline{\gamma} \vdash G} \rho [\overline{\gamma := C}][\overline{\alpha := \_}] }]
    [\phi := \setsem{\Gamma; \overline{\beta}, \overline{\gamma} \vdash K} \rho [\overline{\gamma := C}][\overline{\beta := \_}] ]
  },
  \overline{C}
  } \\
  & \circ \setsem{\Gamma; \overline{\beta}, \overline{\gamma} \vdash H} \Id_{\rho [\overline{\gamma := C}]} [\overline{
    \beta := \setsem{\Gamma; \phi, \overline{\psi}, \overline{\gamma} \vdash \tau}
      \Id_{\rho [\overline{\gamma := C}]}
      [\overline{\psi := \lambda \overline{A}. \epsilon_{\overline{A}, \overline{C}}}]
      [\phi := \lambda \overline{B}. \eta_{\overline{B}, \overline{C}}]
    }] \\
= & \lambda \eta. \lambda \overline{\epsilon}. \lambda \overline{C}.
  \eta_{
  \overline{
    \setsem{\Gamma; \phi, \overline{\psi}, \overline{\gamma} \vdash \tau} \rho
    [\overline{\gamma := C}]
    [\overline{\psi := \setsem{\Gamma; \overline{\alpha}, \overline{\gamma} \vdash G} \rho [\overline{\gamma := C}][\overline{\alpha := \_}] }]
    [\phi := \setsem{\Gamma; \overline{\beta}, \overline{\gamma} \vdash K} \rho [\overline{\gamma := C}][\overline{\beta := \_}] ]
  },
  \overline{C}
  } \\
  & \circ (\lambda \ol{B}. \setsem{\Gamma; \overline{\beta}, \overline{\gamma} \vdash H} \rho [\overline{\gamma := C}] [\ol{\beta := B}])
    (\overline{ \setsem{\Gamma; \phi, \overline{\psi}, \overline{\gamma} \vdash \tau}
      \Id_{\rho [\overline{\gamma := C}]}
      [\overline{\psi := \lambda \overline{A}. \epsilon_{\overline{A}, \overline{C}}}]
      [\phi := \lambda \overline{B}. \eta_{\overline{B}, \overline{C}}]
    }) \\
= & \lambda \eta. \lambda \overline{\epsilon}. \lambda \overline{C}.
      \setsem{\Gamma; \phi, \overline{\psi}, \overline{\gamma} \vdash \phi \overline{\tau}}
      \Id_{\rho [\overline{\gamma := C}]}
      [\phi := \lambda \overline{B}. \eta_{\overline{B}, \overline{C}}]
      [\overline{\psi := \lambda \overline{A}. \epsilon_{\overline{A}, \overline{C}}}] \\
= & \lambda \eta. \lambda \overline{\epsilon}. \lambda \overline{C}. \big(
      \setsem{\Gamma; \emptyset \,|\, \emptyset \vdash \map^{\overline{I}, \overline{J}}_{\phi \overline{\tau}}} \rho \ast \eta \overline{\epsilon}
    \big)_{\overline{C}} \\
= & \setsem{\Gamma; \emptyset \,|\, \emptyset \vdash \map^{\overline{I}, \overline{J}}_{\phi \overline{\tau}}} \rho \ast
\end{array}
\]
where the third equality is given by the definition of functorial
action of the semantic interpretation for type application, in
Definition~\ref{set-sem-funcs} (the definition of the action of set
interpretations of types on morphisms in $\setenv$).

Finally, the proof for the relation interpretation is analogous to the above proof for the set interpretation.
\end{proof}

\begin{lemma}\label{lemma:in-fold-map-I}
The terms
\[
\Gamma; \emptyset \,|\, x: \Nat^{\overline{\beta}, \overline{\gamma}} H[\phi := F][\overline{\alpha := \beta}]\, F
\vdash ((\fold_{H, F})_{\emptyset} x) \circ \tin_{H}
: \Nat^{\overline{\beta}, \overline{\gamma}} H[\phi := (\mu \phi. \lambda \overline{\alpha}. H)\overline{\beta}][\overline{\alpha := \beta}]\, F
\]
and
\begin{align*}
\Gamma; \emptyset \,|\, x: \Nat^{\overline{\beta}, \overline{\gamma}} H[\phi := F][\overline{\alpha := \beta}]\, F
&\vdash x \circ \big( (\map_{H [\ol{\alpha := \beta}]}^{(\mu \phi. \lambda \overline{\alpha} H) \overline{\beta}, F})_{\emptyset} ((\fold_{H, F})_{\emptyset} x) \big) \\
&: \Nat^{\overline{\beta}, \overline{\gamma}} H[\phi := (\mu \phi. \lambda \overline{\alpha}. H)\overline{\beta}][\overline{\alpha := \beta}]\, F
\end{align*}
are semantically equivalent.
\end{lemma}
\begin{proof}
Let $\rho$ be a set environment, $\ol{B}$ and $\ol{C}$ be sets, and $\eta$ be a natural transformation in
$\setsem{\Gamma; \emptyset \vdash \Nat^{\overline{\beta}, \overline{\gamma}} H[\phi := F][\overline{\alpha := \beta}]\, F}\rho$.
Then we have that
\[
\begin{array}{rl}
  & (\setsem{\Gamma; \emptyset \,|\, x \vdash ((\fold_{H, F})_{\emptyset} x) \circ \tin_{H}} \rho \eta)_{\overline{B}, \overline{C}} \\
= & (\setsem{\Gamma; \emptyset \,|\, x \vdash (\fold_{H, F})_{\emptyset} x} \rho \eta)_{\overline{B}, \overline{C}}
    \circ (\setsem{\Gamma; \emptyset \,|\, \emptyset \vdash \tin_{H}} \rho \ast)_{\overline{B}, \overline{C}} \\
= & (\setsem{\Gamma; \emptyset \,|\, \emptyset \vdash \fold_{H, F}} \rho \ast \eta)_{\overline{B}, \overline{C}}
    \circ (\textit{in}_{T_{\rho [\overline{\gamma := C}]}})_{\overline{B}} \\
= & (\textit{fold}_{
      T_{\rho [\overline{\gamma := C}]}
      % \setsem{\Gamma; \overline{\alpha}, \overline{\gamma} \vdash F} \rho [\overline{\alpha := \_}] [\overline{\gamma := C}]
    } (\lambda \overline{A}. \eta_{\overline{A}, \overline{C}}) )_{\overline{B}}
    \circ (\textit{in}_{T_{\rho [\overline{\gamma := C}]}})_{\overline{B}} \\
= & ((\textit{fold}_{
      T_{\rho [\overline{\gamma := C}]}
      %\setsem{\Gamma; \overline{\alpha}, \overline{\gamma} \vdash F} \rho [\overline{\alpha := \_}] [\overline{\gamma := C}]
    } (\lambda \overline{A}. \eta_{\overline{A}, \overline{C}}))
    \circ \textit{in}_{T_{\rho [\overline{\gamma := C}]}})_{\overline{B}} \\
= & \big(
      (\lambda \overline{A}. \eta_{\overline{A}, \overline{C}})
      \circ (T_{\rho [\overline{\gamma := C}]} (\textit{fold}_{T_{\rho [\overline{\gamma := C}]}} (\lambda \overline{A}. \eta_{\overline{A}, \overline{C}})))
    \big)_{\overline{B}} \\
= & (\lambda \overline{A}. \eta_{\overline{A}, \overline{C}})_{\overline{B}}
      \circ
    \big(
      T_{\rho [\overline{\gamma := C}]} (\textit{fold}_{T_{\rho [\overline{\gamma := C}]}} (\lambda \overline{A}. \eta_{\overline{A}, \overline{C}}))
    \big)_{\overline{B}} \\
= & \eta_{\overline{B}, \overline{C}}
      \circ
    \setsem{\Gamma; \phi, \overline{\alpha}, \overline{\gamma} \vdash H}
      Id_{\rho [\overline{\alpha := B}] [\overline{\gamma := C}]}
      [\phi := \textit{fold}_{T_{\rho [\overline{\gamma := C}]}} (\lambda \overline{A}. \eta_{\overline{A}, \overline{C}})] \\
= & \eta_{\overline{B}, \overline{C}}
    \circ \setsem{\Gamma; \phi, \overline{\alpha}, \overline{\gamma} \vdash H}
      Id_{\rho [\overline{\alpha := B}] [\overline{\gamma := C}]}
      [\phi := \lambda \overline{A'}. (\textit{fold}_{T_{\rho [\overline{\gamma := C}]}} (\lambda \overline{A}. \eta_{\overline{A}, \overline{C}}))_{\overline{A'}}] \\
= & \eta_{\overline{B}, \overline{C}}
    \circ \setsem{\Gamma; \phi, \overline{\alpha}, \overline{\gamma} \vdash H}
      Id_{\rho [\overline{\alpha := B}] [\overline{\gamma := C}]}
      [\phi := \lambda \overline{A'}. (\setsem{\Gamma; \emptyset \,|\, \emptyset \vdash \fold_{H}^{F}} \rho \ast \eta)_{\overline{A'}, \overline{C}}] \\
= & \eta_{\overline{B}, \overline{C}}
    \circ \setsem{\Gamma; \phi, \overline{\alpha}, \overline{\gamma} \vdash H}
      Id_{\rho [\overline{\alpha := B}] [\overline{\gamma := C}]}
      [\phi := \lambda \overline{A'}. (\setsem{\Gamma; \emptyset \,|\, \emptyset \vdash \fold_{H}^{F}} \rho \ast \eta)_{\overline{A'}, \ol{B}, \overline{C}}] \\
= & \eta_{\overline{B}, \overline{C}}
    \circ \big(
      \setsem{\Gamma; \emptyset \,|\, \emptyset \vdash \map_{H}^{(\mu \phi. \lambda \overline{\alpha}. H) \overline{\beta}, F}} \rho \ast
      (\setsem{\Gamma; \emptyset \,|\, \emptyset \vdash \fold_{H}^{F}} \rho \ast \eta)
    \big)_{\overline{B}, \overline{C}} \\
= & \eta_{\overline{B}, \overline{C}}
    \circ \big(
      \setsem{\Gamma; \emptyset \,|\, \emptyset \vdash \map_{H}^{(\mu \phi. \lambda \overline{\alpha}. H) \overline{\beta}, F}} \rho \ast
      (\setsem{\Gamma; \emptyset \,|\, x \vdash (\fold_{H}^{F})_{\emptyset} x} \rho \eta)
    \big)_{\overline{B}, \overline{C}} \\
= & (\setsem{\Gamma; \emptyset \,|\, x \vdash x} \rho \eta)_{\overline{B}, \overline{C}}
    \circ (\setsem{\Gamma; \emptyset \,|\, x \vdash (\map_{H}^{(\mu \phi. \lambda \overline{\alpha}. H) \overline{\beta}, F})_{\emptyset} ((\fold_{H}^{F})_{\emptyset} x) } \rho \eta)_{\overline{B}, \overline{C}} \\
= & (\setsem{\Gamma; \emptyset \,|\, x \vdash x \circ \big( (\map_{H}^{(\mu \phi. \lambda \overline{\alpha}. H) \overline{\beta}, F})_{\emptyset} ((\fold_{H}^{F})_{\emptyset} x) \big)} \rho \eta)_{\overline{B}, \overline{C}}
\end{array}
\]
where the tenth equality is given by weakening.
\end{proof}

It is a general property of the semantic $\textit{in}$ to be invertible,
its inverse being given in terms of $\textit{fold}$.
We shall prove that this fact holds for the semantic interpretation of the syntactic $\tin$ and $\fold$.

The next lemma states the syntactic analogue of the fact that, for an endofunctor $H$,
the composition $\textit{in}_H \circ \textit{fold}_H(H \textit{in}_H)$ is the identity on the fixed point $\mu H$.

\begin{lemma}\label{lemma:in-fold-map-II}
The terms
\begin{align*}
\Gamma; \emptyset \,|\, \emptyset
&\vdash \tin_H 
\circ (\fold_{H, H[\phi := (\mu \phi. \lambda \overline{\alpha}. H)\overline{\beta}]})_{\emptyset}
((\map_{H}^{H[\phi := (\mu \phi. \lambda \overline{\alpha}. H)\overline{\beta}][\overline{\alpha := \beta}], (\mu \phi. \lambda \overline{\alpha}. H)\overline{\beta}})_{\emptyset} \tin_H) \\
&: \Nat^{\overline{\beta}, \overline{\gamma}} (\mu \phi. \lambda \overline{\alpha}. H)\overline{\beta}\, (\mu \phi. \lambda \overline{\alpha}. H)\overline{\beta}
\end{align*}
and
\[
\Gamma; \emptyset \,|\, \emptyset
\vdash \Id_{(\mu \phi. \lambda \overline{\alpha}. H)\overline{\beta}}
: \Nat^{\overline{\beta}, \overline{\gamma}} (\mu \phi. \lambda \overline{\alpha}. H)\overline{\beta}\, (\mu \phi. \lambda \overline{\alpha}. H)\overline{\beta}
\]
are semantically equivalent.
\end{lemma}
\begin{proof}
Let $\rho$ be a set environment, $\ol{B}$ and $\ol{C}$ be sets, and $\ast$ be the unique element of the singleton.
Then we have that
\[
\begin{array}{rl}
  &(\setsem{\Gamma; \emptyset \,|\, \emptyset
\vdash \tin_H 
\circ (\fold_{H, H[\phi := (\mu \phi. \lambda \overline{\alpha}. H)\overline{\beta}]})_{\emptyset}
((\map_{H}^{H[\phi := (\mu \phi. \lambda \overline{\alpha}. H)\overline{\beta}][\overline{\alpha := \beta}], (\mu \phi. \lambda \overline{\alpha}. H)\overline{\beta}})_{\emptyset} \tin_H)}\rho \ast)_{\ol{B}, \ol{C}} \\
= &(\setsem{\Gamma; \emptyset \,|\, \emptyset \vdash \tin_H}\rho \ast)_{\ol{B}, \ol{C}} \\
  &\circ (\setsem{\Gamma; \emptyset \,|\, \emptyset
	\vdash(\fold_{H, H[\phi := (\mu \phi. \lambda \overline{\alpha}. H)\overline{\beta}]})_{\emptyset}
	((\map_{H}^{H[\phi := (\mu \phi. \lambda \overline{\alpha}. H)\overline{\beta}][\overline{\alpha := \beta}], (\mu \phi. \lambda \overline{\alpha}. H)\overline{\beta}})_{\emptyset} \tin_H)}\rho \ast)_{\ol{B}, \ol{C}} \\
= &( \textit{in}_{T^{\set}_{\rho [\ol{\gamma := C}]}} )_{\ol{B}} \\
  &\circ \big( \setsem{\Gamma; \emptyset \,|\, \emptyset \vdash \fold_{H, H[\phi := (\mu \phi. \lambda \overline{\alpha}. H)\overline{\beta}]} } \rho \ast \\
  &(\setsem{\Gamma; \emptyset \,|\, \emptyset \vdash \map_{H}^{H[\phi := (\mu \phi. \lambda \overline{\alpha}. H)\overline{\beta}][\overline{\alpha := \beta}], (\mu \phi. \lambda \overline{\alpha}. H)\overline{\beta}}} \rho \ast
	(\setsem{\Gamma; \emptyset \,|\, \emptyset \vdash  \tin_H} \rho \ast)) \big)_{\ol{B}, \ol{C}} \\
= &( \textit{in}_{T^{\set}_{\rho [\ol{\gamma := C}]}} )_{\ol{B}} \\
  &\circ \big( \textit{fold}_{T^{\set}_{\rho [\ol{\gamma := C}]}} \lambda \ol{A}.
	(\setsem{\Gamma; \emptyset \,|\, \emptyset \vdash \map_{H}^{H[\phi := (\mu \phi. \lambda \overline{\alpha}. H)\overline{\beta}][\overline{\alpha := \beta}], (\mu \phi. \lambda \overline{\alpha}. H)\overline{\beta}}} \rho \ast
		(\setsem{\Gamma; \emptyset \,|\, \emptyset \vdash  \tin_H} \rho \ast))_{\ol{A}, \ol{C}}
	\big)_{\ol{B}}
\end{array}
\]
Notice that
\[
\begin{array}{rl}
  &\lambda \ol{A}.
	(\setsem{\Gamma; \emptyset \,|\, \emptyset \vdash \map_{H}^{H[\phi := (\mu \phi. \lambda \overline{\alpha}. H)\overline{\beta}][\overline{\alpha := \beta}], (\mu \phi. \lambda \overline{\alpha}. H)\overline{\beta}}} \rho \ast
		(\setsem{\Gamma; \emptyset \,|\, \emptyset \vdash  \tin_H} \rho \ast))_{\ol{A}, \ol{C}} \\
= &\lambda \ol{A}.
		\setsem{\Gamma; \phi, \ol{\alpha}, \ol{\gamma} \vdash H} \Id_{\rho [\ol{\alpha := A}][\ol{\gamma := C}]}
		[\phi := \lambda \ol{B'}. (\setsem{\Gamma; \emptyset \,|\, \emptyset \vdash  \tin_H} \rho \ast)_{\ol{A}, \ol{B'}, \ol{C}}] \\
= &\lambda \ol{A}.
		\setsem{\Gamma; \phi, \ol{\alpha}, \ol{\gamma} \vdash H} \Id_{\rho [\ol{\alpha := A}][\ol{\gamma := C}]}
		[\phi := \lambda \ol{B'}. (\setsem{\Gamma; \emptyset \,|\, \emptyset \vdash  \tin_H} \rho \ast)_{\ol{B'}, \ol{C}}] \\
= &\lambda \ol{A}.
		\setsem{\Gamma; \phi, \ol{\alpha}, \ol{\gamma} \vdash H} \Id_{\rho [\ol{\alpha := A}][\ol{\gamma := C}]}
		[\phi := \lambda \ol{B'}. ( \textit{in}_{T^{\set}_{\rho [\ol{\gamma := C}]}} )_{\ol{B'}}] \\
= &\lambda \ol{A}.
		\setsem{\Gamma; \phi, \ol{\alpha}, \ol{\gamma} \vdash H} \Id_{\rho [\ol{\alpha := A}][\ol{\gamma := C}]}
		[\phi := \textit{in}_{T^{\set}_{\rho [\ol{\gamma := C}]}}] \\
= &\lambda \ol{A}.
	T^{\set}_{\rho [\ol{\gamma := C}]} \textit{in}_{T^{\set}_{\rho [\ol{\gamma := C}]}} \ol{A} \\
= &T^{\set}_{\rho [\ol{\gamma := C}]} \textit{in}_{T^{\set}_{\rho [\ol{\gamma := C}]}}
\end{array}
\]
Then we use the above calculation to compute
\[
\begin{array}{rl}
  &( \textit{in}_{T^{\set}_{\rho [\ol{\gamma := C}]}} )_{\ol{B}} \\
  &\circ \big( \textit{fold}_{T^{\set}_{\rho [\ol{\gamma := C}]}} ( \lambda \ol{A}.
	(\setsem{\Gamma; \emptyset \,|\, \emptyset \vdash \map_{H}^{H[\phi := (\mu \phi. \lambda \overline{\alpha}. H)\overline{\beta}][\overline{\alpha := \beta}], (\mu \phi. \lambda \overline{\alpha}. H)\overline{\beta}}} \rho \ast
		(\setsem{\Gamma; \emptyset \,|\, \emptyset \vdash  \tin_H} \rho \ast))_{\ol{A}, \ol{C}}
	) \big)_{\ol{B}} \\
= &( \textit{in}_{T^{\set}_{\rho [\ol{\gamma := C}]}} )_{\ol{B}}
	\circ \big( \textit{fold}_{T^{\set}_{\rho [\ol{\gamma := C}]}} (
		T^{\set}_{\rho [\ol{\gamma := C}]} \textit{in}_{T^{\set}_{\rho [\ol{\gamma := C}]}}
	) \big)_{\ol{B}} \\
= &\big( \textit{in}_{T^{\set}_{\rho [\ol{\gamma := C}]}}
	\circ \textit{fold}_{T^{\set}_{\rho [\ol{\gamma := C}]}} (
		T^{\set}_{\rho [\ol{\gamma := C}]} \textit{in}_{T^{\set}_{\rho [\ol{\gamma := C}]}}
	) \big)_{\ol{B}} \\
= &( \Id_{\mu T^{\set}_{\rho [\ol{\gamma := C}]}} )_{\ol{B}} \\
= &\Id_{( \mu T^{\set}_{\rho [\ol{\gamma := C}]} )_{\ol{B}}} \\
= &\Id_{\setsem{\Gamma; \ol{\beta}, \ol{\gamma} \vdash (\mu \phi. \lambda \ol{\alpha}. H)\ol{\beta} }\rho [\ol{\beta := B}][\ol{\gamma := C}]} \\
= &(\setsem{\Gamma; \emptyset \,|\, \emptyset \vdash \Id_{(\mu \phi. \lambda \overline{\alpha}. H)\overline{\beta}}}\rho \ast)_{\ol{B}, \ol{C}}
\end{array}
\]
where the first equality follows from the previous calculation,
the third equality is a semantic property of $\textit{fold}$ and $\textit{in}$,
and the last equality is by Lemma~\ref{lem:interpretation-identity}.

The proof for the relation interpretation is analogous.
\end{proof}

The next lemma states the syntactic analogue of the fact that, for an endofunctor $H$,
the composition $\textit{fold}_H(H \textit{in}_H) \circ \textit{in}_H$ is the identity on the fixed point $H(\mu H)$.

\begin{lemma}\label{lemma:in-fold-map-III}
The terms
\begin{align*}
\Gamma; \emptyset \,|\, \emptyset
&\vdash (\fold_{H, H[\phi := (\mu \phi. \lambda \overline{\alpha}. H)\overline{\beta}]})_{\emptyset}
((\map_{H}^{H[\phi := (\mu \phi. \lambda \overline{\alpha}. H)\overline{\beta}][\overline{\alpha := \beta}], (\mu \phi. \lambda \overline{\alpha}. H)\overline{\beta}})_{\emptyset} \tin_H)
\circ  \tin_H \\
&: \Nat^{\overline{\beta}, \overline{\gamma}}
H[\phi := (\mu \phi. \lambda \overline{\alpha}. H)\overline{\beta}]\,
H[\phi := (\mu \phi. \lambda \overline{\alpha}. H)\overline{\beta}]
\end{align*}
and
\[
\Gamma; \emptyset \,|\, \emptyset
\vdash \Id_{H[\phi := (\mu \phi. \lambda \overline{\alpha}. H)\overline{\beta}]}
: \Nat^{\overline{\beta}, \overline{\gamma}}
H[\phi := (\mu \phi. \lambda \overline{\alpha}. H)\overline{\beta}]\,
H[\phi := (\mu \phi. \lambda \overline{\alpha}. H)\overline{\beta}]
\]
are semantically equivalent.
\end{lemma}
\begin{proof}
The term
\[
\Gamma; \emptyset \,|\, \emptyset
\vdash (\fold_{H, H[\phi := (\mu \phi. \lambda \overline{\alpha}. H)\overline{\beta}]})_{\emptyset}
((\map_{H}^{H[\phi := (\mu \phi. \lambda \overline{\alpha}. H)\overline{\beta}][\overline{\alpha := \beta}], (\mu \phi. \lambda \overline{\alpha}. H)\overline{\beta}})_{\emptyset} \tin_H)
\circ  \tin_H
\]
because of Lemma~\ref{lemma:in-fold-map-I}, is semantically equivalent to
\begin{multline*}
\Gamma; \emptyset \,|\, \emptyset
\vdash ((\map_{H}^{H[\phi := (\mu \phi. \lambda \overline{\alpha}. H)\overline{\beta}][\overline{\alpha := \beta}], (\mu \phi. \lambda \overline{\alpha}. H)\overline{\beta}})_{\emptyset} \tin_H) \\
\circ (\map_{H}^{(\mu \phi. \lambda \overline{\alpha}. H)\overline{\beta}, H[\phi := (\mu \phi. \lambda \overline{\alpha}. H)\overline{\beta}][\overline{\alpha := \beta}]})_{\emptyset} \\
((\fold_{H, H[\phi := (\mu \phi. \lambda \overline{\alpha}. H)\overline{\beta}]})_{\emptyset}
((\map_{H}^{H[\phi := (\mu \phi. \lambda \overline{\alpha}. H)\overline{\beta}][\overline{\alpha := \beta}], (\mu \phi. \lambda \overline{\alpha}. H)\overline{\beta}})_{\emptyset} \tin_H))
\end{multline*}
which, because of functoriality (map notes, pages 1--2), is semantically equivalent to
\begin{multline*}
\Gamma; \emptyset \,|\, \emptyset
\vdash 
(\map_{H}^{(\mu \phi. \lambda \overline{\alpha}. H)\overline{\beta}, (\mu \phi. \lambda \overline{\alpha}. H) \overline{\beta}})_{\emptyset} \\
(
\tin_H
\circ (\fold_{H, H[\phi := (\mu \phi. \lambda \overline{\alpha}. H)\overline{\beta}]})_{\emptyset}
((\map_{H}^{H[\phi := (\mu \phi. \lambda \overline{\alpha}. H)\overline{\beta}][\overline{\alpha := \beta}], (\mu \phi. \lambda \overline{\alpha}. H)\overline{\beta}})_{\emptyset} \tin_H))
\end{multline*}
which, because of Lemma~\ref{lemma:in-fold-map-II}, is semantically equivalent to
\[
\Gamma; \emptyset \,|\, \emptyset
\vdash 
(\map_{H}^{(\mu \phi. \lambda \overline{\alpha}. H)\overline{\beta}, (\mu \phi. \lambda \overline{\alpha}. H) \overline{\beta}})_{\emptyset}
(\Id_{(\mu \phi. \lambda \overline{\alpha}. H) \overline{\beta}})
\]
which, because of functoriality (map notes, page 3), is semantically equivalent to
\[
\Gamma; \emptyset \,|\, \emptyset
\vdash 
\Id_{H[\phi := (\mu \phi. \lambda \overline{\alpha}. H)\overline{\beta}]} \qedhere
\]
\end{proof}

\bibliography{references}

\end{document}

