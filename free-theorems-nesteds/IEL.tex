% For double-blind review submission, w/o CCS and ACM Reference (max
% submission space)
\documentclass[acmsmall,review,anonymous]{acmart}
\settopmatter{printfolios=true,printccs=false,printacmref=false}
%% For double-blind review submission, w/ CCS and ACM Reference
%\documentclass[acmsmall,review,anonymous]{acmart}\settopmatter{printfolios=true}
%% For single-blind review submission, w/o CCS and ACM Reference (max submission space)
%\documentclass[acmsmall,review]{acmart}\settopmatter{printfolios=true,printccs=false,printacmref=false}
%% For single-blind review submission, w/ CCS and ACM Reference
%\documentclass[acmsmall,review]{acmart}\settopmatter{printfolios=true}
%% For final camera-ready submission, w/ required CCS and ACM Reference
%\documentclass[acmsmall]{acmart}\settopmatter{}


%% Journal information
%% Supplied to authors by publisher for camera-ready submission;
%% use defaults for review submission.
\acmJournal{PACMPL}
\acmVolume{1}
\acmNumber{POPL} % CONF = POPL or ICFP or OOPSLA
\acmArticle{1}
\acmYear{2020}
\acmMonth{1}
\acmDOI{} % \acmDOI{10.1145/nnnnnnn.nnnnnnn}
\startPage{1}

%% Copyright information
%% Supplied to authors (based on authors' rights management selection;
%% see authors.acm.org) by publisher for camera-ready submission;
%% use 'none' for review submission.
\setcopyright{none}
%\setcopyright{acmcopyright}
%\setcopyright{acmlicensed}
%\setcopyright{rightsretained}
%\copyrightyear{2018}           %% If different from \acmYear

%% Bibliography style
\bibliographystyle{ACM-Reference-Format}
%% Citation style
%% Note: author/year citations are required for papers published as an
%% issue of PACMPL.
\citestyle{acmauthoryear}   %% For author/year citations
%\citestyle{acmnumeric}

%%%%%%%%%%%%%%%%%%%%%%%%%%%%%%%%%%%%%%%%%%%%%%%%%%%%%%%%%%%%%%%%%%%%%%
%% Note: Authors migrating a paper from PACMPL format to traditional
%% SIGPLAN proceedings format must update the '\documentclass' and
%% topmatter commands above; see 'acmart-sigplanproc-template.tex'.
%%%%%%%%%%%%%%%%%%%%%%%%%%%%%%%%%%%%%%%%%%%%%%%%%%%%%%%%%%%%%%%%%%%%%%



\usepackage[utf8]{inputenc}
\usepackage{ccicons}
\usepackage{verbatim}

\usepackage{amsmath}
\usepackage{amsthm}
\usepackage{amscd}
%\usepackage{MnSymbol}
\usepackage{xcolor}

\usepackage{bbold}
\usepackage{url}
\usepackage{upgreek}
%\usepackage{stmaryrd}

\usepackage{lipsum}
\usepackage{tikz-cd}
\usetikzlibrary{cd}
\usetikzlibrary{calc}
\usetikzlibrary{arrows}

\usepackage{bussproofs}
\EnableBpAbbreviations

\DeclareMathAlphabet{\mathpzc}{OT1}{pzc}{m}{it}

%\usepackage[amsmath]{ntheorem}

\newcommand{\lam}{\lambda}
\newcommand{\eps}{\varepsilon}
\newcommand{\ups}{\upsilon}
\newcommand{\mcB}{\mathcal{B}}
\newcommand{\mcD}{\mathcal{D}}
\newcommand{\mcE}{\mathcal{E}}
\newcommand{\mcF}{\mathcal{F}}
\newcommand{\mcP}{\mathcal{P}}
\newcommand{\mcI}{\mathcal{I}}
\newcommand{\mcJ}{\mathcal{J}}
\newcommand{\mcK}{\mathcal{K}}
\newcommand{\mcL}{\mathcal{L}}
\newcommand{\WW}{\mathcal{W}}

\newcommand{\ex}{\mcE_x}
\newcommand{\ey}{\mcE_y}
\newcommand{\bzero}{\boldsymbol{0}}
\newcommand{\bone}{{\boldsymbol{1}}}
\newcommand{\tB}{{\bone_\mcB}}
\newcommand{\tE}{{\bone_\mcE}}
\newcommand{\bt}{\mathbf{t}}
\newcommand{\bp}{\mathbf{p}}
\newcommand{\bsig}{\mathbf{\Sigma}}
\newcommand{\bpi}{\boldsymbol{\pi}}
\newcommand{\Empty}{\mathtt{Empty}}
\newcommand{\truthf}{\mathtt{t}}
\newcommand{\id}{id}
\newcommand{\coo}{\mathtt{coo\ }}
\newcommand{\mcC}{\mathcal{C}}
\newcommand{\Rec}{\mathpzc{Rec}}
\newcommand{\types}{\mathcal{T}}



%\newcommand{\semof}[1]{\llbracket{#1}\rrbracket^\rel}
\newcommand{\sem}[1]{\llbracket{#1}\rrbracket}
\newcommand{\setsem}[1]{\llbracket{#1}\rrbracket^\set}
\newcommand{\relsem}[1]{\llbracket{#1}\rrbracket^\rel}
\newcommand{\dsem}[1]{\llbracket{#1}\rrbracket^{\mathsf D}}
\newcommand{\setenv}{\mathsf{SetEnv}}
\newcommand{\relenv}{\mathsf{RelEnv}}
\newcommand{\denv}{\mathsf{DEnv}}

\newcommand{\rel}{\mathsf{Rel}}
\newcommand{\setof}[1]{\{#1\}}
\newcommand{\letin}[1]{\texttt{let }#1\texttt{ in }}
\newcommand{\comp}[1]{{\{#1\}}}
\newcommand{\bcomp}[1]{\{\![#1]\!\}}
\newcommand{\beps}{\boldsymbol{\eps}}
%\newcommand{\B}{\mcB}
%\newcommand{\Bo}{{|\mcB|}}

\newcommand{\lmt}{\longmapsto}
\newcommand{\RA}{\Rightarrow}
\newcommand{\LA}{\Leftarrow}
\newcommand{\rras}{\rightrightarrows}
\newcommand{\colim}[2]{{{\underrightarrow{\lim}}_{#1}{#2}}}
\newcommand{\lift}[1]{{#1}\,{\hat{} \; \hat{}}}
\newcommand{\graph}[1]{\langle {#1} \rangle}

\newcommand{\carAT}{\mathsf{car}({\mathcal A}^T)}
\newcommand{\isoAto}{\mathsf{Iso}({\mcA^\to})}
\newcommand{\falg}{\mathsf{Alg}_F}
\newcommand{\CC}{\mathsf{Pres}(\mathcal{A})}
\newcommand{\PP}{\mathcal{P}}
\newcommand{\DD}{D_{(A,B,f)}}
\newcommand{\from}{\leftarrow}
\newcommand{\upset}[1]{{#1}{\uparrow}}
\newcommand{\smupset}[1]{{#1}\!\uparrow}

\newcommand{\Coo}{\mathpzc{Coo}}
\newcommand{\code}{\#}
\newcommand{\nat}{\mathpzc{Nat}}

\newcommand{\eq}{\; = \;}
\newcommand{\of}{\; : \;}
\newcommand{\df}{\; := \;}
\newcommand{\bnf}{\; ::= \;}

\newcommand{\zmap}[1]{{\!{\between\!\!}_{#1}\!}}
\newcommand{\bSet}{\mathbf{Set}}

\newcommand{\dom}{\mathsf{dom}}
\newcommand{\cod}{\mathsf{cod}}
\newcommand{\adjoint}[2]{\mathrel{\mathop{\leftrightarrows}^{#1}_{#2}}}
\newcommand{\isofunc}{\mathpzc{Iso}}
\newcommand{\ebang}{{\eta_!}}
\newcommand{\lras}{\leftrightarrows}
\newcommand{\rlas}{\rightleftarrows}
\newcommand{\then}{\quad\Longrightarrow\quad}
\newcommand{\hookup}{\hookrightarrow}

\newcommand{\spanme}[5]{\begin{CD} #1 @<#2<< #3 @>#4>> #5 \end{CD}}
\newcommand{\spanm}[3]{\begin{CD} #1 @>#2>> #3\end{CD}}
\newcommand{\pushout}{\textsf{Pushout}}
\newcommand{\mospace}{\qquad\qquad\!\!\!\!}

\newcommand{\natur}[2]{#1 \propto #2}

\newcommand{\Tree}{\mathsf{Tree}\,}
\newcommand{\GRose}{\mathsf{GRose}\,}
\newcommand{\List}{\mathsf{List}\,}
\newcommand{\PTree}{\mathsf{PTree}\,}
\newcommand{\Bush}{\mathsf{Bush}\,}
\newcommand{\Forest}{\mathsf{Forest}\,}
\newcommand{\Lam}{\mathsf{Lam}\,}
\newcommand{\LamES}{\mathsf{Lam}^+}
\newcommand{\Expr}{\mathsf{Expr}\,}

\newcommand{\ListNil}{\mathsf{Nil}}
\newcommand{\ListCons}{\mathsf{Cons}}
\newcommand{\LamVar}{\mathsf{Var}}
\newcommand{\LamApp}{\mathsf{App}}
\newcommand{\LamAbs}{\mathsf{Abs}}
\newcommand{\LamSub}{\mathsf{Sub}}
\newcommand{\ExprConst}{\mathsf{Const}}
\newcommand{\ExprPair}{\mathsf{Pair}}
\newcommand{\ExprProj}{\mathsf{Proj}}
\newcommand{\ExprAbs}{\mathsf{Abs}}
\newcommand{\ExprApp}{\mathsf{App}}
\newcommand{\Ptree}{\mathsf{Ptree}}

\newcommand{\kinds}{\mathpzc{K}}
\newcommand{\tvars}{\mathbb{T}}
\newcommand{\fvars}{\mathbb{F}}
\newcommand{\consts}{\mathpzc{C}}
\newcommand{\Lan}{\mathsf{Lan}}
\newcommand{\zerot}{\mathbb{0}}
\newcommand{\onet}{\mathbb{1}}
\newcommand{\bool}{\mathbb{2}}
\renewcommand{\nat}{\mathbb{N}}
%\newcommand{\semof}[1]{[\![#1]\!]}
%\newcommand{\setsem}[1]{\llbracket{#1}\rrbracket^\set}
\newcommand{\predsem}[1]{\llbracket{#1}\rrbracket^\pred}
%\newcommand{\todot}{\stackrel{.}{\to}}
\newcommand{\todot}{\Rightarrow}
\newcommand{\bphi}{{\bm \phi}}

\newcommand{\bm}[1]{\boldsymbol{#1}}

\newcommand{\cL}{\mathcal{L}}
\newcommand{\T}{\mathcal{T}}
\newcommand{\Pos}{P\!}
%\newcommand{\Pos}{\mathcal{P}\!}
\newcommand{\Neg}{\mathcal{N}}
\newcommand{\Hf}{\mathcal{H}}
\newcommand{\V}{\mathbb{V}}
\newcommand{\I}{\mathcal{I}}
\newcommand{\Set}{\mathsf{Set}}
\newcommand{\Nat}{\mathsf{Nat}}
\newcommand{\Homrel}{\mathsf{Hom_{Rel}}}
\newcommand{\CV}{\mathcal{CV}}
\newcommand{\lan}{\mathsf{Lan}}
\newcommand{\Id}{\mathit{Id}}
\newcommand{\mcA}{\mathcal{A}}
\newcommand{\inl}{\mathsf{inl}}
\newcommand{\inr}{\mathsf{inr}}
\newcommand{\case}[3]{\mathsf{case}\,{#1}\,\mathsf{of}\,\{{#2};\,{#3}\}}
\newcommand{\tin}{\mathsf{in}}
\newcommand{\fold}{\mathsf{fold}}
\newcommand{\Eq}{\mathsf{Eq}}
\newcommand{\Hom}{\mathsf{Hom}}
\newcommand{\curry}{\mathsf{curry}}
\newcommand{\uncurry}{\mathsf{uncurry}}
\newcommand{\eval}{\mathsf{eval}}
\newcommand{\apply}{\mathsf{apply}}

\newcommand{\ar}[1]{\##1}
\newcommand{\mcG}{\mathcal{G}}
\newcommand{\mcH}{\mathcal{H}}
\newcommand{\TV}{\mathpzc{V}}

\newcommand{\essim}[1]{\mathsf{EssIm}(#1)}
\newcommand{\hra}{\hookrightarrow}

\newcommand{\ol}[1]{\overline{#1}}
\newcommand{\ul}[1]{\underline{#1}}
\newcommand{\op}{\mathsf{op}}
\newcommand{\trige}{\trianglerighteq}
\newcommand{\trile}{\trianglelefteq}
\newcommand{\LFP}{\mathsf{LFP}}
\newcommand{\LAN}{\mathsf{Lan}}
%\newcommand{\Mu}{{\mu\hskip-4pt\mu}}
\newcommand{\Mu}{{\mu\hskip-5.5pt\mu}}
%\newcommand{\Mu}{\boldsymbol{\upmu}}
\newcommand{\Terms}{\mathpzc{Terms}}
\newcommand{\Ord}{\mathpzc{Ord}}
\newcommand{\Anote}[1]{{\color{blue} {#1}}}
\newcommand{\Pnote}[1]{{\color{red} {#1}}}

\newcommand{\greyout}[1]{{\color{gray} {#1}}}
\newcommand{\ora}[1]{\overrightarrow{#1}}

%\newcommand{\?}{{.\ }}
%\theoremheaderfont{\scshape}
%\theorembodyfont{\normalfont}
%\theoremseparator{.\ \ }
\newtheorem{thm}{Theorem}
\newtheorem{dfn}[thm]{Definition}
\newtheorem{prop}[thm]{Proposition}
\newtheorem{cor}[thm]{Corollary}
\newtheorem{lemma}[thm]{Lemma}
\newtheorem{rmk}[thm]{Remark}
\newtheorem{expl}[thm]{Example}
\newtheorem{notn}[thm]{Notation}
%\theoremstyle{nonumberplain}
%\theoremsymbol{\Box}


\theoremstyle{definition}
\newtheorem{exmpl}{Example}

\renewcommand{\greyout}[1]{} %{{\color{gray} {#1}}} -- toggle to remove greyed text

\newcommand{\emptyfun}{{[]}}
\newcommand{\cal}{\mathcal}
%\newcommand{\fold}{\mathit{fold}}
\newcommand{\F}{\mathcal{F}}
\renewcommand{\G}{\mathcal{G}}
\newcommand{\N}{\mathcal{N}}
\newcommand{\E}{\mathcal{E}}
\newcommand{\B}{\mathcal{B}}
\renewcommand{\P}{\mathcal{A}}
\newcommand{\pred}{\mathsf{Fam}}
\newcommand{\env}{\mathsf{Env}}
\newcommand{\set}{\mathsf{Set}}
\renewcommand{\S}{\mathcal S}
\renewcommand{\C}{\mathcal{C}}
\newcommand{\D}{\mathcal{D}}
\newcommand{\A}{\mathcal{A}}
\renewcommand{\id}{\mathit{id}}
\newcommand{\map}{\mathsf{map}}
\newcommand{\pid}{\underline{\mathit{id}}}
\newcommand{\pcirc}{\,\underline{\circ}\,}
\newcommand{\pzero}{\underline{0}}
\newcommand{\pone}{\underline{1}}
\newcommand{\psum}{\,\underline{+}\,}
\newcommand{\pinl}{\underline{\mathit{inL}}\,}
\newcommand{\pinr}{\underline{\mathit{inR}}\,}
\newcommand{\ptimes}{\,\underline{\times}\,}
\newcommand{\ppi}{\underline{\pi_1}}
\newcommand{\pppi}{\underline{\pi_2}}
\newcommand{\pmu}{\underline{\mu}}
\newcommand{\semmap}{\mathit{map}}
\newcommand{\subst}{\mathit{subst}}

\newcommand{\tb}[1]{~~ \mbox{#1} ~~}
\newcommand{\listt}[1]{(\mu \phi. \lambda \beta . \onet + \beta \times
  \phi \beta) #1} 
\newcommand{\filtype}{\Nat^\emptyset 
 (\Nat^\emptyset \, \alpha \, \mathit{Bool})\, (\Nat^\emptyset 
  (List \, \alpha) \, (List \, \alpha))} 
\newcommand{\filtypeGRose}{\Nat^\emptyset 
 (\Nat^\emptyset \, \alpha \, \mathit{Bool})\, (\Nat^\emptyset 
  (\mathit{GRose}\,\psi \, \alpha) \, (\mathit{GRose}\,\psi \, (\alpha
  + \onet)))} 
\newcommand{\maplist}{\mathit{map}_{\lambda A. \setsem{\emptyset; \alpha
      \vdash \mathit{List} \, \alpha} \rho[\alpha := A]}} 
\newcommand{\PLeaves}{\mathsf{PLeaves}}
\newcommand{\swap}{\mathsf{swap}}
\newcommand{\reverse}{\mathsf{reverse}}
\newcommand{\Bcons}{\mathit{Bcons}}
\newcommand{\Bnil}{\mathit{Bnil}}

\title[Identity Extension Lemma (new Nat type)]{Identity Extension Lemma (new Nat type)} %% [Short Title] is optional; when present,
                         %% will be used in header instead of Full
                         %% Title.
%\titlenote{with title note}             %% \titlenote is optional;
                                        %% can be repeated if necessary;
                                        %% contents suppressed with 'anonymous'
%\subtitle{Subtitle}                     %% \subtitle is optional
%\subtitlenote{with subtitle note}       %% \subtitlenote is optional;
                                        %% can be repeated if necessary;
                                        %% contents suppressed with 'anonymous'


%% Author information
%% Contents and number of authors suppressed with 'anonymous'.
%% Each author should be introduced by \author, followed by
%% \authornote (optional), \orcid (optional), \affiliation, and
%% \email.
%% An author may have multiple affiliations and/or emails; repeat the
%% appropriate command.
%% Many elements are not rendered, but should be provided for metadata
%% extraction tools.

%% Author with single affiliation.
\author{Patricia Johann, Enrico Ghiorzi, and Daniel Jeffries}
%\authornote{with author1 note}          %% \authornote is optional;
%                                        %% can be repeated if necessary
%\orcid{nnnn-nnnn-nnnn-nnnn}             %% \orcid is optional
\affiliation{
%  \position{Position1}
%  \department{Department1}              %% \department is recommended
  \institution{Appalachian State University}            %% \institution is required
%  \streetaddress{Street1 Address1}
%  \city{City1}
%  \state{State1}
%  \postcode{Post-Code1}
%  \country{Country1}                    %% \country is recommended
}
\email{johannp@appstate.edu, ghiorzie@appstate.edu, jeffriesd@appstate.edu}          %% \email is recommended


\begin{document}

%\begin{CCSXML}
%<ccs2012>
%<concept>
%<concept_id>10011007.10011006.10011008</concept_id>
%<concept_desc>Software and its engineering~General programming languages</concept_desc>
%<concept_significance>500</concept_significance>
%</concept>
%<concept>
%<concept_id>10003456.10003457.10003521.10003525</concept_id>
%<concept_desc>Social and professional topics~History of programming languages</concept_desc>
%<concept_significance>300</concept_significance>
%</concept>
%</ccs2012>
%\end{CCSXML}
%
%\ccsdesc[500]{Software and its engineering~General programming languages}
%\ccsdesc[300]{Social and professional topics~History of programming languages}
%% End of generated code


%% Keywords
%% comma separated list
%\keywords{keyword1, keyword2, keyword3}  %% \keywords is optional


\maketitle



\begin{thm}\label{thm:iel}
  If $\rho$ is a set environment, and $\Gamma; \Phi \vdash \tau : \F$,
  then $\relsem{\Gamma; \Phi \vdash \tau} \Eq_\rho =
  \Eq_{\setsem{\Gamma; \Phi \vdash \tau}\rho}$.
\end{thm}
\begin{proof}
By induction on the structure of $\tau$.
\begin{itemize}
\item $\relsem{\Gamma; \emptyset \vdash v}\Eq_{\rho} = \Eq_{\rho} v =
  \Eq_{\rho v} = \Eq_{\setsem{\Gamma; \emptyset \vdash v}\rho}$ where
  $v \in \Gamma$.
%\item $\relsem{\Gamma; \emptyset \vdash \sigma \to \tau} \Eq_{\rho} =
%  \relsem{\Gamma; \emptyset \vdash \sigma} \Eq_{\rho} \to
%  \relsem{\Gamma; \emptyset \vdash \tau} \Eq_{\rho} =
%  \Eq_{\setsem{\Gamma; \emptyset \vdash \sigma}\rho} \to
%  \Eq_{\setsem{\Gamma; \emptyset \vdash \tau}\rho} =
%  \Eq_{\setsem{\Gamma; \emptyset \vdash \sigma \to \tau}\rho}$, where
%  the second equality is by the induction hypothesis.
%% \item {\color{red} $\tau = \forall v.\tau_1$}
\item {\color{red} CHANGED!} By definition, $\relsem{\Gamma; \Phi \vdash
  \Nat^{\overline\alpha} \,F\,G} \Eq_{\rho}$ is the relation on
  $\setsem{\Gamma; \Phi \vdash \Nat^{\overline\alpha} \,F\,G}
  \rho$ relating $t$ and $t'$ if, for all ${R_1 :
    \rel(A_1,B_1)},...,{R_k : \rel(A_k,B_k)}$, $(t_{\overline{A}},
  t'_{\overline{B}})$ is a morphism from $\relsem{\Gamma;
    \overline\alpha \vdash F} \Eq_{\rho}\overline{[\alpha := R]}$ to
  $\relsem{\Gamma ; \Phi, \overline\alpha \vdash G}
  \Eq_{\rho}\overline{[\alpha := R]}$ in $\rel$.  To prove that this
  is equal to $\Eq_{\setsem{\Gamma; \Phi \vdash
      \Nat^{\overline\alpha} \,F\,G} \rho}$ we need to show that
  $(t_{\overline{A}}, t'_{\overline{B}})$ is a morphism from
  $\relsem{\Gamma; \overline\alpha \vdash F}
  \Eq_{\rho}\overline{[\alpha := R]}$ to $\relsem{\Gamma ;
  \Phi, \overline\alpha \vdash G} \Eq_{\rho}\overline{[\alpha := R]}$ in
  $\rel$ for all ${R_1 : \rel(A_1,B_1)},...,{R_k : \rel(A_k,B_k)}$ if
  and only if $t = t'$ and $(t_{\overline{A}}, t_{\overline{B}})$ is a
  morphism from $\relsem{\Gamma; \overline\alpha \vdash F}
  \Eq_{\rho}\overline{[\alpha := R]}$ to $\relsem{\Gamma ; \Phi, \overline
    \alpha \vdash G} \Eq_{\rho}\overline{[\alpha := R]}$ in $\rel$ for
  all ${R_1 : \rel(A_1,B_1)}, ...,$ ${R_k : \rel(A_k,B_k)}$. The only
  interesting part of this equivalence is to show that if
  $(t_{\overline{A}}, t'_{\overline{B}})$ is a morphism from
  $\relsem{\Gamma; \overline\alpha \vdash F}
  \Eq_{\rho}\overline{[\alpha := R]}$ to $\relsem{\Gamma ; \Phi,
    \overline\alpha \vdash G} \Eq_{\rho}\overline{[\alpha := R]}$ in
  $\rel$ for all ${R_1 : \rel(A_1,B_1),}$ $...,{R_k : \rel(A_k,B_k)}$,
  then $t = t'$.  By hypothesis, $(t_{\overline{A}},
  t'_{\overline{A}})$ is a morphism from $\relsem{\Gamma;
    \overline\alpha \vdash F} \Eq_{\rho}\overline{[\alpha :=
      \Eq_{A}]}$ to $\relsem{\Gamma ; \Phi, \overline\alpha \vdash G}
  \Eq_{\rho}\overline{[\alpha := \Eq_{A}]}$ in $\rel$ for all
  $A_1\,...\,A_k : \set$. By the induction hypothesis, it is therefore
  a morphism from $\Eq_{\setsem{\Gamma; \overline\alpha \vdash F}
    \rho\overline{[\alpha := A]}}$ to $\Eq_{\setsem{\Gamma ; \Phi,
      \overline\alpha \vdash G} \rho\overline{[\alpha := A]}}$ in
  $\rel$. This means that, for every $x : \Eq_{\setsem{\Gamma;
      \overline\alpha \vdash F} \rho\overline{[\alpha := A]}}$,
  $t_{\overline{A}}x = t'_{\overline{A}}x$.  Then, by extensionality,
  $t = t'$.
\item $\relsem{\Gamma; \Phi \vdash \zerot} \Eq_{\rho} = 0_\rel =
  \Eq_{0_\set} = \Eq_{\setsem{\Gamma; \Phi \vdash \zerot}\rho}$
\item $\relsem{\Gamma; \Phi \vdash \onet} \Eq_{\rho} = 1_\rel =
  \Eq_{1_\set} = \Eq_{\setsem{\Gamma; \Phi \vdash \onet}\rho}$
\item The application case is proved by the following sequence of
  equalities, where the second equality is by the induction hypothesis
  and the definition of the relation environment $\Eq_\rho$, the third
  is by the definition of application of relation transformers from
  Definition~\ref{def:rel-transf}, and the fourth is by
  Lemma~\ref{lem:eq-reln-equalities}:
\[
\begin{split}
\relsem{\Gamma; \Phi \vdash \phi\ol{\tau}}\Eq_{\rho} &=
(\Eq_{\rho}\phi)\ol{\relsem{\Gamma; \Phi \vdash \tau}
\Eq_{\rho}}\\
&= \Eq_{\rho \phi}\, \ol{\Eq_{\setsem{\Gamma; \Phi \vdash \tau}
  \rho}}\\
&= (\Eq_{\rho \phi})^* \,\ol{\Eq_{\setsem{\Gamma; \Phi \vdash \tau}
  \rho}}\\
&= \Eq_{(\rho \phi) \,\ol{\setsem{\Gamma; \Phi \vdash \tau} \rho}}\\
%&= (\Eq_{\rho \phi})^* \ol{\setsem{\Gamma; \Phi \vdash \tau} \rho}\\
&= \Eq_{\setsem{\Gamma; \Phi \vdash \phi\ol{\tau}}\rho}
\end{split}
\]
\item The fixed point case is proven by the sequence of equalities
\[
\begin{split}
\relsem{\Gamma; \Phi \vdash (\mu \phi.\lambda
  \ol{\alpha}. H)\ol{\tau}}\Eq_{\rho} 
&=(\mu {T_{\Eq_{\rho}}}) \,\ol{\relsem{\Gamma; \Phi \vdash \tau}\Eq_{\rho}}\\ 
&= \colim{n \in \nat}{T^n_{\Eq_{\rho}} K_0}\, \ol{\relsem{\Gamma; \Phi
  \vdash \tau}\Eq_{\rho}}\\
%&= \colim{n \in \nat}{(T^n_{\Eq_{\rho}} K_0)}
%\ol{\Eq_{\setsem{\Gamma; \Phi \vdash \tau}\rho}}\\
&= \colim{n \in \nat}{ T^n_{\Eq_{\rho}} K_0 \,\ol{\Eq_{\setsem{\Gamma;
    \Phi \vdash \tau}\rho}}}\\
&= \colim{n \in \nat}{(\Eq_{(T^\set_\rho)^n K_0})^*
  \ol{\Eq_{\setsem{\Gamma; \Phi \vdash \tau}\rho}}}\\
&= \colim{n \in \nat}{\Eq_{(T^\set_\rho)^n K_0 \,\ol{\setsem{\Gamma;
        \Phi \vdash \tau}\rho}}}\\ 
&= \Eq_{\colim{n \in \nat}{ (T^\set_{\rho})^n K_0\,
    \ol{\setsem{\Gamma; \Phi \vdash \tau}\rho}}}\\
&= \Eq_{\setsem{\Gamma; \Phi \vdash (\mu \phi.\lambda
      \ol{\alpha}. H)\ol{\tau}}\rho}
\end{split}
\]
Here, the third equality is by induction hypothesis, the fifth is by
Lemma~\ref{lem:eq-reln-equalities} and the fourth equality is because,
for every $n \in \nat$, the following two statements can be proved by
simultaneous induction:
\begin{equation}\label{eq:iel-fix-point-intermediate1}
T^n_{\Eq_{\rho}} K_0\, \ol{\Eq_{\setsem{\Gamma; \Phi \vdash
      \tau}\rho}} = (\Eq_{(T^\set_\rho)^n K_0})^*
\ol{\Eq_{\setsem{\Gamma; \Phi \vdash \tau}\rho}}
\end{equation}
and
\begin{equation}\label{eq:iel-fix-point-intermediate2}
\begin{split}
  \relsem{\Gamma; \Phi, \phi, \ol{\alpha} \vdash H}
\Eq_{\rho} [\phi := 
 & T^{n}_{\Eq_{\rho}} K_0] \overline{[\alpha :=
    \Eq_{\setsem{\Gamma; \Phi \vdash \tau}\rho}]} \\
=\;\; & \relsem{\Gamma; \Phi, \phi, \ol{\alpha} \vdash H} \Eq_{\rho} [\phi
  := \Eq_{(T^\set_\rho)^n K_0}] \overline{[\alpha :=
    \Eq_{\setsem{\Gamma; \Phi \vdash \tau}\rho}]}
\end{split}
\end{equation}
We prove~\eqref{eq:iel-fix-point-intermediate1}.  The case $n=0$ is
trivial, because $T^0_{\Eq_{\rho}} K_0 = K_0$ and
$(T^\set_\rho)^0 K_0 = K_0$; the inductive step is
proved by the following sequence of equalities:
\[
\begin{split}
T^{n+1}_{\Eq_{\rho}} K_0\, \overline{\Eq_{\setsem{\Gamma; \Phi \vdash \tau}\rho}}
&= T^\rel_{\Eq_{\rho}} (T^{n}_{\Eq_{\rho}} K_0)
\overline{\Eq_{\setsem{\Gamma; \Phi \vdash \tau}\rho}} \\ 
&= \relsem{\Gamma; \Phi, \phi, \ol{\alpha} \vdash H} \Eq_{\rho} [\phi
  := T^{n}_{\Eq_{\rho}} K_0] \overline{[\alpha :=
    \Eq_{\setsem{\Gamma; \Phi \vdash \tau}\rho}]} \\ 
&= \relsem{\Gamma; \Phi, \phi, \ol{\alpha} \vdash H} \Eq_{\rho} [\phi
  := \Eq_{(T^\set_\rho)^n K_0}] \overline{[\alpha :=
    \Eq_{\setsem{\Gamma; \Phi \vdash \tau}\rho}]} \\ 
&= \relsem{\Gamma; \Phi, \phi, \ol{\alpha} \vdash H} \Eq_{\rho [\phi
    := (T^\set_\rho)^n K_0] \overline{[\alpha :=
      \setsem{\Gamma; \Phi \vdash \tau}\rho]}} \\ 
&= \Eq_{\setsem{\Gamma; \Phi, \phi, \ol{\alpha} \vdash H} \rho [\phi
    := (T^\set_\rho)^n K_0] \overline{[\alpha :=
      \setsem{\Gamma; \Phi \vdash \tau}\rho]}} \\ 
&= \Eq_{(T^\set_\rho)^{n+1} K_0 \overline{\setsem{\Gamma; \Phi
      \vdash \tau}\rho}} \\ 
&= (\Eq_{(T^\set_\rho)^{n+1} K_0})^*\, \overline{\Eq_{\setsem{\Gamma;
      \Phi \vdash \tau}\rho}} 
\end{split}
\]
Here, the third equality is by~\eqref{eq:iel-fix-point-intermediate2},
the fifth by the induction hypothesis on $H$, and the last by
Lemma~\ref{lem:eq-reln-equalities}.  We prove the induction step
of~\eqref{eq:iel-fix-point-intermediate2} by structural induction on
$H$: the only interesting case, though, is when $\phi$ is applied,
i.e., for $H = \phi \ol{\sigma}$, which is proved by the sequence of
equalities:
\[
\begin{split}
& \relsem{\Gamma; \Phi, \phi, \ol{\alpha} \vdash \phi
    \ol{\sigma}} \Eq_{\rho} [\phi := T^{n}_{\Eq_{\rho}} K_0]
  \overline{[\alpha := \Eq_{\setsem{\Gamma; \Phi \vdash \tau}\rho}]}
  \\
&= T^{n}_{\Eq_{\rho}} K_0\, \overline{\relsem{\Gamma; \Phi,
      \phi, \ol{\alpha} \vdash \sigma} \Eq_{\rho} [\phi :=
      T^{n}_{\Eq_{\rho}} K_0] \overline{[\alpha :=
        \Eq_{\setsem{\Gamma; \Phi \vdash \tau} \rho}]}} \\ 
&= T^{n}_{\Eq_{\rho}} K_0\, \overline{\relsem{\Gamma; \Phi,
      \phi, \ol{\alpha} \vdash \sigma} \Eq_{\rho} [\phi :=
      \Eq_{(T^\set_\rho)^{n} K_0}] \overline{[\alpha :=
        \Eq_{\setsem{\Gamma; \Phi \vdash \tau} \rho}]}} \\ 
&= T^{n}_{\Eq_{\rho}} K_0\, \overline{\relsem{\Gamma; \Phi,
      \phi, \ol{\alpha} \vdash \sigma} \Eq_{\rho [\phi := (T^\set_\rho)^{n}
        K_0] \overline{[\alpha := \setsem{\Gamma; \Phi \vdash
            \tau} \rho]}}} \\ 
&= T^{n}_{\Eq_{\rho}} K_0\, \overline{\Eq_{\setsem{\Gamma;
        \Phi, \phi, \ol{\alpha} \vdash \sigma} \rho [\phi :=
        (T^\set_\rho)^{n} K_0] \overline{[\alpha :=
          \setsem{\Gamma; \Phi \vdash \tau} \rho]}}} \\ 
&= (\Eq_{(T^\set_\rho)^{n} K_0})^* \,\overline{\Eq_{\setsem{\Gamma;
        \Phi, \phi, \ol{\alpha} \vdash \sigma} \rho [\phi :=
        (T^\set_\rho)^{n} K_0] \overline{[\alpha :=
          \setsem{\Gamma; \Phi \vdash \tau} \rho]}}} \\ 
&= (\Eq_{(T^\set_{\rho})^{n} K_0})^* \overline{\relsem{\Gamma;
      \Phi, \phi, \ol{\alpha} \vdash \sigma} \Eq_{\rho} [\phi :=
      \Eq_{(T^\set_{\rho})^{n} K_0}] \overline{[\alpha :=
        \Eq_{\setsem{\Gamma; \Phi \vdash \tau}\rho}]}} \\ 
&= \relsem{\Gamma; \Phi, \phi, \ol{\alpha} \vdash \phi \ol{\sigma}}
  \Eq_{\rho} [\phi := \Eq_{(T^\set_{\rho})^{n} K_0}]
  \overline{[\alpha := \Eq_{\setsem{\Gamma; \Phi \vdash \tau}\rho}]} 
\end{split}
\]
Here, the second equality is by the induction hypothesis
for~\eqref{eq:iel-fix-point-intermediate2} on the $\sigma$s, the
fourth is by the induction hypothesis for Theorem~\ref{thm:iel} on the
$\sigma$s, and the fifth is by the induction hypothesis on $n$
for~\eqref{eq:iel-fix-point-intermediate1}.
\item $\relsem{\Gamma; \Phi \vdash \sigma + \tau} \Eq_{\rho} =
  \relsem{\Gamma; \Phi \vdash \sigma} \Eq_{\rho} + \relsem{\Gamma;
    \Phi \vdash \tau} \Eq_{\rho} = \Eq_{\setsem{\Gamma; \Phi \vdash
      \sigma}\rho} + \Eq_{\setsem{\Gamma; \Phi \vdash \tau}\rho} =
  \Eq_{\setsem{\Gamma; \Phi \vdash \sigma}\rho + \setsem{\Gamma; \Phi
      \vdash \tau}\rho} = \Eq_{\setsem{\Gamma; \Phi \vdash \sigma +
      \tau}\rho}$
\item $\relsem{\Gamma; \Phi \vdash \sigma \times \tau} \Eq_{\rho} =
  \relsem{\Gamma; \Phi \vdash \sigma}\Eq_{\rho} \times \relsem{\Gamma;
    \Phi \vdash \tau}\Eq_{\rho} = \Eq_{\setsem{\Gamma; \Phi \vdash
      \sigma}\rho} \times \Eq_{\setsem{\Gamma; \Phi \vdash \tau}\rho}
  = \Eq_{\setsem{\Gamma; \Phi \vdash \sigma}\rho \times
    \setsem{\Gamma; \Phi \vdash \tau}\rho} = \Eq_{\setsem{\Gamma;
      \Phi \vdash \sigma \times \tau}\rho}$
\end{itemize}
\end{proof}

\end{document}


  
