% For double-blind review submission, w/o CCS and ACM Reference (max
% submission space)
\documentclass[acmsmall,review,anonymous]{acmart}
\settopmatter{printfolios=true,printccs=false,printacmref=false}
%% For double-blind review submission, w/ CCS and ACM Reference
%\documentclass[acmsmall,review,anonymous]{acmart}\settopmatter{printfolios=true}
%% For single-blind review submission, w/o CCS and ACM Reference (max submission space)
%\documentclass[acmsmall,review]{acmart}\settopmatter{printfolios=true,printccs=false,printacmref=false}
%% For single-blind review submission, w/ CCS and ACM Reference
%\documentclass[acmsmall,review]{acmart}\settopmatter{printfolios=true}
%% For final camera-ready submission, w/ required CCS and ACM Reference
%\documentclass[acmsmall]{acmart}\settopmatter{}

%% Journal information
%% Supplied to authors by publisher for camera-ready submission;
%% use defaults for review submission.
\acmJournal{PACMPL}
\acmVolume{1}
\acmNumber{POPL} % CONF = POPL or ICFP or OOPSLA
\acmArticle{1}
\acmYear{2020}
\acmMonth{1}
\acmDOI{} % \acmDOI{10.1145/nnnnnnn.nnnnnnn}
\startPage{1}

%% Copyright information
%% Supplied to authors (based on authors' rights management selection;
%% see authors.acm.org) by publisher for camera-ready submission;
%% use 'none' for review submission.
\setcopyright{none}
%\setcopyright{acmcopyright}
%\setcopyright{acmlicensed}
%\setcopyright{rightsretained}
%\copyrightyear{2018}           %% If different from \acmYear

%% Bibliography style
\bibliographystyle{ACM-Reference-Format}
%% Citation style
%% Note: author/year citations are required for papers published as an
%% issue of PACMPL.
\citestyle{acmauthoryear}   %% For author/year citations
%\citestyle{acmnumeric}

%%%%%%%%%%%%%%%%%%%%%%%%%%%%%%%%%%%%%%%%%%%%%%%%%%%%%%%%%%%%%%%%%%%%%%
%% Note: Authors migrating a paper from PACMPL format to traditional
%% SIGPLAN proceedings format must update the '\documentclass' and
%% topmatter commands above; see 'acmart-sigplanproc-template.tex'.
%%%%%%%%%%%%%%%%%%%%%%%%%%%%%%%%%%%%%%%%%%%%%%%%%%%%%%%%%%%%%%%%%%%%%%



\usepackage[utf8]{inputenc}
\usepackage{ccicons}

\usepackage{amsmath}
\usepackage{amsthm}
\usepackage{amscd}
%\usepackage{MnSymbol}
\usepackage{xcolor}

\usepackage{bbold}
\usepackage{url}
\usepackage{upgreek}
%\usepackage{stmaryrd}

\usepackage{lipsum}
\usepackage{tikz-cd}
\usetikzlibrary{cd}
\usetikzlibrary{calc}
\usetikzlibrary{arrows}

\usepackage{bussproofs}
\EnableBpAbbreviations

\DeclareMathAlphabet{\mathpzc}{OT1}{pzc}{m}{it}

%\usepackage[amsmath]{ntheorem}

\newcommand{\lam}{\lambda}
\newcommand{\eps}{\varepsilon}
\newcommand{\ups}{\upsilon}
\newcommand{\mcB}{\mathcal{B}}
\newcommand{\mcD}{\mathcal{D}}
\newcommand{\mcE}{\mathcal{E}}
\newcommand{\mcF}{\mathcal{F}}
\newcommand{\mcP}{\mathcal{P}}
\newcommand{\mcI}{\mathcal{I}}
\newcommand{\mcJ}{\mathcal{J}}
\newcommand{\mcK}{\mathcal{K}}
\newcommand{\mcL}{\mathcal{L}}
\newcommand{\WW}{\mathcal{W}}

\newcommand{\ex}{\mcE_x}
\newcommand{\ey}{\mcE_y}
\newcommand{\bzero}{\boldsymbol{0}}
\newcommand{\bone}{{\boldsymbol{1}}}
\newcommand{\tB}{{\bone_\mcB}}
\newcommand{\tE}{{\bone_\mcE}}
\newcommand{\bt}{\mathbf{t}}
\newcommand{\bp}{\mathbf{p}}
\newcommand{\bsig}{\mathbf{\Sigma}}
\newcommand{\bpi}{\boldsymbol{\pi}}
\newcommand{\Empty}{\mathtt{Empty}}
\newcommand{\truthf}{\mathtt{t}}
\newcommand{\id}{id}
\newcommand{\coo}{\mathtt{coo\ }}
\newcommand{\mcC}{\mathcal{C}}
\newcommand{\Rec}{\mathpzc{Rec}}
\newcommand{\types}{\mathcal{T}}



%\newcommand{\semof}[1]{\llbracket{#1}\rrbracket^\rel}
\newcommand{\sem}[1]{\llbracket{#1}\rrbracket}
\newcommand{\setsem}[1]{\llbracket{#1}\rrbracket^\set}
\newcommand{\relsem}[1]{\llbracket{#1}\rrbracket^\rel}
\newcommand{\dsem}[1]{\llbracket{#1}\rrbracket^{\mathsf D}}
\newcommand{\setenv}{\mathsf{SetEnv}}
\newcommand{\relenv}{\mathsf{RelEnv}}
\newcommand{\denv}{\mathsf{DEnv}}

\newcommand{\rel}{\mathsf{Rel}}
\newcommand{\setof}[1]{\{#1\}}
\newcommand{\letin}[1]{\texttt{let }#1\texttt{ in }}
\newcommand{\comp}[1]{{\{#1\}}}
\newcommand{\bcomp}[1]{\{\![#1]\!\}}
\newcommand{\beps}{\boldsymbol{\eps}}
%\newcommand{\B}{\mcB}
%\newcommand{\Bo}{{|\mcB|}}

\newcommand{\lmt}{\longmapsto}
\newcommand{\RA}{\Rightarrow}
\newcommand{\LA}{\Leftarrow}
\newcommand{\rras}{\rightrightarrows}
\newcommand{\colim}[2]{{{\underrightarrow{\lim}}_{#1}{#2}}}
\newcommand{\lift}[1]{{#1}\,{\hat{} \; \hat{}}}
\newcommand{\graph}[1]{\langle {#1} \rangle}

\newcommand{\carAT}{\mathsf{car}({\mathcal A}^T)}
\newcommand{\isoAto}{\mathsf{Iso}({\mcA^\to})}
\newcommand{\falg}{\mathsf{Alg}_F}
\newcommand{\CC}{\mathsf{Pres}(\mathcal{A})}
\newcommand{\PP}{\mathcal{P}}
\newcommand{\DD}{D_{(A,B,f)}}
\newcommand{\from}{\leftarrow}
\newcommand{\upset}[1]{{#1}{\uparrow}}
\newcommand{\smupset}[1]{{#1}\!\uparrow}

\newcommand{\Coo}{\mathpzc{Coo}}
\newcommand{\code}{\#}
\newcommand{\nat}{\mathpzc{Nat}}

\newcommand{\eq}{\; = \;}
\newcommand{\of}{\; : \;}
\newcommand{\df}{\; := \;}
\newcommand{\bnf}{\; ::= \;}

\newcommand{\zmap}[1]{{\!{\between\!\!}_{#1}\!}}
\newcommand{\bSet}{\mathbf{Set}}

\newcommand{\dom}{\mathsf{dom}}
\newcommand{\cod}{\mathsf{cod}}
\newcommand{\adjoint}[2]{\mathrel{\mathop{\leftrightarrows}^{#1}_{#2}}}
\newcommand{\isofunc}{\mathpzc{Iso}}
\newcommand{\ebang}{{\eta_!}}
\newcommand{\lras}{\leftrightarrows}
\newcommand{\rlas}{\rightleftarrows}
\newcommand{\then}{\quad\Longrightarrow\quad}
\newcommand{\hookup}{\hookrightarrow}

\newcommand{\spanme}[5]{\begin{CD} #1 @<#2<< #3 @>#4>> #5 \end{CD}}
\newcommand{\spanm}[3]{\begin{CD} #1 @>#2>> #3\end{CD}}
\newcommand{\pushout}{\textsf{Pushout}}
\newcommand{\mospace}{\qquad\qquad\!\!\!\!}

\newcommand{\natur}[2]{#1 \propto #2}

\newcommand{\Tree}{\mathsf{Tree}\,}
\newcommand{\GRose}{\mathsf{GRose}\,}
\newcommand{\List}{\mathsf{List}\,}
\newcommand{\PTree}{\mathsf{PTree}\,}
\newcommand{\Bush}{\mathsf{Bush}\,}
\newcommand{\Forest}{\mathsf{Forest}\,}
\newcommand{\Lam}{\mathsf{Lam}\,}
\newcommand{\LamES}{\mathsf{Lam}^+}
\newcommand{\Expr}{\mathsf{Expr}\,}

\newcommand{\ListNil}{\mathsf{Nil}}
\newcommand{\ListCons}{\mathsf{Cons}}
\newcommand{\LamVar}{\mathsf{Var}}
\newcommand{\LamApp}{\mathsf{App}}
\newcommand{\LamAbs}{\mathsf{Abs}}
\newcommand{\LamSub}{\mathsf{Sub}}
\newcommand{\ExprConst}{\mathsf{Const}}
\newcommand{\ExprPair}{\mathsf{Pair}}
\newcommand{\ExprProj}{\mathsf{Proj}}
\newcommand{\ExprAbs}{\mathsf{Abs}}
\newcommand{\ExprApp}{\mathsf{App}}
\newcommand{\Ptree}{\mathsf{Ptree}}

\newcommand{\kinds}{\mathpzc{K}}
\newcommand{\tvars}{\mathbb{T}}
\newcommand{\fvars}{\mathbb{F}}
\newcommand{\consts}{\mathpzc{C}}
\newcommand{\Lan}{\mathsf{Lan}}
\newcommand{\zerot}{\mathbb{0}}
\newcommand{\onet}{\mathbb{1}}
\newcommand{\bool}{\mathbb{2}}
\renewcommand{\nat}{\mathbb{N}}
%\newcommand{\semof}[1]{[\![#1]\!]}
%\newcommand{\setsem}[1]{\llbracket{#1}\rrbracket^\set}
\newcommand{\predsem}[1]{\llbracket{#1}\rrbracket^\pred}
%\newcommand{\todot}{\stackrel{.}{\to}}
\newcommand{\todot}{\Rightarrow}
\newcommand{\bphi}{{\bm \phi}}

\newcommand{\bm}[1]{\boldsymbol{#1}}

\newcommand{\cL}{\mathcal{L}}
\newcommand{\T}{\mathcal{T}}
\newcommand{\Pos}{P\!}
%\newcommand{\Pos}{\mathcal{P}\!}
\newcommand{\Neg}{\mathcal{N}}
\newcommand{\Hf}{\mathcal{H}}
\newcommand{\V}{\mathbb{V}}
\newcommand{\I}{\mathcal{I}}
\newcommand{\Set}{\mathsf{Set}}
\newcommand{\Nat}{\mathsf{Nat}}
\newcommand{\Homrel}{\mathsf{Hom_{Rel}}}
\newcommand{\CV}{\mathcal{CV}}
\newcommand{\lan}{\mathsf{Lan}}
\newcommand{\Id}{\mathit{Id}}
\newcommand{\mcA}{\mathcal{A}}
\newcommand{\inl}{\mathsf{inl}}
\newcommand{\inr}{\mathsf{inr}}
\newcommand{\case}[3]{\mathsf{case}\,{#1}\,\mathsf{of}\,\{{#2};\,{#3}\}}
\newcommand{\tin}{\mathsf{in}}
\newcommand{\fold}{\mathsf{fold}}
\newcommand{\Eq}{\mathsf{Eq}}
\newcommand{\Hom}{\mathsf{Hom}}
\newcommand{\curry}{\mathsf{curry}}
\newcommand{\uncurry}{\mathsf{uncurry}}
\newcommand{\eval}{\mathsf{eval}}
\newcommand{\apply}{\mathsf{apply}}

\newcommand{\ar}[1]{\##1}
\newcommand{\mcG}{\mathcal{G}}
\newcommand{\mcH}{\mathcal{H}}
\newcommand{\TV}{\mathpzc{V}}

\newcommand{\essim}[1]{\mathsf{EssIm}(#1)}
\newcommand{\hra}{\hookrightarrow}

\newcommand{\ol}[1]{\overline{#1}}
\newcommand{\ul}[1]{\underline{#1}}
\newcommand{\op}{\mathsf{op}}
\newcommand{\trige}{\trianglerighteq}
\newcommand{\trile}{\trianglelefteq}
\newcommand{\LFP}{\mathsf{LFP}}
\newcommand{\LAN}{\mathsf{Lan}}
%\newcommand{\Mu}{{\mu\hskip-4pt\mu}}
\newcommand{\Mu}{{\mu\hskip-5.5pt\mu}}
%\newcommand{\Mu}{\boldsymbol{\upmu}}
\newcommand{\Terms}{\mathpzc{Terms}}
\newcommand{\Ord}{\mathpzc{Ord}}
\newcommand{\Anote}[1]{{\color{blue} {#1}}}
\newcommand{\Pnote}[1]{{\color{red} {#1}}}

\newcommand{\greyout}[1]{{\color{gray} {#1}}}
\newcommand{\ora}[1]{\overrightarrow{#1}}

%\newcommand{\?}{{.\ }}
%\theoremheaderfont{\scshape}
%\theorembodyfont{\normalfont}
%\theoremseparator{.\ \ }
\newtheorem{thm}{Theorem}
\newtheorem{dfn}[thm]{Definition}
\newtheorem{prop}[thm]{Proposition}
\newtheorem{cor}[thm]{Corollary}
\newtheorem{lemma}[thm]{Lemma}
\newtheorem{rmk}[thm]{Remark}
\newtheorem{expl}[thm]{Example}
\newtheorem{notn}[thm]{Notation}
%\theoremstyle{nonumberplain}
%\theoremsymbol{\Box}


\theoremstyle{definition}
\newtheorem{exmpl}{Example}

\renewcommand{\greyout}[1]{} %{{\color{gray} {#1}}} -- toggle to remove greyed text

\newcommand{\emptyfun}{{[]}}
\newcommand{\cal}{\mathcal}
%\newcommand{\fold}{\mathit{fold}}
\newcommand{\F}{\mathcal{F}}
\renewcommand{\G}{\mathcal{G}}
\newcommand{\N}{\mathcal{N}}
\newcommand{\E}{\mathcal{E}}
\newcommand{\B}{\mathcal{B}}
\renewcommand{\P}{\mathcal{A}}
\newcommand{\pred}{\mathsf{Fam}}
\newcommand{\env}{\mathsf{Env}}
\newcommand{\set}{\mathsf{Set}}
\renewcommand{\S}{\mathcal S}
\renewcommand{\C}{\mathcal{C}}
\newcommand{\D}{\mathcal{D}}
\newcommand{\A}{\mathcal{A}}
\renewcommand{\id}{\mathit{id}}
\newcommand{\map}{\mathsf{map}}
\newcommand{\pid}{\underline{\mathit{id}}}
\newcommand{\pcirc}{\,\underline{\circ}\,}
\newcommand{\pzero}{\underline{0}}
\newcommand{\pone}{\underline{1}}
\newcommand{\psum}{\,\underline{+}\,}
%\newcommand{\inl}{\mathit{inL}\,}
%\newcommand{\inr}{\mathit{inR}\,}
\newcommand{\pinl}{\underline{\mathit{inL}}\,}
\newcommand{\pinr}{\underline{\mathit{inR}}\,}
\newcommand{\ptimes}{\,\underline{\times}\,}
\newcommand{\ppi}{\underline{\pi_1}}
\newcommand{\pppi}{\underline{\pi_2}}
\newcommand{\pmu}{\underline{\mu}}

\title[Free Theorems for Nested Types]{Free Theorems for 
Nested Types} %% [Short Title] is optional;
                                        %% when present, will be used in
                                        %% header instead of Full Title.
%\titlenote{with title note}             %% \titlenote is optional;
                                        %% can be repeated if necessary;
                                        %% contents suppressed with 'anonymous'
%\subtitle{Subtitle}                     %% \subtitle is optional
%\subtitlenote{with subtitle note}       %% \subtitlenote is optional;
                                        %% can be repeated if necessary;
                                        %% contents suppressed with 'anonymous'


%% Author information
%% Contents and number of authors suppressed with 'anonymous'.
%% Each author should be introduced by \author, followed by
%% \authornote (optional), \orcid (optional), \affiliation, and
%% \email.
%% An author may have multiple affiliations and/or emails; repeat the
%% appropriate command.
%% Many elements are not rendered, but should be provided for metadata
%% extraction tools.

%% Author with single affiliation.
\author{Patricia Johann and Andrew Polonsky}
%\authornote{with author1 note}          %% \authornote is optional;
%                                        %% can be repeated if necessary
%\orcid{nnnn-nnnn-nnnn-nnnn}             %% \orcid is optional
\affiliation{
%  \position{Position1}
%  \department{Department1}              %% \department is recommended
  \institution{Appalachian State University}            %% \institution is required
%  \streetaddress{Street1 Address1}
%  \city{City1}
%  \state{State1}
%  \postcode{Post-Code1}
%  \country{Country1}                    %% \country is recommended
}
\email{johannp@appstate.edu, andrew.polonsky@gmail.com}          %% \email is recommended




\begin{document}
\newcommand{\BBush}{\mathit{B}\,}
\newcommand{\nilHelp}{\mathit{nilHelp}}
\newcommand{\consHelp}{\mathit{consHelp}}
\renewcommand{\F}[1]{\Nat^{#1} (\BBush #1)\, (\BBush (\beta \times #1))}
\section{Zip function for Bush datatype}

We can define a zip function for bushes with the following type

$$\vdash \Nat^\beta \, (\BBush \beta) \, (\Nat^\delta (\BBush \delta) \, 
(\BBush (\beta \times \delta)))$$

as a fold in our calculus. The unapplied fold will be formed as

\begin{align*}
\vdash \fold^{\F{\delta}}_{\onet + \alpha \times \phi(\phi\alpha)} : 
  \Nat^\emptyset &(\Nat^\beta (\onet + 
%  FFB
  \beta \times 
(\Nat^\delta \BBush \delta \, (\BBush ((\Nat^{\delta'} \BBush \delta'
\, \BBush (\beta \times \delta')) \times \delta ))
)
) \, (\F{\delta})) \\
  &(\Nat^\beta \BBush \beta \,\, (\F{\delta}))
\end{align*}

and the algebra components to which we will apply the fold have the 
following types:

$$\vdash \nilHelp : \Nat^\beta \onet \, (\F{\delta})$$
$$\vdash \consHelp : \Nat^\beta (\beta \times 
(\Nat^\delta \BBush \delta \, (\BBush ((\Nat^{\delta'} \BBush \delta'
\, \BBush (\beta \times \delta')) \times \delta ))))
\, (\F{\delta})$$

We first define $\nilHelp$ as a constant function which always returns an empty
bush:

$$\vdash L_\beta *. L_\delta \, b. bnil_{\beta\times\delta} * : 
\Nat^\beta \onet \, (\F{\delta})$$

\noindent \\
$\consHelp$ has a more involved definition. It takes two arguments (an element of
$\beta$, b,  and a natural transformation over bushes, r) and returns a natural transformation
from $\BBush \delta$ to $\BBush (\beta \times \delta)$. Since the return type is a natural
transformation, we can see $\consHelp$ as taking three arguments and returning
an element of $\BBush (\beta \times \delta)$. 

We define $\consHelp$ by pattern matching on the third argument. If this bush is empty,
we return an empty bush. Otherwise we pair the head element of the third argument
with the first argument and combine this pair with the recursively processed tail:

\newcommand{\bd}{bd}
\begin{align*}
  \vdash L_\beta \, (b , r). 
  &L_\delta \, \bd. \, 
  case \, ((\tin^{-1}_{\onet + \alpha \times \phi(\phi(\alpha))})_\delta \, \bd)\, \mathsf{of}  \\
  &\{  
  * \mapsto bnil_{\beta\times\delta} \, *; \\
  &(d , bbd) \mapsto bcons_{\beta \times \delta} 
  ((b, d), (\map^{(\F{\delta})\times\BBush\delta, \BBush (\beta \times \delta)}_{\BBush \alpha}
    \, app)_{\beta,\delta} (r_{(\BBush \delta)} \, bbd))\}
\end{align*}

where $ app : \Nat^{\beta, \delta} ((\Nat^{\delta'} \BBush \delta' \, \BBush (\beta \times \delta')) \times \BBush \delta)
\, (\BBush(\beta\times\delta))$
is the following application function:
$$\vdash L_{\beta,\delta} \, (\eta , \bd). \eta_{\delta} \, \bd$$

We can now define the zip function for bushes as 
\begin{align*}
  \vdash (\fold^{\F{\delta}}_{\onet + \alpha \times \phi(\phi\alpha)})_\emptyset \,
  alg :
  (\Nat^\beta (\BBush \beta) \,\, (\F{\delta}))
\end{align*}

where $alg : \Nat^\beta (\onet + 
%  F F (B) 
  \beta \times 
(\Nat^\delta \BBush \delta \, (\BBush ((\Nat^{\delta'} \BBush \delta'
\, \BBush (\beta \times \delta')) \times \delta ))
)
) \, (\F{\delta})$ is the sum of the algebra components:
$$\vdash L_{\beta} \, s. \case{(s)}{* \mapsto \nilHelp_\beta \, *}{(b, r) \mapsto \consHelp_\beta \, (b, r)}$$

\section{Type check}
To see that $\consHelp$ is well-typed, first observe that $(r_{(\BBush \delta)} \, bbd)$
has type $\emptyset; \beta, \delta \vdash \BBush ((\Nat^{\delta'} \BBush \delta' \, \BBush(\beta \times \delta')) \times 
\BBush \delta))$. 
Using the map rule, we can derive

\[\begin{array}{c}
  \AXC{$\emptyset ; \alpha, \beta, \delta \vdash \BBush \alpha$}
  \AXC{$\emptyset ; \beta, \delta \vdash (\F{\delta'}) \times \BBush \delta $}
  \AXC{$\emptyset ; \beta, \delta \vdash \BBush (\beta \times \delta) $}
  \AXC{$\vdash app : \Nat^{\beta, \delta} ((\Nat^{\delta'} \BBush \delta' \, \BBush (\beta \times \delta')) \times \BBush \delta)
  \, (\BBush(\beta\times\delta))$}
  \QIC{$\vdash \map
  % ^{(\F{\delta'})\times\BBush\delta,\BBush(\beta\times\delta)}
  ^{...}
  _{\BBush \alpha} 
  \, app : \Nat^{\beta,\delta} (\BBush (\F{\delta'} \times \BBush\delta)) (\BBush(\BBush(\beta\times\delta)))
  $}
  \DP
\end{array}\]

and we can apply this result to $(r_{(\BBush \delta)} bbd)$ to get (omitting the type of $r$ for space)
$$\emptyset; \beta, \delta \,|\, b : \beta, r : -\,, \bd : \BBush \delta, d : \delta, bbd : \BBush (\BBush \delta) \vdash (\map^{...}_{\BBush \alpha} \, app)_{\beta,\delta} \,
(r_{(\BBush \delta)} \, bbd) : \BBush(\BBush(\beta \times \delta))$$
Applying $bcons$ to this last result gives 
$$\emptyset; \beta, \delta \,|\, b : \beta, r : -\,, \bd : \BBush \delta, d : \delta, bbd : \BBush (\BBush \delta) \vdash
bcons_{\beta \times \delta} \, 
((b, d), (\map^{...}_{\BBush \alpha} \, app)_{\beta,\delta} \,
(r_{(\BBush \delta)} \, bbd)) : \BBush (\beta \times \delta)$$

We can now show the case term has the correct type. Let $R$ stand for the 
last term above, let $\Delta = b : \beta, r : -\,, \bd : \BBush \delta$ and
let $bd' = (\tin^{-1}_{\onet + \alpha \times \phi(\phi(\alpha))})_\delta \, \bd$

\[\begin{array}{c}
  \AXC{$\emptyset; \beta, \delta \,|\, \Delta, * : \onet \vdash bnil_{\beta\times\delta} \, * : \BBush (\beta \times \delta)$}
  \AXC{$\emptyset; \beta, \delta \,|\, \Delta, d : \delta , bbd : \BBush (\BBush \delta) 
        \vdash R : \BBush (\beta \times \delta)$}
  \AXC{$\emptyset; \beta, \delta \,|\, \Delta \vdash bd'
  : \onet + \delta \times \BBush (\BBush \delta) $}
  \TIC{$ \emptyset; \beta, \delta \,|\, \Delta \vdash \case{bd'}
  {* \mapsto bnil_{\beta \times \delta}\, *}
  {(d, bbd) \mapsto R } : \BBush (\beta \times \delta) $}
  \DP
\end{array}\]
% note abuse of notation (x, y) \mapsto ... isn't really valid because (x, y)
% is supposed to be a variable. just didn't want to introduce a bunch of \pi's 
% when the meaning is clear

\noindent
Let $C$ stand for the entire $case$ term above.  Now we can derive
\[\begin{array}{c}
  \AXC{$\emptyset; \delta \vdash \BBush \delta$}
  \AXC{$\emptyset; \beta, \delta \vdash \BBush (\beta \times \delta)$}
  \AXC{$\emptyset; \beta, \delta \,|\, b : \beta, r : -,\, \bd: \BBush\delta 
          \vdash C : \BBush (\beta \times \delta) $}
  \TIC{$ \emptyset; \beta \,|\, b : \beta, r : - \vdash 
    L_\delta \, \bd. C : \Nat^\delta (\BBush \delta) \, (\BBush (\beta \times \delta))
  $}
  \DP
\end{array}\]

and finally, if $F = \beta \times 
(\Nat^\delta \BBush \delta \, (\BBush ((\Nat^{\delta'} \BBush \delta'
\, \BBush (\beta \times \delta')) \times \delta )))$

\[\begin{array}{c}
  \AXC{$\emptyset; \beta \vdash F
  $}
  \AXC{$\emptyset; \beta \vdash \F{\delta}$}
  \AXC{$\emptyset; \beta \,|\, b : \beta, r : - \vdash 
          L_\delta \, \bd. C : \F{\delta} $}
  \TIC{$ \vdash L_\beta \, (b, r). \, 
    L_\delta \, \bd. C : \Nat^\beta F \, (\F{\delta})
  $}
  \DP
\end{array}\]


\end{document}
